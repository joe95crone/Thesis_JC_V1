%%%%%%%%%%%%%%%%%%%%%%%%%%%%%%%%%%%%%%%%%%%%%%%%%%%%%%%%%%
%
% Doctoral Thesis Template @ The University of Manchester
% LaTeX Chapter Template
% Version 1 (23/07/2020)
% Joe Crone
%
% This template is based on:
% The University of Manchester, Presentation of Thesis Policy
% Research Office Graduate Education Team
% June 2017
% http://www.regulations.manchester.ac.uk/pgr-presentation-theses/
%
%%%%%%%%%%%%%%%%%%%%%%%%%%%%%%%%%%%%%%%%%%%%%%%%%%%%%%%%%%
\documentclass[../main.tex]{subfiles}
\begin{document}

% Title
%--------------------------------------------------------
\chapter{FEBE Inverse Compton Scattering Experiment}
\label{FEBE_Inverse_Compton_Scattering_Experiment} % to reference use \ref{ChapterTemplate}

\section{Motivation for a FEBE ICS Experiment}
\textcolor{blue}{**TRIED TO WRITE WELL BUT MORE GETTING IDEAS OUT OF MY HEAD**}

An inverse Compton scattering experiment performed using the full energy beam exploitation (FEBE) mode of the compact linear accelerator for research and applications (CLARA) utilising the planned \si{\tera\watt}-scale high power laser could act as a preliminary demonstration of inverse Compton scattering within the $\gamma$-ray regime. A FEBE ICS experiment would allow for experimental validation of the analytical formula for collimated flux of an ICS source (Eq.~\ref{eq:collimated_flux}), as well as the \textsc{ICARUS} spectrum code and optimisation methods outlined in Chapter~\ref{Optimisation_and_Characterisation_of_Inverse_Compton Scattering_Spectra}. Furthermore, a $\gamma$-ray ICS experiment at FEBE may be the first demonstration of \si{\mega\electronvolt}-scale photon production by inverse Compton scattering in Europe.

A range of experimentation is possible from such a modest gamma-ray source such as stochasticity measurements of inverse Compton scattering and observation of non-linear inverse Compton scattering phenomena such as ponderomotive broadening. Ponderomotive broadening has previously only been tested against theory \cite{krafft2004spectral} with the ELBE \cite{} \textcolor{blue}{reference? ask Geoff + Balsa!}. Moreover, extension of the laser parameters presented here to the full CLARA specification \cite{} \textcolor{blue}{need to find a reference for these}, will enable experiments to focus on higher order non-linear behaviours such as harmonic generation which could be used to experimentally benchmark recently developed non-linear spectrum codes \cite{terzic2019improving,terzic2021laser}.

Technical upgrades focused on the increase in flux of the source such as use of a Fabry--Perot optical re-circulator cavity could be explored using the FEBE ICS. This could drive development of high average stored power Fabry--Perot cavities. Alternative approaches to flux enhancement such as crab cavity arrangements for both the electron bunch \cite{variola2011luminosity,koshiba2018luminosity} and laser pulse could also be investigated. Higher flux $\gamma$-ray source demonstrations at FEBE could lead to preliminary nuclear resonance fluorescence and tomography experiments for imaging of waste streams and reactor studies; facilitating partnership with the UK nuclear sector. Ultimately, design and operation of a FEBE ICS experiment would be a technical proving ground for a high average power $\gamma$-ray ICS source driven by an ERL - DIANA.

\section{Linac Electron Beam and Optical Cavity Laser Pulse Parameters}

\section{Spectral Output}

\section{Experimental Design and Layout}

\section{Future Experimental Opportunities}

\end{document}