%%%%%%%%%%%%%%%%%%%%%%%%%%%%%%%%%%%%%%%%%%%%%%%%%%%%%%%%%%
%
% Doctoral Thesis Template @ The University of Manchester
% LaTeX Chapter Template
% Version 1 (23/07/2020)
% Joe Crone
%
% This template is based on:
% The University of Manchester, Presentation of Thesis Policy
% Research Office Graduate Education Team
% June 2017
% http://www.regulations.manchester.ac.uk/pgr-presentation-theses/
%
%%%%%%%%%%%%%%%%%%%%%%%%%%%%%%%%%%%%%%%%%%%%%%%%%%%%%%%%%%
\documentclass[../main.tex]{subfiles}
\begin{document}

% Title
%--------------------------------------------------------
\chapter{Energy Recovery Linac Design}
\label{Energy_Recovery_Linac_Design} % to reference use \ref{ChapterTemplate}

\textcolor{blue}{**HYWEL SUGGESTS TO FOLLOW JAMES JONES THESIS, BUT WITH LESS DETAIL**}
\textcolor{blue}{https://www.research.manchester.ac.uk/portal/en/theses/design-of-a-novel-stacked-storage-ring-for-low-emittance-light-sources(53cedd16-ea4d-4343-abc9-78886a934a9b).html}

% Follow James Jones Thesis (minus some irrelevant parts) but include sections on ERLs and  Collective Effects

\section{Equations of Motion in Particle Accelerators}

\subsection{Co-ordinate System}
\textcolor{blue}{**DEFINE REFERENCE ORBIT**}

In particle accelerators bending fields are often used to direct the particles to a certain experimental station downstream of the particle sources, and many particle accelerators re-circulate particle beams. The particles are directed along an ideal trajectory named -- since in a recirculated machine the trajectory is closed -- a reference orbit. Therefore, since bending forces are introduced, the motion of particles within an accelerator environment can best be described by a right handed rotating co-moving co-ordinate system \cite{wille2000physics}. The right handed co-moving co-ordinate system is shown in Fig. \textcolor{blue}{**DIAGRAM OF CIRCULAR CO-ORD**}   

The co-ordinate system in Fig. \textcolor{blue}{**REF DIAGRAM COORD**} used to describe motion in particle accelerators within this thesis is of the form $\Kappa = \left\{x,y,s\right\}$, describing the local co-ordinate system where the transverse motion along the horizontal direction is in the $x$-axis and the vertical axis in $y$ is orthogonal to this and the direction of longitudinal motion $s$. Within this thesis, the $\Kappa = \left\{x,y,s\right\}$ co-ordinate system is maintained in situations with no bending magnets, such as linear accelerators.     

\subsection{Magnetic Fields in Particle Accelerators}

The motion of a charged particle in an electromagnetic field is given by the Lorentz force 
\begin{equation}
\overrightarrow{F} = q\left(\overrightarrow{E}+\overightarrow{v}\times\overrightarrow{B}\right),
\label{eq:Lorentz_force}    
\end{equation}
where $q$ is the charge of the particle with velocity vector $\overrightarrow{v}$, subject to an electric field $\overrightarrow{E}$ and magnetic field $\overrightarrow{B}$. When an elementary particle is subject solely to a magnetic field ($\overrightarrow{E}=0$), as common in an accelerator magnet, the Lorentz force (Eq.~\ref{eq:Lorentz_force}) can be re-cast into a more appropriate form using the displacement $x$ and time $t$
\begin{equation}
\frac{d^{2}\overrightarrow{x}}{dt^{2}} = \frac{e}{\gamma m}\left(\overrightarrow{v}\times\overrightarrow{B}\right)
\label{eq:displacement_Lorentz}    
\end{equation}
where $e$ is the elementary charge of the particle, $m$ is its rest mass and
\begin{equation}
\gamma = \frac{1}{\sqrt{1-\beta^{2}}} = \frac{E+m}{m},
\label{eq:Lorentz_factor}    
\end{equation}
is the Lorentz factor where $\beta = v/c$ is the Lorentz speed factor and $E$ is the kinetic energy of the particle. Note that, unless explicitly stated, the ultra-relativistic approximation $pc = \sqrt{E^{2}-m^{2}c^{4}} \approx E$, valid for particles with $E \gg mc^{2}$ is used throughout this thesis as the particle energies involved consistently satisfy this approximation.   

Since this thesis is concerned with re -circulated accelerators, particles typically require bending within the horizontal plane. To illustrate particle motion in a bending system we consider a pure, homogenous vertical magnetic field $B_{y}$ produced by a magnet with infinite pole width acting upon the particle of momentum $p_{0}$ (in convenient units of \si{\electronvolt}$/c$) to provide a bending radius of $\rho$ given by
\begin{equation}
\frac{1}{\rho} = \frac{eB_{y}}{p_{0}} = k_{0},
\label{magnet_bending_radius}    
\end{equation}
where $k_{0}$ is the field strength of the magnet -- named a dipole magnet -- with no transverse co-ordinate dependence of the field. By convention of $n$th order magnets, we classify this as the zeroth-order magnet ($n=0$). The magnetic beam rigidity $B\rho$ can therefore be defined as
\begin{equation}
B\rho = \frac{p_{0}}{e},
\label{eq:magnetic_beam_rigidity}    
\end{equation}
and the bending angle $\alpha_{0}$ can be defined with reference to the magnetic beam rigidity
\begin{equation}
\alpha_{0} = \frac{\int B_{y} ds}{B\rho} = \frac{L_{eff}}{\rho}, 
\label{eq:dipole_bending_angle}    
\end{equation}
where the vertical magnetic field (dipole magnetic field) is integrated over the longitudinal distance traversed -- $L_{\mathrm{eff}}$ the effective longitudinal distance traverse by the particle in the field.

Accelerators are typically concerned with beams of particles -- an ensemble of particles -- which can be transversely distributed around the ideal reference orbit hence particles may be  displaced horizontally ($\Delta x$) or vertically ($\Delta y$) from the reference orbit, which can occur due to their natural divergence. Therefore, a restorative 'focusing' force is required to counter the natural divergence of these particles. Within accelerators a quadrupole magnet, a magnet with 4 alternating equidistant (from the pole centre), poles is typically used to provide the required focusing field, which takes the form of a transverse linearly varying magnetic field increasing in strength with distance from the pole centre. We term this a first order magnet ($n=1$). An azimuthal field gradient of the form $g = B_{\phi}/dr$ is pprovided, with $r$ the radial distance from the pole centre. The scalar potential of such a quadrupole field has the from
\begin{equation}
V = -gxy,
\label{eq:quadrupole_potential}    
\end{equation}
where the partial derivatives form the linearly varying magnetic fields
\begin{align}
B_{x} &= -\frac{\partial V}{\partial x} = gy, \nonumber\\
B_{y} &= -\frac{\partial C}{\partial y} = gx.
\end{align}
Therefore, using the same formalism as (Eq.~\ref{eq:dipole_bending_angle}), the focusing angle $\alpha_{1}$ of a quadrupole can be generalised to
\begin{equation}
\alpha_{1} = \frac{e}{p_{0}}grl_{\mathrm{quad}} = -k_{1}rl_{quad},
\label{eq:quadrupole_focusing_angle}    
\end{equation}
where $l_{\mathrm{quad}}$ is the length of the quadrupole magnet and the normalised quadrupole gradient $k_{1}$ becomes
\begin{equation}
k_{1} = \frac{e}{p_{0}}g.
\label{eq:quadrupole_normalised_gradient}
\end{equation}
Quadrupole focusing is analogous to focusing with an optical lens, however a lens focuses simultaneously in all transverse directions whereas a quadrupole that is focusing in the horizontal plane is defocusing in the vertical plane. In the standard defined right handed co-ordinate system \textcolor{blue}{**REFERENCE CO-ORD IMAGE**}, when $k_{1} > 0$, the quadrupole is horizontally focusing (vertically defocusing) and when $k_{1} < 0$, the quadrupole is vertically focusing (horizontally defocusing). Consequently, with the transverse variation in focusing behaviour, a quadrupole can not be considered a true 'magnetic lens' however, taking inspiration from ray optics, the focal length of a quadrupole $f_{\mathrm{quad}}$ can be described as
\begin{equation}
\frac{1}{f_{\mathrm{quad}}} = k_{1}l_{\mathrm{quad}}.
\label{eq:focal_length_quadrupole}    
\end{equation}
The similarities between an optical lens and a quadrupole are highlighted in Fig. \textcolor{blue}{**LENS/DOUBLET DIAGRAM**}, which shows a simple focusing defocusing scheme -- the quadrupole doublet. 

\textcolor{blue}{**Higher Order Magnets + Multipoles**}
The computation of the magnetic field of an arbitrary $n$th order magnet typically uses a multipole expansion of the scalar potential of the magnetic field, which for convenience is presented in cylindrical polar co-ordinates \cite{shepherd2016magnet}
\begin{equation}
V = \sum_{n=0}^{\infty} J_{n+1}r^{n+1}\cos\left[\left(n+1\right)\theta\theta\right]+K_{n+1}r^{n+1}\sin\left[\left(n+1\right)\theta\right],
\label{eq:multipole_scalar_potential}    
\end{equation}
where $n \geq 0$ is the order of the magnet, $\theta$ is the polar angle from the horizontal plane  $x$--$z$ plane, and the coefficients $J_{n+1}$ and $K_{n+1}$ relate to the geometry of the magnet and its normalised field strength, for example for a quadrupole magnet ($n=1$) $K_{2}=0$ means this is a skew quadrupole rotated 90\si{\degree} around the pole axis and $J_{2}=0$ relates to a normal quadrupole field. The corresponding magnetic flux density of the magnet can be calculated for the radial $B_{r}$ or polar angle $B_{\theta}$ case by    
\begin{align}
B_{r} &= -\frac{\partial V}{\partial r}, & B_{\theta} &= -\frac{\partial V}{\partial \theta},
\label{eq:multipole_magnetic_field_cylindrical}    
\end{align}
and similarly in Cartesian co-ordinates, with proper transformation the magnetic flux density in each transverse plane is
\begin{align}
B_{x} &= -\frac{\partial V}{\partial x}, & B_{y} &= -\frac{\partial V}{\partial y}.
\label{eq:multipole_magnetic_field_cartesian}
\end{align}
The normalised field gradient of an $n$th order magnet is generalised to
\begin{equation}
k_{n} = \frac{e}{p_{0}}\frac{\partial^{n} B_{y}}{\partial x^{n}}.
\label{eq:multipole_normalised_field_gradient}    
\end{equation}
For example, for a sextupole ($n=2$) magnet with 6 poles the magnetic flux density is given by
\begin{align}
B_{x} &= k_{2}xy, \nonumber \\
B_{y} &= \frac{k_{2}}{2}\left(x^{2}-y^{2}\right),
\label{eq:sextupole_magnetic_field}    
\end{align}
wiith the normalised field gradient given by 
\begin{equation}
k_{2} = \frac{e}{p_{0}}\frac{\partial^{2}B_{y}}{\partial x^{2}}.
\label{eq:sextupole_field_gradient}    
\end{equation}

A magnet may have more than a single order field simultaneously. For example, a magnet utilising both linear fields (dipole and quadrupole fields) can be constructed -- named a combined function magnet -- which in this case would combine a dipole bending field component and a quadrupole focusing field component. A linear combined function magnet of can be achieved either by offsetting the pole centre of a quadrupole from the reference trajectory or by linearly varying the gap between magnetic poles transversely across the pole face of a dipole magnet. The former strategy is employed in the CBETA ERL FFA return loop, an accelerator explained in more detail in Chapter~\ref{CBETA_Multi-Pass_Commissioning}. The transverse magnetic field of a combined function dipole--quadrupole magnet, bending in the horizontal plane, is given by 
\begin{align}
B_{x} &= \frac{p_{0}}{e}k_{1}y, \nonumber\\
B_{y} &= \frac{p_{0}}{e}\left(k_{0}+k_{1}x\right).
\label{eq:combined_function_field}    
\end{align}

\subsection{Linear Equations of Motions}
\label{sec:equations_of_motion}

The linear equations of motion for a particle follow the derivation of Rossbach and Schmuser \cite{rossbach1993basic} and K. Willie \cite{wille2000physics}, using the co-ordinate system presented in Fig. \textcolor{blue}{**CO-ORD SYS DIAG**}. Here an equation of particle motion due to the Lorentz force (Eq.~\ref{eq:Lorentz_force}) is developed for the co-ordinate system which rotates due to bending forces, which we limit to the horizontal $x$--$s$ plane. We define the position vector of a particle at an arbitrary point in it's trajectory, in the global cyclindrical co-ordinate system $\left\{r,\alpha,z\right\}$ as
\begin{equation}
\boldsymbol{R} = \boldsymbol{R}_{0} + r\boldsymbol{u}_{r} + z\boldsymbol{u}_{z},    
\label{eq:particle_position_vector}
\end{equation}
where $R_{0}$ is the distance of the particle from an arbitrary origin point and $\boldsymbol{u}_{r}$, $\boldsymbol{u}_{\alpha}$, $\boldsymbol{u}_{z}$ are unit vectors describing the motion of the particle. Assuming a small variation in azimuthal angle $d\theta$, the relations between the unit vectors become
\begin{align}
\frac{d\boldsymbol{u}_{r}}{d\alpha} &= \boldsymbol{u}_{\alpha}, & \frac{d\boldsymbol{u}_{\alpha}}{d\alpha} &= -\boldsymbol{u}_{r}, & \frac{d\boldsymbol{u}_{z}}{d\alpha} &= 0.
\label{eq:unit_vector_angular_derivatives}    
\end{align}
The velocity of the particle is given by
\begin{align}
\frac{d\boldsymbol{R}}{dt} &= \frac{dr}{dt}\boldsymbol{u}_{r}+r\frac{d\boldsymbol{u}_{r}}{dt} +\frac{dz}{dt}\boldsymbol{u}_{z}, \nonumber \\
\frac{d\boldsymbol{R}}{dt} &= \frac{dr}{dt}\boldsymbol{u}_{r} + r\frac{d\alpha}{dt}\boldsymbol{u}_{\alpha} + \frac{dz}{dt}\boldsymbol{u}_{z},
\label{eq:unit_vector_velocity}    
\end{align}
and differentiating within respect to time, the acceleration becomes
\begin{equation}
\frac{d^{2}\boldsymbol{E}}{dt^{2}} = \left(\frac{d^{2}r}{dt^{2}}-r\frac{d^{2}\alpha}{dt^{2}}\right)\boldsymbol{u}_{r} + \left(2\frac{dr}{dt}\frac{d\alpha}{dt}\right)\boldsymbol{u}_{\alpha} + \frac{d^{2}z}{dt^{2}}\boldsymbol{u}_{z},
\label{eq:unit_vector_acceleration}    
\end{equation}
within the first $\boldsymbol{u}_{r}$ term, the first part describes the effect of the variation of bending on the acceleration and the second part relates to the centrifugal acceleration of the particle. The force of the particle of mass $m$, using the acceleration (Eq.~\ref{eq:unit_vector_acceleration}), can be equated with the Lorentz force (Eq.~\ref{eq:Lorentz_force}) of the magnetic field due to the accelerator magnets ($\overightarrow{E}=0$)
\begin{equation}
m\frac{d^{2}\boldsymbol{R}}{dt^{2}}=-e\left(\overightarrow{v}\times\overightarrow{B}\right),
\label{eq:Lorentz_particle_equating_forces}
\end{equation}
which, by describing the magnetic flux density vector in cylindrical co-ordinates $\overightarrow{B} = \left(B_{r},B_{\alpha},B_{z}\right)$, can be expanded to
\begin{equation}
m\frac{d^{2}\boldsymbol{R}}{dt^{2}} = -e\left[\left(r\frac{d\alpha}{dt}B_{z}-\frac{dz}{dt}B_{\alpha}\right)\boldsymbol{u}_{r}+\left(\frac{dz}{dt}B_{r}-\frac{dr}{dt}B_{z}\right)\boldsymbol{u}_{\alpha}+\left(\frac{dr}{dt}B_{\alpha}-r\frac{d\alpha}{dt}B_{r}\right)\boldsymbol{u}_{z}\right].
\label{eq:Lorentz_particle_equating_forces_expanded}    
\end{equation}
Assuming that $B_{\alpha}=0$, as there is no longitudinally applied magnetic field -- as this would be in the direction of motion of the particle -- we obtain a set of two differential equations
\begin{align}
m\left(\frac{d^{2}r}{dt^{2}}-r\frac{d^{2}\alpha}{dt^{2}}\right)=-er\frac{d\alpha}{dt}B_{z},
\label{eq:radial_differential_equation} \\
m\frac{d^{2}z}{dt^{2}}=-er\frac{d\alpha}{dt}B_{r}.
\label{eq:angular_differential_equation}
\end{align}
Assuming a combination of focusing and bending magnets within the accelerator, we approximate these fields by using a the field of a combined function magnet (Eq.~\ref{eq:combined_function_field}). The defined field of a combined function magnet (Eq.~\ref{eq:combined_function_field}) in the local co-ordinate system can be simply equated to the field in the global cylindrical co-ordinate system  because the radial direction of the global co-ordinate system is the same as the horizontal $x$ direction of the local co-ordinate system and the vertical $y$ local co-ordinate direction is analogous to the global $z$ direction. The equivalence of the fields in each co-ordinate system can be expressed as
\begin{align}
B_{r} = B_{x} &= \frac{p_{0}}{e}k_{1}y, \\
B_{z} = B_{y} &= \frac{p_{0}}{e}\left(k_{0}+k_{1}x\right),
\label{eq:equivalent_combined_function_field}
\end{align}
Substituting the combined function magnetic flux density (Eq.~\ref{eq:equivalent_combined_function_field}) into the differential equations (Eqs.~\ref{eq:radial_differential_equation}, \ref{eq:angula_differential_equation}) and making the substitution $r=\rho+x$ to account for varying transverse position around the reference orbit yields
\begin{align}
m\left(\frac{d^{2}x}{dt^{2}}-\left(p+x\right)\frac{d^{2}\alpha}{dt^{2}}\right) &=-p_{0}\left(\rho+x\right)\frac{d\alpha}{dt}\left(k_{0}+k_{1}\right), 
\label{eq:horizontal_differential_equation}\\   
m\frac{d^{2}z}{dt^{2}} &= -p_{0}\left(\rho+x\right)\frac{d\alpha}{dt}k_{1}z.
\label{eq:vertical_differential_equation}
\end{align}
The azimuthal component of the velocity has the form
\begin{equation}
v_{\alpha} = \left(\rho+x\right)\frac{d\alpha}{dt},
\label{eq:azimuthal_velocity}    
\end{equation}
which is typically much larger than the radial and vertical component ($v_{\alpha}\gg v_{r},~v_{z}$), which enables the approximation $v\approx v_{\alpha}$ that the azimuthal velocity is the velocity of the particle. Using this approximation a series of variables can be replaced using
\begin{align}
s&=vt, & \frac{d^{2}x}{dt^{2}} &= v^{2}\frac{d^{2}x}{ds^{2}},
\label{eq:velocity_approximation_replacing_variables}    
\end{align}
which can then be used to re-cast (Eqs.~\ref{eq:horizontal_differential_equation}, \ref{eq:vertical_differential_equation}) to obtain
\begin{align}
\frac{d^{2}x}{ds^{2}} &= -\frac{p_{0}}{mv}\frac{1}{\rho+x}\left(k_{0}+k_{1}x\right),
\label{eq:horizontal_recast_differential_equation} \\
\frac{d^{2}z}{ds^{2}} &= -\frac{p_{0}}{mv}k_{1}z.
\label{eq:vertical_recast_differential_equation}
\end{align}
Variation in the design momentum $\Delta p$ is introduced via
\begin{equation}
mv = p = p_{0}\left(1+\frac{\Delta p}{p_{0}}\right),
\label{eq:design_momentum_variation}    
\end{equation}
which means the transverse $r$ co-ordinate can be re-cast, when $\rho \gg x$, as
\begin{equation}
\frac{1}{r}\approx \frac{1}{\rho}\left(1-\frac{x}{\rho}\right). 
\label{eq:r_rho_relation}
\end{equation}
By introducing the momentum variation (Eq.~\ref{eq:design_momentum_variation}) and the approximation of the radial component (Eq.~\ref{eq:r_rho_relation}), the equations of motion can be shown in their familiar form \cite{rossbach1993basic,wille2000physics}
\begin{align}
\frac{d^{2}x}{ds^{2}} +\left(k_{1}+\frac{1}{\rho^{2}}\right)x &= \frac{1}{\rho}\frac{\Delta p}{p_{0}},
\label{eq:horizontal_equation_of_motion} \\
\frac{d^{2}z}{ds^{2}} - k_{1}z &= 0,
\label{eq:vertical_equation_of_motion}
\end{align}
where the focusing terms of the equations of motion, arising from the focusing component $k_{1}$ and the bending component $1/\rho^{2}$ can be combined into a single term
\begin{align}
\frac{d^{2}x}{ds^{2}} + K_{x}x &= \frac{1}{\rho}\frac{\Delta p}{p_{0}},
\label{eq:horizontal_equation_of_motion_simplified} \\
\frac{d^{2}z}{ds^{2}} + K_{z}z &= 0,
\label{eq:vertical_equation_of_motion_simplified}
\end{align}
where
\begin{align}
K_{x} &= k_{1} + \frac{1}{\rho^{2}}. \nonumber\\
K_{z} &= -k_{1}.
\label{eq:focusing_equation_of_motion}    
\end{align}
The equations of motion (Eqs.~\ref{eq:horizontal_equation_of_motion_simplified}, \ref{eq:vertical_equation_of_motion_simplified}) have the form of a harmonic oscillator.

Solutions to the equations of motion (Eqs.~\ref{eq:horizontal_equation_of_motion_simplified}, \ref{eq:vertical_equation_of_motion_simplified}) can therefore be found by linear combinations of sine and cosine terms which, for the focusing case ($K>0$) have the form
\begin{align}
C\left(s\right) &= \cos\left(\sqrt{K}s\right), \nonumber\\
S\left(s\right) &= \frac{1}{\sqrt{K}}\sin\left(\sqrt{K}s\right),
\label{eq:focusing_solution_equation_of_motion}
\end{align}
and for the defocusing case ($K<0$) become
\begin{align}
C\left(s\right) &= \cosh\left(\left|K\right|s\right), \nonumber \\
S\left(s\right) &= \frac{1}{\sqrt{\left|K\right|}}\sinh\left(\left|K\right|s\right).
\label{eq:defocusing_solution_equation_of_motion}    
\end{align}
Generalising for each plane $u\left(s\right)\equiv x\left(s\right) \parallel z\left(s\right)$ the solutions become
\begin{align}
u\left(s\right) &= aC\left(s\right) + bS\left(s\right) + \frac{\Delta p}{p_{0}}D\left(s\right), \nonumber \\
u'\left(s\right) &= aC'\left(s\right) + bS'\left(s\right) + \frac{\Delta p}{p_{0}}D\left(s\right),
\label{eq:generalised_solution}
\end{align}
where $D\left(s\right)$ is the dispersion function (see Section~\ref{sec:sec:dispersion_off_momentum_particles}), which accounts for the effect of momentum on particle trajectories and $a$, $b$ are coefficients related to the initial conditions of the particles. Commonly, for convenience, the solution (Eq.~\ref{eq:generalised_solution}) to the equations of motion is expressed in matrix formalism
\begin{equation}
\begin{pmatrix}
u\left(s\right) \\
u'\left(s\right) \\
\Delta p/p_{0}
\end{pmatrix} = 
\begin{pmatrix}
C\left(s\right) & S\left(s\right) & D\left(s\right) \\
C'\left(s\right) & S'\left(s\right) & D'\left(s\right) \\
0 & 0 & 1
\end{pmatrix}
\begin{pmatrix}
u\left(s_{0}\right) \\
u'\left(s_{0}\right) \\
\Delta p/p_{0}
\end{pmatrix},
\label{eq:general_solution_matrix}    
\end{equation}
where $s_{0}$ is the initial longitudinal position of the particle.

\section{Transport Matricies} 

Previously, the solution to the equations of motion within a periodic focusing and bending particle accelerator (Eqs.~\ref{eq:horizontal_equation_of_motion_simplified}, \ref{eq:vertical_equation_of_motion_simplified}) have been derived and presented using a matrix formalism (Eq.~\ref{eq:general_solution_matrix}). By considering the dynamics in both transverse planes ($x$ and $z$) simulataneously, the effect of momentum deviation from the design momentum $\Delta p/p_{0},$ as well as accounting for the longitudinal position of the particle $s=ct$ with $t$ the time-of-flight of the particle we see that a 6D co-ordinate system is required. Therefore, within a particle accelerator the position of a particle can be adequately described using a 6D vector 
\begin{equation}
\boldsymbol{X} = 
\begin{bmatrix}
x \\
x' \\
y \\
y' \\
ct \\
\Delta p/p
\end{bmatrix},
\label{eq:6D_vector}
\end{equation}
where $x$, $y$, $ct$ are the horizontal, vertical and longitudinal positions of the particle and $x'$, $y'$ are the divergence of the particle in either plane.

Within an accelerator, the particle is typically subjected to a series of magnet elements, such as dipoles, quadrupoles etc. the effect of which can be described overall by a $6~\times~6$ transform or transport matricies $\boldsymbol{R}$ to first order. With an initial particle vector $\boldsymbol{X}_{0}$ of the form (Eq.~\ref{eq:6D_vector}), the transport matrix can relate the initial state to a final state $\boldsymbol{X}$, where the particle has traversed a distance $s-s_{0}$ i.e
\begin{equation}
\boldsymbol{X}_{1} = \boldsymbol{R}\boldsymbol{X}_{0}.
\label{eq:overall_transport_matrix}
\end{equation}
The transport matrix is typically constructed from a series of magnetic elements and drift spaces, therefore the transport matrix has the form
\begin{equation}
\boldsymbol{R} = \boldsymbol{R}_{n}\boldsymbol{R}_{n-1}\ldots\boldsymbol{R_{1}},
\label{eq:subseries_transport_matrix}    
\end{equation}
where $R_{n}$ are the transport matricies of each individual element. A full series of magnetic elements is termed a beamline, where a subset of elements is named a cell -- cells can be repeated several times to form a periodic beamline. Within the following subsections, the individual transport matricies of linear magnets are described. 


\subsection{Drift Space}

A drift space within an accelerator is a section of beamline in which no magnets are placed -- the particles traverse through a vaccuum chamber subject to negligible external magnetic forces. As such, the beam is not bent or focused, therefore the position of the particles transforms only due to their divergence $u_{1} = u_{0} + L_{\mathrm{drift}}u_{0}'$, with $L_{\mathrm{drift}}$ the length of the drift space. The $\boldsymbol{R}$ matrix for a drift space is consequently given by  
\begin{equation}
\boldsymbol{R}_{\mathrm{drift}} =
\begin{pmatrix}
1 & L_{\mathrm{drift}} & 0 & 0 & 0 & 0 \\
0 & 1 & 0 & 0 & 0 & 0 \\
0 & 0 & 1 & L_{\mathrm{drift}} & 0 & 0 \\
0 & 0 & 0 & 1 & 0 & 0 \\
0 & 0 & 0 & 0 & 0 & 0 \\
0 & 0 & 0 & 0 & 0 & 1
\end{pmatrix}.
\label{eq:drift_matrix}    
\end{equation}

\subsection{Quadrupole Magnet}

The quadrupole element focuses a particle over a magnetic length $L_{\mathrm{quad}}$, which is identical to the width of the magnet for a hard-edged field model in which the field terminated at the end of the magnet yoke. The quadrupole element is focusing in one plane and defocusing in another plane, as shown in (Eqs.~\ref{eq:focusing_solution_equation_of_motion}, \ref{eq:defocusing_solution_equation_of_motion}), which is encapsulated within the $\boldsymbol{R}$ matrix of this element. The quadrupole has a normalised field strength $k_{1}$ (Eq.~\ref{eq:quadrupole_normalised_gradient}), which is used to characterise the focusing in this transport matrix. The $\boldsymbol{R}$ matrix for a focusing quadrupole is given by
\resizebox{\columnwidth}{!}{
\begin{equation}
\boldsymbol{R}_{\mathrm{quad}} = 
\begin{pmatrix}
\cos\left(\sqrt{k_{1}}L_{\mathrm{quad}}\right) & \frac{1}{\sqrt{k_{1}}}\sin\left(\sqrt{k_{1}}L_{\mathrm{quad}}\right) & 0 & 0 & 0 & 0 \\
-\sqrt{k_{1}}\sin\left(\sqrt{k_{1}}L_{\mathrm{quad}}\right) & \cos\left(\sqrt{k_{1}}L_{\mathrm{quad}}\right) & 0 & 0 & 0 & 0\\
0 & 0 & \cosh\left(\sqrt{k_{1}}L_{\mathrm{quad}}\right) & \frac{1}{\sqrt{k_{1}}}\sinh\left(\sqrt{k_{1}}L_{\mathrm{quad}}\right) & 0 & 0 \\
0 & 0 & \sqrt{k_{1}}\sinh\left(\sqrt{k_{1}}L_{\mathrm{quad}}\right) &  \cosh\left(\sqrt{k_{1}}L_{\mathrm{quad}}\right) & 0 & 0\\
0 & 0 & 0 & 0 & 1 & 0\\
0 & 0 & 0 & 0 & 0 & 1
\end{pmatrix}.
\label{eq:quadrupole_matrix_thick}    
\end{equation}}
The $\boldsymbol{R}$ matrix of a quadrupole (Eq.~\ref{eq:quadrupole_matrix_thick}) can be simplified using the thin lens approximation, which states that the focal length of the quadrupole (Eq.~\ref{eq:focal_length_quadrupole}) is much larger than the magnetic length of the quadrupole $f \gg L_{\mathrm{quad}}$. The transport matrix for a thin lens quadrupole becomes
\begin{equation}
\boldsymbol{R}_{\mathrm{quad, thin}} = 
\begin{pmatrix}
1 & 0 & 0 & 0 & 0 & 0 \\
-\sqrt{k_{1}}\sin\left(\sqrt{k_{1}}L_{\mathrm{quad}}\right) & 1 & 0 & 0 & 0 & 0 \\
0 & 0 & 1 & 0 & 0 & 0 \\
0 & 0 &  \sqrt{k_{1}}\sinh\left(\sqrt{k_{1}}L_{\mathrm{quad}}\right) & 1 & 0 & 0 \\
0 & 0 & 0 & 0 & 1 & 0 \\
0 & 0 & 0 & 0 & 0 & 1 
\end{pmatrix}.
\label{eq:quadrupole_matrix_thin}    
\end{equation}
The thin lens approximation is describing a quadrupole field in the zero-length regime i.e $L_{\mathrm{quad}}\rightarrow 0$\cite{rossbach1993basic}.

\subsection{Dipole Magnet}

\textcolor{blue}{**NEEDS DIAGRAM OF A DIPOLE** \\ **INCLUDE POLE FACES**\\}
Here we assume a sector dipole magnet with no focusing terms i.e the magnetic field is homogenous transversely throughout dipole magnet. The sector dipole is assumed to have exit and entrance pole faces that are perpendicular to the direction of the reference orbit ($\beta_{1}=0$, $\beta_{2}=0$). A dipole is characterised in its simplest form by two parameters: the bending radius $\rho$ and the magnetic length of the dipole $L_{\mathrm{dip}}$, which is defined as the path length along the central trajectory. Therefore, the $\boldsymbol{R}$ matrix of a non-focusing sector dipole is given by 
\resizebox{\columnwidth}{!}{
\begin{equation}
\boldsymbol{R}_{\mathrm{dip}} =  
\begin{pmatrix}
\cos\left(L_{\mathrm{dip}}/\rho\right) & \frac{1}{\rho}\sin\left(L_{\mathrm{dip}}/\rho\right) & 0 & 0 & 0 & \rho\left[1-\cos\left(L_{\mathrm{dip}}/\rho\right)\right] \\
-\frac{1}{\rho}\sin\left(L_{\mathrm{dip}}/\rho\right) & \sin\left(L_{\mathrm{dip}}/\rho\right) & 0 & 0 & 0 & \sin\left(L_{\mathrm{dip}}/\rho\right) \\
0 & 0 & 1 & L_{\mathrm{dip}} & 0 & 0 \\
0 & 0 & 0 & 1 & 0 & 0 \\
\sin\left(L_{\mathrm{dip}}/\rho\right) & \rho\left[1-\cos\left(L_{\mathrm{drift}}/\rho\right)\right] & 0 & 0 & 1 & \rho\left[\left(L_{\mathrm{dip}}/\rho\right)-\sin\left(L_{\mathrm{dip}}/\rho\right)\right] \\
0 & 0 & 0 & 0 & 0 & 1
\end{pmatrix}.
\label{eq:dipole_matrix}
\end{equation}}
Dipoles are also commonly available with pole faces that are perpendicular to the reference orbit at the centre of the dipole ($\beta_{1}=\alpha_{0}/2$, $\beta_{2}=\alpha_{0}/2$), where $a_{0}$ is the bending angle of the dipole (Eq.~\ref{eq:dipole_bending_angle}), which are termed rectangular dipoles. The rotated pole faces act like thin quadrupoles (Eq.~\ref{eq:quadrupole_matrix_thin}), providing transverse focusing and there effect of the dynamics of a particle beam can be expressed via the $\boldsymbol{R}$ matrix
\begin{equation}
\boldsymbol{R}_{\mathrm{pole}} =
\begin{pmatrix}
1 & 0 & 0 & 0 & 0 & 0 \\
\frac{1}{\rho}\tan\beta & 1 & 0 & 0 & 0 & 0 \\
0 & 0 & 1 & 0 & 0 & 0 \\
0 & 0 & -\frac{1}{\rho}\tan\left(\beta-\psi\right) & 1 & 0 & 0 \\
0 & 0 & 0 & 0 & 1 & 0 \\
0 & 0 & 0 & 0 & 0 & 1
\end{pmatrix},
\label{eq:pole_face_matrix}    
\end{equation}
where $\beta$ is the angle of the pole face and $\psi = k\left(h/\rho\right)\frac{1+\sin^{2}\beta}{\cos\beta}$, where $h$ is the gap between the poles of the magnet -- the $\psi$ term is typically negligible. 

\subsection{Quadrupole Doublet}

A quadrupole doublet is considered here to demonstrate how a beamline is constructed from multiple elements using (Eq.~\ref{eq:subseries_transport_matrix}). For simplicity, thin lens quadrupoles (Eq.~\ref{eq:quadrupole_matrix_thin}) with identical magnitude field gradient and length are used for the calculation. The quadrupole doublet, of the form shown in Fig.~\textcolor{blue}{**QUADRUPOLE DOUBLET DIAGRAM**}, is constructed from a defocusing and focusing quadrupole separated by a drift space; the overall $\boldsymbol{R}$ matrix has the form
\begin{equation}
\boldsymbol{R} = \boldsymbol{R}_{\mathrm{F, quad}}\boldsymbol{R}_{\mathrm{drift}}\boldsymbol{},
\label{eq:doublet_R_matrix}
\end{equation}
where $\boldsymbol{R}_{\mathrm,{F, quad}}$ is a focusing quadrupole (Eq.~\ref{eq:quadrupole_matrix_thin}) and $\boldsymbol{R}_{\mathrm{drift}}$ is a drift space (Eq.~\ref{eq:drift_matrix}) and $\boldsybol{R}_{\mathrm{D, quad}}$ is a defocusing quadrupole with thin lens transport matrix
\begin{equation}
\boldsymbol{R}_{\mathrm{D, quad}} =
\begin{pmatrix}
1 & 0 & 0 & 0 & 0 & 0\\
\sqrt{k_{1}}\sinh\left(\sqrt{k_{1}}L_{\mathrm{quad}}\right) & 1 & 0 & 0 & 0 & 0 \\
0 & 0 & 1 & 0 & 0 & 0 \\
0 & 0 & -\sqrt{k_{1}}\sin\left(\sqrt{k_{1}}L_{\mathrm{quad}}\right) & 1 & 0 & 0 \\
0 & 0 & 0 & 0 & 1 & 0 \\
0 & 0 & 0 & 0 & 0 & 1
\end{pmatrix}.
\label{eq:defocusing_quadrupole_matrix}    
\end{equation}
Therefore, using (Eq.~\ref{eq:doublet_R_matrix}) the transport matrix of a quadrupole doublet becomes 
\textcolor{blue}{**CHECK TRANSPORT MATRICIES + CALCULATE**}

\section{Twiss Parameters and Emittance}
\textcolor{blue}{**INCLUDE DISPERSION**\\}

\subsection{Generalisation of the Equations of Motion}

The equations of motion currently derived (Eqs.~) have only been appropriate for a single test particle, however as we aim to understand the dynamics of a bunch of many particles the equations of motion must be generalised to the many particle scenario. Generalisation of the equations of motion () is achieved via setting $1/\rho=0$, assuming a very large bending radius, and $\Delta p/p_{0}=0$, which assumes the bunch is monoenergetic, which reveals a longitudinal dependence within the quadrupole focusing term. Generalisation by these assumptions yields Hill's equation
\begin{equation}
x''\left(s\right)-k\left(s\right)x\left(s\right) = 0,
\label{eq:Hills_equation}    
\end{equation}
where $x\left(s\right)$ describes the trajectory of the particles which is an oscillation about the reference orbit named a betatron oscillation. Whilst $x\left(s\right)$ is used as an example, Hill's equation can be generalised to either transverse plane. The amplitude and phase of the betatron oscillation are dependent on $s$, the longitudinal position around the reference orbit. Therefore, a trial solution to Hill's equation is developed 
\begin{equation}
x\left(s\right) = \sqrt{\epsilon\beta\left(s\right)}\cos\left[\Psi\left(s\right)+\phi\right],
\label{eq:Hills_trial_solution}    
\end{equation}
where $\sqrt{\epsilon\beta\left(s\right)}$ is an amplitude function with a constant amplitude $\epsilon$ termed the emittance and $\beta\left(s\right)$ termed the $\beta$-function which accounts for the oscillatory amplitude of the betatron oscillation and $\Psi\left(s\right)$ is the phase of the oscillation with $\phi$ the phase offset -- an integration constant resulting from initial conditions. The amplitude function of the trial solution is given by
\begin{equation}
E\left(s\right) = \sqrt{\epsilon\beta\left(s\right)},
\label{eq:envelope_function}    
\end{equation}
which defines a beam envelope, as shown diagrammatically in Fig.~\textcolor{blue}{**BEAM ENVELOPE DIAGRAM**}, which defines the bounds of the trajectories of each of the many particles within a bunch. 

The derivative of the trial solution (Eq.~\ref{eq:Hills_trial_solution}) shows the divergence of the particle trajectory with respect to time and is given by
\begin{equation}
x'\left(s\right) = -\frac{\sqrt{\epsilon}}{\sqrt{\beta\left(s\right)}}\left\{\alpha\left(s\right)\cos\left[\Psi\left(s\right)+\phi\right]+\sin\left[\Psi\left(s\right)+\phi\right]\right\}
\label{eq:Hills_trial_solution_derivative}    
\end{equation}
where the $\alpha\left(s\right)$ function, which relates to the orientation of the beam in phase space, is given by
\begin{equation}
\alpha\left(s\right) = -\frac{\beta'\left(s\right)}{2}.
\label{eq:alpha_function}    
\end{equation}
Taking the second derivative of the trial solution and substituting the trial solution (Eq.~\ref{eq:Hills_trial_solution}) and it's derivatives into Hill's equation (Eq.~\ref{eq:Hills_equation}) whilst understanding that the phase varies continually around the orbit and has a different value at each point on the trajectory \cite{wille2000physics}, the betatron phase can be calculated by
\begin{equation}
\Psi\left(s\right) = \int_{0}^{s}\frac{1}{\beta\left(s\right)}ds.
\label{eq:betatron_phase}    
\end{equation}

\subsection{Phase Space}

The phase space of a collection of particles represents the possible states of the dynamical system in a 6D representation, with each point in phase space illustrating a singular particle \cite{jones2016design}. The 6D phase space is presented within the horizontal ($x$--$x'$), vertical ($y$--$y'$) and longitudinal ($z$--$z'$) planes. The phase space planes are position--divergence planes, in which the divergence of particle is closely related to it's momentum ($x' \propto p_{x}$). The previously derived solution to Hill's equation (Eq.~\ref{eq:Hills_trial_solution}) and it's derivative (Eq.~\ref{eq:Hills_trial_solution_derivative}) can be used to map the behaviour of a collection of particles within this phase space. 

Firstly, the trial solution and its derivative must be re-arranged to eliminate the betatron phase $\Psi\left(s\right)$ dependence. Re-arranging (Eq.~\ref{eq:Hills_trial_solution}) we obtain
\begin{equation}
\cos\left[\Psi\left(s\right)+\phi\right] = \frac{x\left(s\right)}{\sqrt{\epsilon\beta\left(s\right)}},
\label{eq:trial_phase_rearrangement}    
\end{equation}
which can be substituted into (Eq.~\ref{eq:Hills_trial_solution_derivative}) and re-arranged to yield
\begin{equation}
\sin\left[\Psi\left(s\right)+\phi\right] = \frac{\sqrt{\beta\left(s\right)}x'\left(s\right)}{\sqrt{\epsilon}} + \frac{\alpha\left(s\right)x\left(s\right)}{\sqrt{\epsilon\beta\left(s\right)}}.
\label{eq:trial_derivative_phase_rearrangement}    
\end{equation}
Using the well-known trigonometric identity $\sin^{2}x+\cos^{2}x=1$, with (Eqs.~\ref{eq:trial_phase_rearrangement}, \ref{eq:trial_derivative_phase_rearrangement}) the phase space variables can be represented by
\begin{equation}
\frac{x\left(s\right)^{2}}{\beta\left(s\right)}+\left[\frac{\alpha\left(s\right)}{\sqrt{\beta\left(s\right)}}x\left(s\right)+\sqrt{\beta\left(s\right)}x'\left(s\right)\right]^{2} = \epsilon,    
\label{eq:phase_space_trigonometry}
\end{equation}
which when expanded and then simplified using the definition
\begin{equation}
\gamma\left(s\right) = \frac{1+\alpha\left(s\right)^{2}}{\beta\left(s\right)},
\label{eq:gamma_twiss}    
\end{equation}
which relates to the angular size of the beam, becomes
\begin{equation}
\epsilon = \gamma\left(s\right)x\left(s\right)^{2}+2\alpha\left(s\right)x\left(s\right)x'\left(s\right)+\beta\left(s\right)x'\left(s\right)^{2},
\label{eq:phase_space_ellipse}    
\end{equation}
where the area of the phase space ellipse is given by
\begin{equation}
A=\pi\epsilon,
\label{eq:phase_space_area}    
\end{equation} 
and $\epsilon$ is defined as the single particle emittance or action. The form in (Eq.~\ref{eq:phase_space_ellipse}) is that of the equation of an ellipse; consequently the motion of a single particle arount an orbit maps out an ellipse in phase space. The parameters ($\beta\left(s\right)$, $\alpha\left(s\right)$, $\gamma\left(s\right)$) we have defined to interpret the phase space of a particle are collectively known as the Twiss parameters. Phase space ellipses are observed in each of the three planes as the results here are general. An example of the phase space ellipse in the $x$--$x'$ plane is shown in Fig.~\textcolor{blue}{**PHASE SPACE ELLIPSE DIAGRAM**}.

\textcolor{blue}{**MAYBE SHOULD RE-THINK SECTION ON EMITTANCE**}
Since the phase space ellipse is for a single particle trajectory, the emittance is the single particle emittance. This is unsatisfactory for describing an ensemble of particles, therefore the concept of emittance requires extension to a many-particle ensemble. In an electron accelerator, which is the focus of this work, the equilibrium distribution of particles can be well approximated by a Gaussian distribution \cite{wille2000physics}, which has a transverse charge density distribution $\rho\left(x,z\right)$ at the longitudinal centroid of the electron bunch of the form
\begin{equation}
\rho\left(x,z\right) = \frac{N_{e}}{2\pi\sigma_{x}\sigma_{z}}\exp\left(\right),
\label{eq:Gaussian_transverse_charge_distribution}    
\end{equation}
where $N_{e}$ is the no. electrons in the distribution, $\sigma_{x}$ and $\sigma_{z}$ are the \textit{rms} horizontal vertical beam sizes of the distribution and we define the emittance in each plane as
\begin{align}
\sigma_{x}\left(s\right) &= \sqrt{\epsilon_{x}\beta_{x}\left(s\right)}, 
\label{eq:horizontal_emittance} \\
\sigma_{z}\left(s\right) &= \sqrt{\epsilon_{z}\beta_{z}\left(s\right)},
\label{eq:vertical_emittance}
\end{align}
where $\beta_{x/y}$ are the betatron amplitude functions in each plane. A similar treatment can be applied to the longitudinal plane.

Liouville's theorem \cite{liouville1838note}, as commonly applied in statistical mechanics \cite{gibbs1902elementary}, maintains that the phase-space density of a system remains unchanged under conservative forces applied to the system. Applied to the accelerator situation of harmonic oscillation around a reference orbit due to conserved magnetic forces, Liouville's theorem implies that the emittance is conserved throughout the beamline unless the beam of particles is acted upon by a non-conservative force. Therefore, under conservative forces, the propagation of Twiss parameters can be accomplished similarly to the transform of phase space co-ordinates. For example, in the horizontal plane phase space co-ordinates can be transformed from an initial state ($x_{0}$, $x'_{0}$) to a final state ($x_{1}$, $x'_{1}$) via transport matricies such that
\begin{equation}
\begin{pmatrix}
x_{1} \\
x'_{1}
\end{pmatrix}
=
\begin{pmatrix}
\boldsymbol{R}_{1,1} & \boldsymbol{R}_{1,2} \\
\boldsymbol{R}_{2,1} & \boldsymbol{R}_{2,2}
\end{pmatrix}
\begin{pmatrix}
x_{0} \\
x'_{0}
\end{pmatrix}.
\label{eq:general_phase_space_transform}
\end{equation}
Expanding the matricies in (Eq.~\ref{eq:general_phase_space_transform}) and re-casting the equations for substitution into (Eq.~\ref{eq:phase_space_ellipse}), the analagous transform of the Twiss parameters from initial state ($\beta_{0}$, $\alpha_{0}$, $\gamma_{0}$) to final state ($\beta_{1}$, $\alpha_{1}$, $\gamma_{1}$) becomes 
\begin{equation}
\begin{pmatrix}
\beta_{1} \\
\alpha_{1} \\
\gamma_{1} 
\end{pmatrix}
=
\begin{pmatrix}
\boldsymbol{R}_{1,1}^{2} & -2\boldsymbol{R}_{1,1}\boldsymbol{R}_{1,2} & \boldsymbol{R}_{1,2}^{2} \\
-\boldsymbol{R}_{1,1}\boldsymbol{R}_{2,1} & \boldsymbol{R}_{1,2}\boldsymbol{R_{2,1}}+\boldsymbol{R}_{1,1}\boldsymbol{R}_{2,2} & -\boldsymbol{R}_{1,2}\boldsymbol{R}_{2,2} \\
\boldsymbol{R}_{2,1}^{2} & -2\boldsymbol{R}_{2,1}\boldsymbol{R}_{2,2} & \boldsymbol{R}_{2,2}^{2} 
\end{pmatrix}
\begin{pmatrix}
\beta_{0} \\
\alpha_{0} \\
\gamma_{0}
\end{pmatrix}.
\label{eq:general_Twiss_transform}
\end{equation}

\subsection{Periodic Lattices and Stability}

A periodic lattice, for example a repeating series of beamline elements, named cells, with length $L_{\mathrm{cell}}$, which adheres to the property $\boldsymbol{R}\left(s\right) = \boldsymbol{R}\left(s+L_{\mathrm{cell}}\right)$, in which the Twiss parameters are unchanged ($\beta_{0}=\beta_{1}=\beta$, $\alpha_{0}=\alpha_{1}=\alpha$, $\gamma_{0}=\gamma_{1}=\gamma$) has the single plane transport matrix of the form
\begin{equation}
\boldsymbol{R} =
\begin{pmatrix}
\cos\mu + \alpha\sin\mu & \beta\sin\mu \\
-\gamma\sin\mu & \cos\mu-\alpha\sin\mu
\end{pmatrix}
\label{eq:periodic_transport_matrix}    
\end{equation}
where $\mu$ is the phase advance per cell. The Twiss parameters of the periodic lattice can therefore be related to the transport matrix in each plane via re-arrangement of the periodic transport matrix (Eq.~\ref{eq:periodic_transport_matrix})
\begin{align}
\beta &= \frac{\boldsymbol{R}_{1,2}}{\sin\mu}, 
\label{eq:periodic_beta_function} \\
\alpha & = \frac{\boldsymbol{R}_{1,1}-\boldsymbol{R}_{2,2}}{2\sin\mu},
\label{eq:periodic_alpha} \\
\gamma &= \frac{1+\left(\boldsymbol{R}_{1,1}-\boldsymbol{R}_{2,2}\right)^{2}}{4\boldsymbol{R}_{1,2}\sin\mu},
\label{eq:periodic_gamma} \\
\mu &= \cos^{-1}\left(\frac{\boldsymbol{R}_{1,1}-\boldsymbol{R}_{2,2}}{2}\right),
\label{eq:periodic_phase_advance}
\end{align}
and the periodic $\gamma$ function is defined by (Eq.~\ref{eq:gamma_twiss}).

The periodic focusing system within an accelerator is stable if the eigenvectors of the transport matrix remain real for many passes through the accelerator beamline because when this condition is met the phase of the particles remains real. Therefore, the eigenvectors of the transport matrix can be found through the well known equation
\begin{equation}
\left(\boldsymbol{R}-\lambda\boldsymbol{I}\right)\boldsymbol{X} = 0,
\label{eq:eigenvalues_equation}    
\end{equation}
where $\boldsymbol{R}$ is the transport matrix, $\lambda$ are the eigenvalues of the system, $\boldsymbol{I}$ is the identity matrix and $\boldsymbol{X}$ is a vector of the relevant phase space variables. It follows from the periodic transport matrix that the sum of the eigenvalues of $\boldsymbol{R}$ must satisfy
\begin{equation}
\sum\lambda = \boldsymbol{\mathrm{Tr}}\left(\boldsymbol{R}\right) = 2\cos\mu,
\label{eq:eigenvalue_stability}
\end{equation}
where $\boldsymbol{\mathrm{Tr}}\left(\boldsymbol{R}\right)$ is the trace of the transport matrix. As previously mentioned, the trace of the transport matrix must satisfy (Eq.~\ref{eq:eigenvalues_equation}) for many passes through the accelerator beamline, which for a periodic lattice is typically constructed from many periodic cells. Consequently the stability criterion is expanded to become
\begin{equation}
\boldsymbol{\mathrm{Tr}}\left(\boldsymbol{R}^{N}\right) = 2\cos\left(N\mu\right) \leq 2, 
\label{eq:stability_criterion}    
\end{equation}
where $N$ is the number of periodic cells traversed.

\subsection{Dispersion and Off-Momenta Particles} 
\label{sec:dispersion_off_momentum_particles}

Dispersion is defined as the variation in position of a particle due to the variation in its momentum from the design momentum, as introduced in Section~\ref{sec:equations_of_motion}. Particles that vary from the design momentum vary from typical betatron motion $\boldsymbol{X}_{\beta}$, with an additional dispersive motion term $\boldsymbol{X}_{\Delta p/p_{0}}$ i.e $\boldsymbol{X} = \boldsymbol{X}_{\beta} + \boldsymbol{X}_{\Delta p/p_{0}}$. Therefore, the matrix representation of the propagation of dispersive particles becomes
\begin{equation}
\boldsymbol{X}_{1} = \boldsymbol{R}\boldsymbol{X}_{\beta} + \boldsymbol{R}\boldsymbol{X}_{\Delta p/p_{0}},
\label{eq:dispersive_transport_matrix_propagation}    
\end{equation}
which, taking the $x$--$x'$ plane as an example can be accounted for using a transform including the dispersive motion
\begin{equation}
\begin{pmatrix}
x_{1} \\
x'_{1} \\
\Delta p /p_{0}
\end{pmatrix}
=
\begin{pmatrix}
\boldsymbol{R}_{1,1} & \boldsymbol{R}_{1,2} & \boldsymbol{R}_{1,6} \\
\boldsymbol{R}_{2,1} & \boldsymbol{R}_{2,2} & \boldsymbold{R}_{2,6} \\
0 & 0 & 1
\end{pmatrix}
\begin{pmatrix}
x_{0} \\
x'_{0} \\
\Delta p / p_{0} 
\end{pmatrix}
\label{eq:dispersive_transport_simplified}
\end{equation}
where the index 6 is used as this is the column of a 6D transport matrix that would include these dispersive terms. A dispersion function $\eta\left(s\right)$ can be defined for the motion of the particle by neglecting the betatron motion and setting $\Delta p/p_{0} =1$
\begin{equation}
\begin{pmatrix}
\eta_{x}\left(s\right) \\
\eta'_{x}\left(s\right) \\
1
\end{pmatrix}
=\boldsymbol{R}
\begin{pmatrix}
x_{0} \\
x'_{0} \\
1
\end{pmatrix}.
\label{eq:dispersion_function}    
\end{equation}
The periodic dispersion ($\eta_{x}\left(s\right)=\eta_{x}$, $\eta'_{x}\left(s\right)=\eta'_{x}$) can therefore be found via re-arrangement of (Eq.~\ref{eq:dispersion_function}) using the transport matrix of the form in (Eq.~\ref{eq:dispersive_transport_simplified}) in terms of the transport matrix elements
\begin{align}
\eta_{x} &= \frac{\boldsymbol{R}_{1,3}\left(1-\boldsymbol{R}_{2,2}\right)+\boldsymbol{R}_{R}_{1,2}\boldsymbol{R}_{2,3}}{\left(1-\boldsymbol{R}_{1,1}\right)\left(1-\boldsymbol{R}_{2,2}\right)-\boldsymbol{R}_{2,1}\boldsymbol{R}_{1,2}},
\label{eq:periodic_dispersion} \\
\eta'_{x} &= \frac{\boldsymbol{R}_{2,3}\left(1-\boldsymbol{R}_{1,1}\right)+\boldsymbol{R}_{2,1}\boldsymbol{R}_{1,3}}{\left(1-\boldsymbol{R}_{1,1}\right)\left(1-\boldsymbol{R}_{2,2}\right)-\boldsymbol{R}_{2,1}\boldsymbol{R}_{1,2}}.
\label{eq:periodic_dispersion_prime}
\end{align}

The beam size of a dispersive ensemble of particles (here for a periodic lattice) can be defined, baring similarity to (Eq.~\ref{eq:dispersive_transport_matrix_propagation}), through an extension of the beam size due to betatron motion (Eq.~\ref{eq:horizontal_emittance})
\begin{equation}
\sigma_{x} = \sqrt{\epsilon_{x}\beta_{x}} + \eta_{x}\frac{\Delta p}{p_{0}}.
\label{eq:dispersive_beam_size}    
\end{equation}
An extension to the definition of the phase space ellipse (Eq.~\ref{eq:phase_space_ellipse}) is also possible for the case of a dispersive beam where the phase space plane variables ($x$--$x'$ etc.) can be replaced by the dispersion functions
\begin{equation}
\mathcal{H} = \beta\eta'^{2}+2\alpha\eta\eta'+\gamma\eta^{2},
\label{eq:dispersive_phase_space_ellipse}    
\end{equation}
where $\mathcal{H}$ is analogous to the emittance in defining the area of the phase space ellipse for a dispersive beam.

\section{Tune and Chromaticity}

The tune of an accelerator is typically defined as the number of betatron oscillations per revolution around the circumference of a re-circulated accelerator
\begin{equation}
Q=\frac{N\mu}{2\pi} = \frac{1}{2\pi}\oint\frac{1}{\beta}ds,
\label{eq:accelerator_tune}    
\end{equation}
where $N$ is the no. periodic cells within the accelerator, and the integral $\oint$ relates to an integral around the circumference of the accelerator. However, often accelerator beamlines aren't entirely periodic. For example, areas of the beamline can be devoted to focusing to a small spot size for a collider and inverse Compton scattering source or be subject to insertion devices such as undulator magnets. Therefore, it is often more useful to define the tune per cell of a section of the beamline with length $L_{\mathrm{cell}}$
\begin{equation}
Q=\frac{\mu}{2\pi} = \frac{1}{2\pi}\int_{0}^{L_{\mathrm{cell}}}\frac{1}{\beta}ds. 
\label{eq:tune_per_cell}    
\end{equation}

In re-circulated accelerators such as storage rings, the particle beam is subject to the same collection of magnetic elements many times. For example, in the MAX-III storage ring the beam lifetime at 250~\si{\milli\ampere} is 11.3~\si{\hour} with a 36~\si{\meter} circumference (120~\si{\nano\second} revolution time) resulting in $\sim3.4\times 10^{11}$ revolutions per bunch. Therefore, the beam is acted upon by periodic transverse forces causing transverse oscillations of the beam which, under certain conditions, may result in the circulating beam resonating \cite{wille2000physics}. A transversely resonating beam oscillation -- known as an optical resonance -- could cause large beam amplitudes which would result in beam sizes that could not be contained within the beampipe and the particles would be lost.

A full discussion and derivation of the conditions required for optical resonances is beyond the scope of this thesis, instead for brevity the discussion by K. Willie \cite{wille2000physics} is recommended and we utilise the result for the relationship between expected optical resonance behaviour and tune
\begin{equation}
mQ_{x}+nQ_{y} = p,
\label{eq:tune_resonances}    
\end{equation}
where $m$, $n$ and $p$ are integers. Resonances in accelerators occur at integer and fractional-integer tunes ($p = n + 1/3, 1/2, 1, 2$ etc.) and can be coupled resonances, hence why the tunes in both transverse planes are included in (Eq.~\ref{eq:tune_resonance}). The order of the resonance is found via $\left|m\right| + \left|n\right|$, and the strength of the resonance decreases rapidly as a function of order. Consequently, the tunes in an accelerator must be chosen to avoid optical resonances, which is fundamental at the design stage and is often named the working point. The working point is often displayed in a $Q_{x}$--$Q_{y}$ tune diagram, as shown in Fig.~\textcolor{blue}{**TUNE DIAGRAM**}, where the resonances up to $n$th order are typically displayed.  

\textcolor{blue}{**TUNE DIAGRAM**}

Chromaticity is defined as the variation in tune due to the variation in momentum within an accelerator as given by
\begin{equation}
\xi = \frac{\Delta Q}{\Delta p/p_{0}},
\label{eq:accelerator_chromaticity}    
\end{equation}
where the variation in tune is driven by the chromatic aberration focusing errors that occur when off-momentum particles traverse focusing magnets, as presented for a quadrupole in Fig.~\textcolor{blue}{**CHROMATIC ABERRATION DIAGRAM**}

The chromatic effect of a quadrupole magnet can there for be considered as
\begin{align}
\xi_{x} &= -\frac{1}{4\pi}\int\beta_{x}k_{1}ds, \\
\xi_{y} &= \frac{1}{4\pi}\int\beta_{y}k_{1}ds.
\label{eq:quadrupole_chromaticity}
\end{align}
where it is noted that for periodic focusing the $\beta_{x}$-function is typically larger upon entrance to the focusing quadrupole and the $\beta_{y}$ function is larger in the defocusing quadrupole, therefore the natural chromaticity throughout a periodic focusing accelerator is negative. 

Correction of chromaticity is typically conducted by sextupole ($n=3$) magnets with fields as described by (Eq.~\ref{eq:sextupole_magnetic_field}, that consequently have focusing dependent on transverse beam position. Sextupoles are placed in dispersive sections of the accelerator where the transverse position of the particles is dependent upon the momentum of the particle so the transverse field dependence can be used to correct the aberrations. 

\section{Longitudinal Dynamics and RF Acceleration}

Previous discussions within this section have currently been focused on the transverse motion of particles throughout the accelerator beamline. However a similar analysis can also be applied to the longitudinal dimension where instead of magnetic bending and focusing fields, electric fields used in the acceleration of particles dominate the dynamics. Analysis of longitudinal dynamics focuses on the energy of the particles, their longitudinal position relative to the centroid of the ensemble as well as the stability and phase of the longitudinal motion. This section focuses on the effect of bending and dispersive motion upon the longitudinal dynamics, then upon investigation of the longitudinal equations of motion and phase space and finally the effect of magnetic elements upon the longitudinal dimensions of the beam is explored.

\subsection{Momentum Compaction}
\label{sec:momentum_compaction}

The path length $L$ of a particle trajectory as a function of momentum variation can be wrote as a Maclaurin series \cite{wolski2012longitudinal}
\begin{equation}
L = L_{0}\left[1+\alpha_{p}\frac{\Delta p}{p_{0}}+\frac{1}{2}\alpha_{p}^{\left(2\right)}\left(\frac{\Delta p}{p_{0}}\right)^{2}\right],
\label{eq:path_length_expansion}    
\end{equation}
where $L_{0}$ is the path length of the reference particle ($\Delta p/p_{0}=0$). Whilst higher order momentum compaction terms are present in the expansion (Eq.~), such as $\alpha_{p}^{\left(2\right)}$, to first order this can be simplified to
\begin{equation}
\alpha_{p} = \frac{1}{L}\frac{dL}{\Delta p/p_{0}},
\label{eq:momentum_compaction_first_order}    
\end{equation}
where $\alpha_{p}$ is the linear momentum compaction. Consider the trajectory of a particle through a curved reference trajectory -- subject to a dipole bending field -- moving with a displacement $x$ from the reference trajectory, the path length of this element is given by
\begin{equation}
dL = \left(\rho+x\right)d\theta = \left(1+\frac{x}{\rho}\right)ds.
\label{eq:displaced_bend_trajectory}    
\end{equation}
The total path length of the displaced trajectory through the bending field is given by the integral of (Eq.~\ref{eq:displaced_bend_trajectory})
\begin{equation}
L = \int_{0}^{L_{0}} 1+\frac{x}{\rho} ds = L_{0} + \int_{0}^{L_{0}}\frac{x}{\rho}ds, 
\label{eq:displaced_bend_path_length}    
\end{equation}
If we consider the displacement of $x$ to arise purely because of dispersive motion, then $x$ becomes
\begin{equation}
x = \eta_{x}\frac{\Delta p}{p} + \frac{1}{2}\eta_{x}^{\left(2\right)}\left(\frac{\Delta p}{p}\right)^{2},
\label{eq:dispersive_displacement}    
\end{equation}
where the higher order dispersion terms $\eta_{x}^{\left(2\right)}$ are neglected. The displacement due to dispersive motion (Eq.~\ref{eq:dispersive_displacement}) is substituted into the path length of the curved trajectory (Eq.~\ref{eq:displaced_bend_path_length}) yielding
\begin{align}
L &= L_{0} + \frac{\Delta p}{p_{0}}\int_{0}^{L_{0}}\frac{\eta_{x}}{\rho}ds,
\Delta L &= \frac{\Delta p}{p_{0}}\int_{0}^{L_{0}}\frac{\eta_{s}}{\rho}ds,
\label{eq:dispersive_path_length_variation}
\end{align}
which via substitution into (Eq.~\ref{eq:momentum_compaction_first_order}) and re-arrangement yields the momentum compaction in terms of dispersion
\begin{equation}
\alpha_{p} = \frac{1}{L_{0}}\int_{0}^{L_{0}}\frac{\eta_{x}}{\rho}ds.
\label{eq:momentum_compaction_dispersion}    
\end{equation}

Inspection of (Eq.~\ref{eq:momentum_compaction_dispersion}), demonstrates that momentum compaction only arises in dispersive sections where the reference trajectory is curved, such as in dipoles, because in straight sections $\rho\rightarrow\infty$ and therefore the contribution of these terms are negligible. 

\subsection{RF Acceleration}

A particle, or collection of particles, can be accelerated by an electric field $\boldsymbol{E}$ satisfying the wave equation, displayed here in vacuum
\begin{equation}
\nabla^{2}\boldsymbol{E}-\frac{1}{c^{2}}\frac{\partial^{2}\boldsymbol{E}}{\partial t^{2}},
\label{eq:electromagnetic_wave_equation}    
\end{equation}
which (in our local co-ordinate system) is applied in the longitudinal direction. Most modern accelerators use powerful radio-frequency systems, or RF cavities, to produce the requisite strong electric fields, with frequencies from \si{\kilo\hertz} to \si{\giga\hertz} \cite{wille2000physics}, required to accelerate the particles to energies on the \si{\mega\electronvolt} to \si{\tera\electronvolt} scale. Whilst other acceleration methods exist such as electrostatic acceleration \cite{}, dielectric wakefield acceleration \cite{} and laser plasma wakefield acceleration, we limit our discussion to RF acceleration.

The solution to (Eq.~\ref{eq:electromagnetic_wave_equation}) for an RF cavity, where the local ($x$, $y$, $z$) co-ordinate system is used, is of the form
\begin{equation}
\boldsymbol{E}\left(z,t\right) = E_{0}\exp\left[i\left(kz-\omega t\right)\right],
\label{eq:RF_cavity_electric_field}    
\end{equation}
where the phase of the electric field is encapsulated by the synchronous phase $\psi = kz-\omega t$. The energy gain of a particle in an electric field, re-cast in terms of the phase, is subsequently given by 
\begin{equation}
\Delta E = e \int E_{0}\exp\left(i\psi\right)dz = eV\left(\psi\right),
\label{eq:particle_energy_gain_RF}    
\end{equation}
where $V\left(\psi\right)$ is the RF voltage as a function of phase. Therefore, to maintain a constant energy gain within all subsequent RF cavities, the phase must remain constant at the RF cavity i.e
\begin{equation}
\frac{d\psi}{dt} = k\beta c-\omega = 0. 
\label{eq:synchronicity_condition}   
\end{equation}
The transit time between RF cavities and revolution frequency of the particle must therefore remain constant for subsequent acceleration at identical phase. For an accelerator of path length $L$, the solution is of the form $k=2\pi/L$ where $\omega=\omega_{\mathrm{rev}}$ is the revolution frequency. The RF frequency can operate at any integer value of this revolution frequency, named the harmonic number $h$, and still satisfy the synchronicity condition (Eq.~\ref{eq:synchronicity_condition}), therefore the wavenumber and frequency of the accelerating RF waveform have the form
\begin{align}
k_{h} = \frac{2\pi h}{L_{0}},
\label{eq:RF_wavenumber} \\
f_{RF} = \frac{h\omega_{\mathrm{rev}}}{2\pi}.
\label{eq:RF_frequency}
\end{align}

The transit time $T$ between RF cavities is affected by off-momenta particles, as in the case of the path length in momnetum compaction studies (see Section~\ref{sec:momentum_compaction}), similarly the transit time with respect to the reference particle ($\Delta p/p_{0}$=0) can be expanded via a Maclaurin series to account for the momentum variation
\begin{equation}
T = T_{0}\left[1+\eta_{p}\frac{\Delta p}{p_{0}}+\frac{1}{2}\eta_{p}^{\left(2\right)}\left(\frac{\Delta p}{p_{0}}\right)^{2}\right],
\label{eq:transit_time_expansion}    
\end{equation}
where $\eta_{p}$ is named the phase slip factor. Neglecting non-linear terms and re-arranging (Eq.~\ref{eq:transit_time_expansion}) becomes
\begin{equation}
\eta_{p} = \frac{\Delta T/T}{\Delta p/p_{}0},
\label{eq:phase_slip_factor_transit_time}    
\end{equation}
where the transit time is a function of path length $L$, $T = L/\beta c$, which can be used to re-cast the phase slip factor (Eq.~\ref{eq:phase_slip_factor_transit_times}) as  
\begin{equation}
\eta_{p} = \frac{\Delta T/T}{\Delta p/p} = \frac{\Delta L/L_{0}}{\Delta p/p_{0}}-\frac{\Delta\beta/\beta}{\Delta p/p_{0}}.
\label{eq:phase_slip_factor_expansion}    
\end{equation}
The Lorentz speed factor term in (Eq.~\ref{eq:phase_slip_factor_expansion}) can be related to the Lorentz factor \cite{wolski2012longitudinal} by 
\begin{equation}
\frac{\Delta \beta}{\Delta p/p_{0}} = \frac{1}{\gamma^{2}},
\label{eq:phase_slip_Lorentz_relation}    
\end{equation}
therefore using the definition of the momentum compaction factor (Eq.~\ref{eq:momentum_compaction_first_order}) and the relationship for the Lorentz speed factor term (Eq.~\ref{eq:phase_slip_factor_expansion}), the phase slip factor becomes
\begin{equation}
\eta_{p} = \alpha_{p} - \frac{1}{\gamma^{2}}.
\label{eq:phase_slip_factor}    
\end{equation}

Three distinct regimes exist within the longitudinal dynamics relating to the phase slip factor (Eq.~\ref{eq:phase_slip_factor}), relating to the transition between relativistic motion:
\begin{itemize}
    \item{Below transition ($\eta_{p}<0$), where $\alpha_{p}<0 \parallel \alpha_{p}<1/\gamma^{2}$ and the synchronous phase is $0<\psi<\pi/2$.
    \begin{itemize}
        \item{The revolution frequency increases with increasing energy.}
        \item{Higher energy particles arrive at the RF cavity first due to their higher velocity.}
    \end{itemize}}
    \item{At transition ($\eta_{p}=0$), where $\alpha_{p}=1/\gamma^{2}$.
    \begin{itemize}
        \item{The revolution frequency is independent of energy.} 
    \end{itemize}}
    \item{Above transition ($\eta_{p}$, where $\alpha_{p}>1/\gamma^{2}$) and the synchronous phase is $\pi/2<\psi<\pi$.
    \begin{itemize}
        \item{The revolution frequency decreases with increasing energy.}
        \item{Higher energy particles arrive at the the RF cavity last because they traverse a longer path and their higher velocity is of no advantage since $\beta\approx1$.}
    \end{itemize}}
\end{itemize}
The longitudinal dynamics above and below transition are shown diagrammatically in Fig.~\textcolor{blue}{**PHASE FOCUSING DIAGRAM**} At first glance, the reason for the mandated synchronous phases above and below transition is unclear however, this is to take advantage of longitudinal phase focusing. 

\textcolor{blue}{**PHASE FOCUSING DIAGRAM HERE**}

Longitudinal phase focusing takes advantage of the sinusoidal nature of the accelerating electric field (Eq.~\ref{eq:RF_cavity_electric_field}) to bias the acceleration of off-momenta particles, with varying transit times, to restore their phase toward the synchronous phase. The longitudinal phase focusing technique is shown in Fig.~\textcolor{}{}. Above transition, a particle with reduced momentum ($\Delta p/p_{0}<0$) will arrive at the RF cavity earlier, and receiving a larger energy gain will reduce its transit time  as its path length will reduce. However, a particle that arrives late ($\Delta p/p_{0}>0$)...
The voltage, and therefore the energy gain, vary sinusoidally as a function of phase where maximum energy gain is at $\psi=\pi/2$ -- the crest of the electromagnetic wave in the RF cavity. However, setting the synchronous phase to an off-crest position in the range $\pi/2<\psi<\pi$, the slope of the RF voltage around this point is near-linear which is appropriate to achieve the varying energy gain required for longitudinal phase focusing.  

Following the example of Jones \cite{jones2016design}, if the variation in energy and other related beam parameters within a recirculated accelerator is small with relation to phase then the longitudinal equation of motion is of the form 
\begin{equation}
\frac{d^{2}\psi}{dt^{2}} + \beta ck_{h}\eta_{p}\frac{\partial}{\partial t}\left(c\frac{\Delta p}{p_{0}}\right) = 0,
\label{eq:longitudinal_equation_of_motion}
\end{equation}
which can be simplified by assuming a linear expansion of the synchronous phase, and expressed in terms of either the phase or the momentum deviation
\begin{align}
\frac{d^{2}\psi}{dt^{2}} + \Omega^{2}\psi = 0,
\label{eq:longitudinal_equation_of_motion_phase} \\
\frac{d^{2}\left(\Delta p/p_{0}\right)}{dt^{2}} + \Omega^{2}\frac{\Delta p}{p_{0}} = 0,
\label{eq:longitudinal_equation_of_motion_momentum}
\end{align}
where the synchronous phase and momentum are related via
\begin{equation}
\frac{\Delta p}{p_{0}} = -\frac{1}{h\omega_{\mathrm{RF}}\eta_{p}}\frac{d\psi}{dt},
\label{eq:momentum_synchronous_phase_relation}    
\end{equation}
and assuming the RF frequency is a harmonic of the revolution frequency $\omega_{\mathrm{rev}}=2\pi f_{\mathrm{RF}}/h$ and that the RF cavity produces a sinusoidal RF wave $V=V_{0}\sin\psi$, with RF voltage $V$, the synchrotron oscillation frequency is given by
\begin{equation}
\Omega^{2} = \frac{\omega_{\mathrm{rev}}^{2}h\eta_{p}eV_{0}\cos\psi}{2\pi\beta cp_{0}}.
\label{eq:synchrotron_oscillation_frequency}    
\end{equation}
Synchrotron oscillations in the longitudinal plane typically have a lower frequency than the aforementioned betatron oscillations in the transverse planes. The equation of motion in the longitudinal plane () consequently produces a stable $\psi$--$\Delta p/p$ phase space ellipse when motion is damped ($\Omega^{2}>0$) and is described by a hyperbola in phase pace when motion is unstable and anti-damped ($\Omega^{2}<0$). The equation of motion therefore generates a series of ellipses separated by regions of unstable hyperbolic regions which form separatrixes in phase space, as displayed in Fig.~\textcolor{blue}{**LONGITUDINAL PHASE SPACE DIAGRAM**}

\textcolor{blue}{**LONGITUDINAL PHASE SPACE DIAGRAM**}

The region of stable motion within the separatrix is named the RF bucket, and particles within an RF bucket form a bunch with bunch length proportional to the size of the RF bucket transformed to the longitudinal position. The maximum momentum deviation of an RF bucket is given by \cite{wolski2012longitudinal}
\begin{equation}
\left(\frac{\Delta p}{p_{0}}\right)_{\mathrm{max}} = \frac{2\Omega}{\omega_{\mathrm{rev}}\beta h\eta_{p}}\sqrt{1+\left(\psi+\frac{\pi}{2}\right)\tan\psi},
\label{eq:RF_bucket_momentum_deviation}    
\end{equation}

\subsection{Magnetic Bunch Compression and $\boldsymbol{R}_{56}$}

The longitudinal dimensions of a particle bunch can be adjusted via combinations of magnetic focusing elements, which can be useful in generating radiation from accelerators such as 3rd generation synchrotron sources and free electron lasers to achieve properties such as short duration and high brilliance. Here two methods of varying the longitudinal dynamics, specifically the bunch length, are investigated in detail: the simple chicane and the dog leg lattices.

A simple dogleg lattice can consist of a pair of dipoles with bending angle $\alpha_{1}$ and $-\alpha_{1}$ respectively, separated by a drift of length $L_{\mathrm{drift}}$, as shown in Fig.~\textcolor{blue}{**DIAGRAM OF DOGLEG**}.

The dogleg therefore has a transport matrix of the form
\begin{equation}
\boldsymbol{R}_{\mathrm{dog}} = \boldsymbol{R}_{\mathrm{dip,2}}\boldsymbol{R}_{\mathrm{drift}}\boldsymbol{R}_{\mathrm{dip,1}},
\label{eq:dogleg_transport_matrix_simple}    
\end{equation}
which yields 
\textcolor{blue}{**FULL 6D TRANSPORT MATRIX**}
In the longitudinal co-ordinates ($z$,$\Deltap/p_{0}$), this yields the relations
\textcolor{blue}{**DOGLEG CO-ORD TRANSFORMS**}
where the $\boldsymbol{R}_{56}$ matrix element is given by
\begin{equation}
\boldsymbol{R}_{56} = L_{\mathrm{drift}}\left(1+\sin^{2}\alpha_{1}\right),
\label{eq:dogleg_R56}    
\end{equation}
therefore $\boldsymbol{R}_{56}>0$ for a dogleg lattice. 
\textcolor{blue}{**WHAT HAPPENS TO Z CO-ORD THEN? DOES THE BUNCH GET LARGER?}


\section{Linear Transport Lattices}

\subsection{FODO Lattice}

\subsection{Multi-Bend Achromat}

\subsection{Fixed Field Accelerating Gradient}


\section{Description of an Energy Recovery Linac}

\textcolor{blue}{**HOLZER'S TRASVERSE BEAM DYNAMICS A GOOD STARTING POINT**}

\textcolor{blue}{\begin{itemize}
    \item{Need to explain a photoinjector and an injector cryomodule}
    \item{Explain particle acceleration better}
    \item{Why is SRF currently preferred?
        \begin{itemize}
            \item{beam current considerations} 
        \end{itemize}}
    \item{single pass ERL}
    \item{multi-pass ERL}
    \item{transport options}
    \item{Dual linac ERLs
        \begin{itemize}
            \item{Asymmetry of linacs etc.}
            \item{what are the benefits?}
        \end{itemize}}
    \item{beam loading}
    \item{push-pull linacs}
\end{itemize}}

\subsection{Single Turn ERL}

Within this first order explanation of an energy recovery linac we consider, for simplicity, the case of a single turn electron ERL, with a single linac. Firstly, the electron bunches are injected typically by a high energy photoinjector and injector accelerator section to an electron beam kinetic energy of 5--10~\si{\mega\electronvolt} with a small emittance either operating in a burst mode, in which a train of bunches is injected with intermittent pauses, or in continuous wave (CW) mode where pulses are injected with a fixed bunch spacing. These particle bunches are then delivered in suitable configuration to a linac, a series of consecutive radio-frequency (RF) accelerating cavities, which accelerate the particle bunch via a generated electric field to some nominal energy. For an ERL either normal conducting RF (NCRF) or superconducting radio-frequency RF (SRF) accelerating cavities can be utilised. The accelerating electromagnetic field in an RF cavity produces a standing wave of wavelength $\lambda_{\mathrm{RF}}$ with a potential difference that is sinusoidally varying in position and time. Therefore, to accelerate the electron bunch it's transit of the cavity must coincide with near-peak voltage of the electric field.   

The particle bunch accelerated to nominal energy is then transported through a return beamline (return loop) consisting of a series of magnets which confine the electron bunch and return it to the linac. During transport the electron bunch could be utilised for a variety of applications, such as driving a light source, or for collider experiments as envisioned in Tigner's original design \cite{tigner1965possible}.

The electron bunch must have a path length of $\left(w+\frac{1}{2}\right)\lambda_{\mathrm{RF}}$, where $w$ is an integer, through the return beamline because this will determine that the electron bunch re-enters the linac at a trough in the potential difference ensuring the particle bunch is decelerated. The kinetic energy of the electron bunch is transferred to the electric field of the accelerating cavities within this deceleration and therefore recovered for acceleration of a subsequent bunch. The initial electron bunch is then transported to a beam stop.

When the energy imparted to the electron beam by an RF cavity is recovered by the identical RF cavity upon deceleration this is termed same cell energy recovery. Several different schemes can exist where energy imparted by a particular RF cavity is recovered by a different RF cavity or the energy used to accelerate an electron bunch is only partially recovered by an RF cavity. Schematics of energy recovery can be further complicated when there are multiple linac sections within an energy recovery linac and complexity is compounded when these are of asymmetric length.     

\subsection{Multi-pass ERL}

Multi-pass ERLs are a prominent subject within this manuscript, these operate differently from single turn ERLs as there are consecutive accelerating passes of the linac, as in a recirculating linac \cite{axel1977status}, followed by consecutive decelerating passes of the linac. In a multi-pass ERL the electron bunch is injected into the linac and is accelerated to the first nominal energy, then transported to the linac via a return beamline. However, we differ from a single turn ERL here as the return beamline must have a path length of $w\lambda_{\mathrm{RF}}$ in order for the electron bunch to coincide with the peak of the accelerating field and therefore be re-accelerated to the next nominal energy.

The acceleration can occur for an unbounded total of $\frac{m}{2}$ passes before the return beamline must conform to the $\left(w+\frac{1}{2}\right)\lambda_{\mathrm{RF}}$ path length condition; a phase change of 180\si{\degree} relative to the electromagnetic wave of the linac RF cavities which converts the electron bunch into a decelerating configuration. A series of $\frac{m}{2}$ decelerating passes then returns the bunch to its injection energy via a total of $m$ linac passes. The decelerating passes must have a path length obeying the $w\lambda_{\mathrm{RF}}$ condition as the bunch is already in the decelerating configuration and therefore must maintain a path length of integer RF wavelengths in order for the electron bunch to coincide with the troughs of the RF cavity electromagnetic wave. Once the electron bunch has returned to the injection energy it is transported to the beam stop. Each deceleration recovers energy for the subsequent electron bunch to be accelerated.        

Here we define the convention that an acceleration and later deceleration is termed a turn and that a single traversal of the linac is termed a pass, accelerating or otherwise. For example for a single linac two turn ERL, the electron bunch is accelerated by the linac twice, resulting in two nominal energies, then decelerated by the linac twice, however the electron bunch is returned to the linac only three times. An $n$ turn single linac ERL involves $m$ passes of the linac and requires $m-1$ traversals of a return beamline before the electron bunch is transported to the beam stop, corresponding to $n$ nominal energies of the ERL.   

\subsection{Multi-pass Electron Transport}

The requirement upon multi-pass ERLs to have multiple return beamlines which have to be in either accelerating or decelerating configuration and operate at several nominal energies enables a variety of solutions to beam transport. The simplest solution conceptually is to have a separate return transport for each individual pass, termed separate transport. This allows for independence in specification of each beamline, with both each nominal energy electron bunch having its own beamline and flexibility between accelerating and decelerating passes. However, separate transport means that $m-1$ beamlines must be configured to transport the electron beam from and to the linac which results in complicated spreader systems with challenging space considerations. Requiring multiple beamlines also increases cost of an ERL and in practice neighbouring beamlines can affect each other.

Within common transport, the accelerating and decelerating passes for each nominal energy that is traversed twice (i.e all but the maximum nominal energy) share a common return transport. This is possible since after the electron beam has been converted to the decelerating configuration on the maximum nominal energy pass (traversed only once) the path length condition $w\lambda_{\mathrm{RF}}$ is identical for the other nominal energies for accelerating and decelerating passes. However, the properties of the electron bunch do not remain invariant between accelerating and decelerating passes therefore the common transport return beamlines are doubly constrained. The advantage of this approach is that only $n$ transport beamlines are required, which reduces cost and the difficulty in design of spreader switch-yards where the beamlines feed into the entrance and exit of the linac.   
Another solution exists wherein multiple nominal energy electron beams in both accelerating and decelerating configuration are transported within the same beamline. To distinguish this from common transport we term this approach multi-energy common transport, This has been demonstrated \cite{bartnik2020cbeta} using a fixed field alternating gradient beamline. However, to operate a multi-pass ERL with a multi-energy common transport the matching constraints of this beamline are $m-1$ fold and the required phase change must be designed into the beamline. There is little room for error in this approach as correction or path length or optics has consequences for every other return pass. Hence, these are often utilised with a common transport spreader arrangement at the entrance and exit of the linac, though these are also highly constrained beamlines. The multi-energy common transport approach is advantageous as potentially only a single beamline need be afforded and theoretically, if all constraints can be satisfied, this avoids the need of a spreader system.  

\section{Collective Effects}

\subsection{Beam Breakup Instability}

The average beam current off energy recovery linacs can be seriously limited by the effects of the beam breakup instability (BBU). Average beam current limitations are challenging for inverse Compton scattering source development with ERL drivers because of the dependency of flux upon electron beam average current. Therefore, this collective effect has substantial consequences for the design of ERL driven ICS sources.

In the beam breakup instability, the electron bunch interacts with a higher order mode (HOM) of the accelerating structure and is subsequently deflected. Electron bunches with a transverse offset traverse an RF cavity and excite a HOM within the cavity. The electron bunch subsequently recieves a transverse kick, thereby generating a further higher order mode due to the transverse offset the kick generates. Beyond some threshold current value $I_{\mathrm{th}}$ this process becomes exponential in nature resulting in beam loss.

Two forms of beam breakup exist; cumulative BBU in which the instability builds up over a series of RF accelerating structures, for example a long (100's~\si{\meter}) linac, and regenerative BBU where the effect build up via multiple passes of the same RF cavity. The former of these was the observed first, with experimentation at SLAC \cite{panofsky1966electrons,altenmueller1966beam} and LLNL \cite{neil1970coherent} and analytically modelled by Panofsky and Bander et al \cite{panofsky1968asymptotic}. However, for ERLs -- especially multi-turn ERLs -- regenerative beam breakup instability is of greater concern due to the recirculated nature of the accelerator in which a short linac section is typical.

Regenerative BBU is particularly present in multi-turn ERLs in comparison to circular machines such as synchrotrons for several reasons for example, the multiple energy bunches in the ERL are deflected differently for the same HOM voltage causing variation in bunch transverse offsets and the change in revolution time for the first decelerating bunch in which the phase is modulated by 180\si{\degree} \cite{setiniyaz2021filling}.

Theoretical studies of regenerative BBU in ERL's were conducted by Hoffstaetter and Bazarov \cite{hoffstaetter2004beam} for uncoupled optics, then furthered for the case of coupled optics \cite{hoffstaetter2007recirculating}. The focus here is typically on the excitation of dipole higher order modes \textcolor{blue}{Why?} Experimentally, regenerative BBU was observed and studied in an ERL at the Jefferson Laboratory FEL \cite{tennant2005first,douglas2006experimental}. \textcolor{blue}{What did this show?} Here conventional methods of suppressing BBU \cite{tennant2004methods} in an ERL, such as HOM dampeners, are explored as well as accelerator optics based solutions \cite{rand1980beam}. 

Recently BBU studies have been performed for the CBETA ERL \cite{lou2019beam}, one of the main accelerators of interest within this work, which have predicted a restrictive average beam current limitation of $I_{\mathrm{th}} = 40$~\si{\milli\ampere}. However, mitigation of regenerative BBU remains an active field exemplified by the proposal of a bunch filling pattern which could increase the threshold current limitation by up to a factor of 5 for a 3-turn ERL \cite{setiniyaz2021filling}.    

\end{document}