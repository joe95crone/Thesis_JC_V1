%%%%%%%%%%%%%%%%%%%%%%%%%%%%%%%%%%%%%%%%%%%%%%%%%%%%%%%%%%
%
% Doctoral Thesis Template @ The University of Manchester
% LaTeX Chapter Template
% Version 1 (23/07/2020)
% Joe Crone
%
% This template is based on:
% The University of Manchester, Presentation of Thesis Policy
% Research Office Graduate Education Team
% June 2017
% http://www.regulations.manchester.ac.uk/pgr-presentation-theses/
%
%%%%%%%%%%%%%%%%%%%%%%%%%%%%%%%%%%%%%%%%%%%%%%%%%%%%%%%%%%
\documentclass[../main.tex]{subfiles}
\begin{document}

% Title
%--------------------------------------------------------
\chapter{Optimisation and Characterisation of Inverse Compton Scattering Spectra}
\label{Optimisation_and_Characterisation_of_Inverse_Compton Scattering_Spectra} % to reference use \ref{ChapterTemplate}

\section{Transverse Profile Matching}

\section{1D Round Beam Optimisation}
\label{sec:RB_optimisation}

The bandwidth of an ICS light source is tuneable and can be selected on the basis of user requirements. Here we develop a method to maximise flux within a selected \textit{rms} bandwidth for the simplified case of an electron beam with a round transverse profile i.e normalised emittance is identical in both directions ($\epsilon_{nx} = \epsilon_{ny} = \epsilon_{n}$). This can simply be converted to a FWHM bandwidth (Eq.~\ref{eq:FWHM_bandwidth}). Bandwidth selection is possible by selecting $\beta^{*}$ at the IP and by setting the collimation angle, $\theta_{\mathrm{col}}$ that collects the scattered photons; these may for example comprise switchable, fixed-aperture collimators.

Within this optimisation derivation, the \textit{rms} bandwidth is given by (Eq.~\ref{eq:RMS_bandwidth}) with the emittance term (Eq.~\ref{eq:emittance_term}) modified for the transversely round bunch case
\begin{equation}
\frac{\sigma_{\epsilon}}{E_{\epsilon}} = \frac{2\gamma\epsilon_{n}}{\left(1+X\right)\beta^{*}},
\label{eq:emittance_term_round_beam}    
\end{equation}
where $X$ is the recoil parameter (Eq.~\ref{eq:X_geometry}), $\epsilon_{n}$ is the transverse \textit{rms} normalised emittance of the electron bunch and $\beta^{*}$ is the $\beta$-function at the interaction point.

Typically the dominant terms that define the bandwidth of an inverse Compton source Eq.~(\ref{eq:RMS_bandwidth}) are the collimation term Eq.~(\ref{eq:collimation_term}) and the emittance term Eq.~(\ref{eq:emittance_term_round_beam}). The free parameter of the collimation term is the collimation angle $\theta_{\mathrm{col}}$, which can be adjusted either by changing the collimator aperture or changing its distance from the IP. Adjustable collimators have been designed for the ELI-NP-GBS $\gamma$-ray ICS source~\cite{paterno2017collimation}; a similar design could be implemented at other ICS sources.

The emittance term is dependent both on the normalised transverse emittance $\epsilon_{n}$ and on $\beta^{*}$. It is more convenient to change the $\beta$-function at the IP with focusing rather than by varying the emittance, since the latter is dependent on the injector and collective effects prior to the IP. By using a larger $\beta^{*}$ and a small collimator aperture, it is possible to reduce the contribution of the collimation and emittance terms so that they are negligible; thus the electron bunch and laser pulse energy spread terms (Eq.~\ref{eq:beam_energy_spread_term},~\ref{eq:laser_energy_spread_term}) become dominant for accelerators with a sufficiently small emittance. This effectively places a lower limit on the bandwidth of an ICS source, i.e. it is limited by the energy spread of the electron beam $\Delta E_{e}/E_{e}$ and laser pulse $\Delta E_{\mathrm{laser}}/E_{\mathrm{laser}}$ as 
\begin{equation}
\left(\frac{\Delta E_{\gamma}}{E_{\gamma}}\right)_{\mathrm{min}} \approx \sqrt{\left[\left(\frac{2+X}{1+X}\right)\frac{\Delta E_{e}}{E_{e}}\right]^{2} + \left[\left(\frac{1}{1+X}\right)\frac{\Delta E_{L}}{E_{L}}\right]^{2}},
\label{eq:bandwidth_limitation_minimum}
\end{equation}
for the low recoil approximation.

Consequently, any bandwidth above this limit can be achieved by an ICS source by tuning of the collimation angle and $\beta^*$ so that a desired bandwidth, $\Delta E_{\gamma}/E_{\gamma}$, is achieved. Since the collimation and emittance terms are typically dominant, all other terms can be excluded and the solutions are bounded by
\begin{equation}
\frac{\Delta E_{\gamma}}{E_{\gamma}} > \sqrt{\left(\frac{ \sigma_{\theta}}{E_{\theta}}\right)^{2}+\left(\frac{\sigma_{\epsilon}}{E_{\epsilon}}\right)^{2}}.    
\end{equation}
This results in myriad combinations of $\beta^{*}$ and $\theta_{\mathrm{col}}$ that satisfy a particular chosen bandwidth larger than this lower limit (Eq.~\ref{eq:bandwidth_limitation_minimum}).

The different $\beta^{*}$, $\theta_{\mathrm{col}}$ combinations each give a different collimated flux; obviously we wish to chose the solution with the largest flux. The collimated flux $\mathcal{F}_{\Psi}$ of each solution is calculated based on a method valid for small collimation angles ($\gamma\theta_{\mathrm{col}} < 1$) derived by Curatolo et al.~\cite{curatolo2017analytical}. Re-cast for our variable definitions, it becomes
\textcolor{blue}{**THIS MAY CHANGE WITH THE NEW COLLIMATED FLUX METHOD**}
% serafini calc
\begin{equation}
\mathcal{F}_{\Psi}\propto \frac{\left(1+\sqrt[3]{X}\Psi^{2}/3\right)\Psi^{2}}{\left[1+\left(1+X/2\right)\Psi^{2}\right]\left(1+\Psi^{2}\right)}, 
\label{eq:curatolo_collimated_flux}
\end{equation}
where $\mathcal{F}$ is the total (uncollimated) flux, $\Psi = \gamma\theta_{\mathrm{col}}$ is the acceptance angle, and $X$ is the recoil parameter. The solution giving the maximal flux is selected.

It is not practicable to calculate the flux from every combination of $\beta^{*}$ and $\theta_{\mathrm{col}}$. Instead, an array of collimation angles $\theta_{\mathrm{col}}$ from 0 to $1/\gamma$ is used ($\gamma\theta_{\mathrm{col}}<1$), and for a given \textit{rms} bandwidth value the corresponding $\beta^*$ is calculated using
\begin{equation}
\beta^{*} = \frac{2\gamma\epsilon_{n}}{\left(1+X\right)\sqrt{\left(\frac{\Delta E_{\gamma}}{E_{\gamma}}\right)^{2}-\left[\left(\frac{\sigma_{\theta}}{E_{\theta}}\right)^{2}+\left(\frac{\sigma_{e}}{E_{e}}\right)^{2}+\left(\frac{\sigma_{L}}{E_{L}}\right)^{2}\right]}},
\label{eq:beta_star_round_beam}
\end{equation}
which is a rearrangement of (Eq.~\ref{eq:RMS_bandwidth}) with the emittance term in the transversely round bunch case (Eq.~\ref{eq:emittance_term_round_beam}). A near identical solution can also be found for the case of an FWHM bandwidth. 

The collimated flux $\mathcal{F}_{\Psi}$ is calculated for each combination produced via this method. The maximal collimated flux is selected and the combination of $\beta^{*}$ and $\theta_{\mathrm{col}}$ corresponding to this solution is returned. This process can be applied to the case of a target bandwidth to determine $\theta_{\mathrm{col}}$, $\beta^{*}$, and collimated flux in the selected bandwidth. In addition, applying this method to a continuum of bandwidths allows us to map the possible operational settings of our ICS source, and to derive tuning curves such as the variation of the collimated flux with bandwidth.

\section{2D Non-Round Beam Optimisation}

\subsection{Genetic Algorithm Approach}

\subsection{Nelder--Mead Approach}

\section{Development of the ICARUS Spectrum code}
\label{sec:development_of_the_ICARUS_spectrum_code}


\begin{multline}
\frac{dN_{\gamma}}{dE_{\gamma}} = \frac{r_{e}^{2}N_{e}N_{L}}{4\pi^{3}L^{2}\hbar c z_{R}\sigma_{e}\sigma_{k}}\int_{k_{\mathrm{min}}}^{k_{\mathrm{max}}}\int_{-\theta_{x,\mathrm{max}}}^{\theta_{x,\mathrm{max}}}\int_{-\theta_{y,\mathrm{max}}}^{\theta_{y,\mathrm{max}}}\int_{y_{\mathrm{min}}}^{y_{\mathrm{max}}}\int_{x_{\mathrm{min}}}^{x_{\mathrm{max}}}\frac{1}{\sqrt{\zeta_{x}\zeta_{y}}\sigma_{\theta_{x}}\sigma_{\theta_{y}}}\frac{\bar{\gamma}}{1+2\gamma E_{L}/m_{e}c^{2}} \\
\times\left\{\frac{1}{4}\left[\frac{4\bar{\gamma}^{2}E_{L}}{E_{\gamma}\left(1+\bar{\gamma}^{2}\theta^{2}\right)}+\frac{E_{\gamma}\left(1+\bar{\gamma}^{2}\theta^{2}\right)}{4\bar{\gamma}^{2}E_{L}}\right]-2\cos^{2}\left(\tau-\phi_{f}\right)\frac{\bar{\gamma}^{2}\theta^{2}}{\left(1+\bar{\gamma}^{2}\theta^{2}\right)^{2}}\right\} \\
\times\exp{\left[-\frac{\left(\theta_{x}-x_{d}/L\right)^{2}}{2\sigma_{\theta_{x}}^{2}}-\frac{\left(\theta_{y}-y_{d}/L\right)^{2}}{2\sigma_{\theta_{y}}^{2}}-\frac{\left(\bar{\gamma}-\gamma\right)^{2}}{2\sigma_{e}^{2}}-\frac{\left(k-k_{0}\right)^{2}}{2\sigma_{k}^{2}}\right]}dx_{d}dy_{d}d\theta_{y}d\theta_{x}dk,
\label{eq:ICARUS_equation}
\end{multline}

\begin{multline}
\frac{dN_{\gamma}}{dE_{\gamma}} \approx \frac{r_{e}^{2}N_{e}N_{L}}{2\pi^{2}\hbar c z_{R}\sqrt{\zeta_{x}}\sigma_{e}\sigma_{\theta_{x}}}\int_{-\theta_{x,\mathrm{max}}}^{\theta_{x,\mathrm{max}}}\int_{y_{\mathrm{min}}}^{y_{\mathrm{max}}}\int_{x_{\mathrm{min}}}^{\mathrm{max}}\frac{\bar{\gamma}}{1+2\gamma E_{L}/m_{e}c^{2}} \\
\times\left\{\frac{1}{4}\left[\frac{4\bar{\gamma}^{2}E_{L}}{E_{\gamma}\left(1+\bar{\gamma}^{2}\theta^{2}\right)}+\frac{E_{\gamma\left(1+\bar{\gamma}^{2\theta^{2}}\right)}}{4\bar{\gamma}^{2}E_{L}}\right]-\frac{\bar{\gamma}^{2}\theta^{2}}{\left(1+\bar{\gamma}^{2}\theta^{2}\right)^{2}}\right\} \\
\times\exp{\left[-\frac{\left(\theta_{x}-x_{d}/L\right)^{2}}{2\sigma_{\theta_{x}}^{2}}-\frac{\left(\bar{\gamma}-\gamma\right)^{2}}{2\sigma_{e}^{2}}\right]}d\theta_{x}dy_{d}dx_{d},
\label{eq:1D_sun_equation}    
\end{multline}
    



\section{Benchmarking of the ICARUS Spectrum Code}
\label{sec:benchmarking_of_the_ICARUS_spectrum_code}


The code \textsc{ICCS3D}, a generalization of the \textsc{ICCS} code \cite{krafft2016laser,ranjan2018simulation}, computes radiation produced in ICS within the linear Compton regime (when $a_{0}\ll 1$, and electron recoil is properly accounted for). In \textsc{ICCS3D}, a 3D laser pulse model replaces the 1D plane wave model used in \textsc{ICCS}. This modification is implemented in a manner described in Terzi\'c et al \cite{terzic2019improving}; instead of all electrons experiencing the same laser field strength $a_{0}$, as they do for a 1D plane wave, their effective laser field strength is dependent on the electron's distance from the laser's center at the moment of scattering. To calculate anticipated spectral output, \textsc{ICCS3D} can use either the parameters of the electron beam or an arbitrary electron distribution, in addition to the laser parameters. For Fig.~\ref{fig:cbetaspectrumplot}, a particle distribution was tracked by \textsc{Tao} \cite{TaoManual} through the bypass lattice to the IP; the distribution at the IP was provided to \textsc{ICCS3D}. When compared to the spectrum calculated using only the electron beam parameters, the differences were negligible. 

\textcolor{blue}{Paper Footnote: Note that there is an error in Equation 53 of Sun's paper, where the prefactor gives that $dN/dE\propto L^{2}$ for a source-to-collimator distance $L$. Clearly, this should be $dN/dE\propto 1/L^{2}$; the other parts of the given equation are correct.}

\textsc{ICARUS}: inverse Compton scattering semi-analytic recoil-corrected ultra-relativistic spectrum code, uses a modified and corrected  version of the 2D formalism of Sun et al. \cite{sun2011theoretical}. \textsc{ICARUS} integrates the photons at small energy intervals that pass through a given 2D collimator aperture (here circular) for the fundamental laser mode. This code assumes that the electron bunch has a 3D Gaussian distribution, approximates the laser as a 3D Gaussian pulse, and is currently only valid for the head-on ($\phi = 0$) geometry; it has been validated against \textsc{ICCS3D} as well as the analytical results of this paper. 

 

\end{document}