%%%%%%%%%%%%%%%%%%%%%%%%%%%%%%%%%%%%%%%%%%%%%%%%%%%%%%%%%%
%
% Doctoral Thesis Template @ The University of Manchester
% LaTeX Chapter Template
% Version 1 (23/07/2020)
% Joe Crone
%
% This template is based on:
% The University of Manchester, Presentation of Thesis Policy
% Research Office Graduate Education Team
% June 2017
% http://www.regulations.manchester.ac.uk/pgr-presentation-theses/
%
%%%%%%%%%%%%%%%%%%%%%%%%%%%%%%%%%%%%%%%%%%%%%%%%%%%%%%%%%%
\documentclass[../main.tex]{subfiles}
\begin{document}

% Title
%--------------------------------------------------------
\chapter{Optimisation and Characterisation of Inverse Compton Scattering Spectra}
\label{Optimisation_and_Characterisation_of_Inverse_Compton Scattering_Spectra} % to reference use \ref{ChapterTemplate}

\section{Motivation for Characterisation of ICS Sources}

Proper characterisation of inverse Compton scattering sources is necessary to quantify source performance and for reliable comparison between ICS sources. Through design and benchmarking of models used to predict ICS source performance and spectra, the ICS interactions of electron bunches and laser pulses can be better understood, which can reveal methods of improving narrowband ICS source design such as the optimisation strategies covered later in this chapter, which were motivated by the characterisation methods developed at the start of this chapter. The spectral output parameters presented in Chapter~\ref{Photon_Production_by_Inverse_Compton_Scattering}, such as flux and (average and peak) brilliance, fail to account for collimation effects, energy spread of the electron bunch and spectral bandwidth of the laser pulse. Therefore, models are required to predict the spectrum of the ICS source and quantify the effect of these parameters on the radiation spectrum available to the users of an ICS source. 

Within the first half of this chapter, an analytical approach to calculating the collimated flux produced by an ICS source is developed and compared to existing methods \cite{curatolo2017analytical} and a semi-analytical spectrum code \textsc{ICARUS}: Inverse Compton Scattering semi-Analytical Recoil-corrected Ultra-relativistic spectrum code. Based on the model by Sun et al \cite{sun2009characterizations,sun2011theoretical} (which is corrected and expanded) \textsc{ICARUS} is developed as part of the present work and benchmarked using the \textsc{ICCS3D} code \cite{krafft2016laser,ranjan2018simulation}. The characterisation methods developed here are tested through the use of three cases outlined in Section~\ref{sec:benchmarking_cases_characterisation_optimisation}, specified to cover the range of accelerators used to provide electron bunches to ICS sources; the characterisation methods may be generalised to any accelerator, not just ERLs.

An analytical collimated flux equation has been developed in Section~\ref{sec:analytical_collimated_flux} in order to provide a quick, reliable method to predict the flux of an ICS source (post-collimation) with minimal assumptions. The analytical collimated flux calculation can also be used to predict the total photon yield (flux) of spectrum codes. Using the analytical method, more accurate simulations such as those from spectrum codes can be evaluated. A wide variety of effects such as collimation, angular crossing and hourglass effects -- described in Section~\ref{sec:geometric_luminosity_reduction} -- have been incorporated into this methodology whilst the effect of energy spread of the electron bunch and spectral bandwidth of the laser pulse are presently neglected. Large-scale spectrum code simulations require can require inordinate computational time \cite{ranjan2018simulation}, on the order of hours (often with parallel computing, see Section~\ref{sec:ICARUS_benchmarking}) whereas analytical calculations can be evaluated in sub-second timescales. For example, the commonly Monte Carlo spectrum code code  Conglom\'{e}rat d'ABEL et d'Interactions Non-Lin\'{e}aires (\textsc{CAIN}) \cite{chen1995cain} can take on the order of 1200 minutes (20 hours) to produce a single spectrum \cite{sun2011theoretical}. Therefore, analytical collimated flux calculations are easier to apply to optimisation procedures rather than the more accurate spectral yield calculations from spectrum codes.

The \textsc{ICARUS} spectrum code developed as part of the current work is a dedicated semi-analytical inverse Compton scattering code, which models the electron bunch--laser pulse interaction assuming Gaussian distributions and produces an ICS interaction spectrum
accounting for phenomena such as the recoil of the electron bunch, the emittance and divergence effects of the electron bunch and laser pulse, and their energy spread and spectral bandwidth, and rectangular and circular collimation. However, the \textsc{ICARUS} spectrum code is limited to linear ICS interactions ($a_{0} \ll 1$) and head-on interactions ($\phi=0$). Linear ICS interactions are those most commonly used for the 're-circulated pulse' approach described in Section~\ref{sec:lasers_fabry_perot}, which uses non-intense (small pulse energy) laser pulses, and are the sources we aim to model here. The luminosity of head-on interactions can be simply corrected for an angular crossing using Eq.~\ref{eq:angular_crossing_factor} and an angular crossing causes only small correction to the scattered photon energy (Eq.~\ref{eq:scattered_photon_energy}). For example, a  500~\si{\mega\electronvolt} electron beam incident head-on with a Nd:YAG laser pulse ($\lambda = 1064$~\si{\nano\meter}) scatters 4.43~\si{\mega\electronvolt} $\gamma$-rays, whereas with a 10~\si{\degree} crossing angle, the scattered photon energy is 4.40~\si{\mega\electronvolt} -- a small $\sim0.7$\% reduction. Methods have also been developed to calculate the flux of the ICS source post collimation from the \textsc{ICARUS} spectrum, as explained in Section~\ref{sec:development_of_the_ICARUS_spectrum_code}. 

The \textsc{ICARUS} spectrum code is compared to two other commonly used ICS spectrum codes: \textsc{CAIN}) \cite{chen1995cain}, a Monte Carlo code for simulation of a broad range of electromagnetic interactions, and the Improved Codes for Compton Simulation (\textsc{ICCS3D}) \cite{krafft2016laser,ranjan2018simulation}, a semi-analytical code for simulation of ICS interactions. The comparison is detailed in Section~\ref{sec:benchmarking_of_the_characterisation_methods} where spectrum simulation methods are discussed and the \textsc{ICARUS} code is benchmarked against the \textsc{ICCS3D} code.

\section{Analytical Collimated Flux}
\label{sec:analytical_collimated_flux}

In this section the collimated flux - the total flux collected within a circular aperture of semi-angle $\theta_{\mathrm{col}}$ -- is derived from first principles, using the work of Berestetskii et al \cite{berestetskii1982quantum}. Unlike others in the literature \cite{curatolo2017analytical}, this method is valid for the angular crossing case ($\phi>0$) as the effect of the crossing angle is encompassed in the cross section as well as the geometric beam--beam angular crossing effect described in Section~\ref{sec:geometric_luminosity_reduction}. The hourglass effect is also fully accounted for in this method, using the prescription of Miyahara \cite{miyahara2008luminosity}. The collimated flux calculation method is a semi-analytic calculation, requiring numerical integration, that is valid within the recoil regime ($X>0$) but is only valid for the linear inverse Compton scattering case ($a_{0}\ll1$). The results of this derivation are benchmarked against a series of other methods including the collimated flux formula by Curatolo et al \cite{curatolo2017analytical} and the \textsc{ICARUS} and \textsc{ICCS3D} spectrum codes in Section~\ref{sec:benchmarking_of_the_characterisation_methods}.

The electron-photon interaction cross section dependence of the scattering angle $\theta$ dependence can be introduced via the differential cross section with respect to $Y$ (Eq.~\ref{eq:differential_cross_section_Y_invariant}), as derived in Section~\ref{sec:electron_photon_interaction_cross_section}
\begin{equation*}
\frac{d\sigma}{dY} = \frac{8\pi r_{e}^{2}}{X^{2}}\left[\left(\frac{1}{X}-\frac{1}{Y}\right)^{2}+\frac{1}{X}-\frac{1}{Y}+\frac{1}{4}\left(\frac{X}{Y}+\frac{Y}{X}\right)\right]
\end{equation*}
where $r_{e}$ is the classical radius of the electron and $X$ (Eq.~\ref{eq:X_geometry}) and $Y$ (Eq.~\ref{eq:Y_geometry}) are the Lorentz invariants. The Lorentz $X$ invariant derived in Section~\ref{sec:electron_photon_interactions} intrinsically has no scattering angle dependence because it relates to the centre of mass $s$ Mandelstam variable. Therefore, the scattering angle dependency originates from the $Y$ Lorentz invariant given by
\begin{equation*}
Y = \frac{2\gamma E_{\gamma}\left(1-\beta\cos\theta\right)}{m_{e}c^{2}},    
\end{equation*}
previously derived in Section~\ref{sec:electron_photon_interactions}. Inspecting (Eq.~\ref{eq:Y_geometry}), we see $Y$ is dependent on $E_{\gamma}$, the scattered photon energy (Eq.~\ref{eq:scattered_photon_energy}) derived in Section~\ref{sec:derivation_of_the_scattered_photon_energy}, which has an explicit dependence on scattering angle. Consequently, expansion of $Y$ in terms of the scattering angle dependence of the scattered photon energy (Eq,~\ref{eq:scattered_photon_energy}) is necessary, which can be simplified in terms of the recoil parameter $X$ (Eq.~\ref{eq:X_geometry}),
\begin{equation}
Y = \frac{2\gamma E_{L}\left(1+\beta\cos\phi\right)\left(1-\beta\cos\theta\right)}{m_{e}c^{2}\left\{1-\beta\cos\theta+\left[1+\cos\left(\phi+\theta\right)\right]E_{L}/E_{e}\right\}} = \frac{X\left(1-\beta\cos\theta\right)}{1-\beta\cos\theta+\left[1+\cos\left(\phi+\theta\right)\right]E_{L}/E_{e}}.
\label{eq:Y_geometry_expanded}
\end{equation}
The derivative of the expanded $Y$ Lorentz invariant (Eq.~\ref{eq:Y_geometry_expanded}) is then found using the quotient rule and the full derivative of $Y$ by $\theta$ becomes
\begin{equation}
\frac{dY}{d\theta} = \frac{X\beta\sin\theta\zeta-X\left(1-\beta\cos\theta\right)\left[\beta\sin\theta-\sin\left(\phi+\theta\right)E_{L}/E_{e}\right]}{\zeta^{2}}.
\label{eq:dY_dtheta}
\end{equation}
which is can be parameterised using
\begin{equation}
\zeta = 1-\beta\cos\theta+\left[1+\cos\left(\phi+\theta\right)\right]E_{L}/E_{e},
\label{eq:zeta_simplification}
\end{equation}
which has no known physical meaning other than to simplify expression of (Eq.~\ref{eq:dY_theta}).

The dependence of the cross section $\sigma$ on the scattering angle $\theta$ is given by the chain rule
\begin{equation}
\frac{d\sigma}{d\theta} = \frac{d\sigma}{dY}\frac{dY}{d\theta},
\label{eq:cross_section_chain_rule}
\end{equation}
therefore the integrated cross section collected within a collimation angle $\theta_{\mathrm{col}}$ becomes
\begin{equation}
\sigma\left(\theta_{\mathrm{col}}\right) = \int_{0}^{\theta_{\mathrm{col}}}\frac{d\sigma}{dY}\frac{dY}{d\theta}d\theta,
\label{eq:cross_section_integral}
\end{equation} 
where $d\sigma/dY$ is given by (Eq.~\ref{eq:differential_cross_section_Y_invariant}) and $dY/d\theta$ is given by (Eq.~\ref{eq:dY_dtheta}). The derivation of the recoil parameter $X = 2\gamma E_{L}\left(1+\beta\cos\phi\right)/m_{e}c^{2}$ (Eq.~\ref{eq:X_geometry}) accounts for the crossing angle $\phi$ between the incident photon and the incident electron, therefore this cross section is fully generalised for any interaction geometry.

The collimated flux can then be derived by modifying the flux equation $\mathcal{F} = \sigma\mathcal{L}_{\mathrm{HEAD-ON}}f$ (Eq.~\ref{eq:headon_flux}) derived in Section~\ref{sec:luminosity_and_flux}, where the total cross section $\sigma$ (Eq.~\ref{eq:full_scattering_angle_cross_section_integral}) is replaced by the scattering angle dependent cross section $\sigma\left(\theta_{\mathrm{col}}\right)$ (Eq.~\ref{eq:cross_section_integral}). The geometric luminosity angular crossing and the hourglass effect, as explained in Section~\ref{sec:geometric_luminosity_reduction}, are accounted for using the luminosity reduction factor $R_{ACHG}$ (Eq.~\ref{eq:miyahara_combined_reduction}) by Miyahara \cite{miyahara2008luminosity} which generalises the collimated flux for any electron bunch--laser pulse interaction geometry. Similarly to  the uncollimated flux $\mathcal{F} = \sigma R_{ACHG}\mathcal{L}_{\mathrm{HEAD-ON}}$ (Eq.~\ref{eq:flux_angular_crossing_hourglass}), the collimated flux becomes
\begin{equation}
\mathcal{F}_{\mathrm{col}} = \sigma\left(\theta_{\mathrm{col}}\right) R_{ACHG}\mathcal{L}_{\mathrm{HEAD-ON}}f.
\label{eq:collimated_flux}
\end{equation}

It is implicit in (Eq.~\ref{eq:collimated_flux}) that the interaction occurs from a point source -- the transverse and longitudinal positions of the electrons within the bunch are neglected. However, as explained in Section~\ref{sec:source_size_divergence}, this point source approximation is valid whilst the transverse (longitudinal) source size is much smaller than the collimator aperture radius (source-to-collimator distance); this may be termed far-field collimation. For example, a collimator placed 10~\si{\meter} downstream of a $\gamma$-ray ICS source will have an aperture radius on the order of millimetres whereas the transverse source size of the electron bunch-laser pulse interaction is typically 10's~\si{\micro\meter} and the longitudinal source size is typically around 1~\si{\milli\meter}. 

Within (Eq.~\ref{eq:collimated_flux}) the effect of the energy spread of the electron bunch, the spectral bandwidth and partially, via the point source approximation, the effect of the emittance (spatial extent) of the bunch are neglected. The codes \textsc{ICARUS} and \textsc{ICCS3D} \cite{krafft2016laser,ranjan2018simulation} properly take into account the energy spread factors but the emission position problem is only solved in Monte Carlo codes such as \texsc{CAIN} \cite{chen1995cain}. However, this effect is expected to be small as most ICS sources use far-field collimation.

\section{Development of the ICARUS Spectrum code}
\label{sec:development_of_the_ICARUS_spectrum_code}

\textsc{ICARUS}: Inverse Compton Scattering semi-Analytic Recoil-corrected Ultra-relativistic Spectrum code, uses a modified (and corrected) version of the 2D ICS spectrum model developed by Sun et al. \cite{sun2009characterizations,sun2011theoretical} to generate the spectrum of radiation produced by an ICS source. The \textsc{ICARUS} code is valid for large electron recoil ($X>0$) and for the linear regime ($a_{0}\ll 1$). \textsc{ICARUS} calculates the number of photons produced in small energy intervals that pass through a given far-field collimator aperture (circular or rectangular) for the fundamental harmonic of the laser (or incident photon source). The simulated radiation spectrum is that observable at a detector placed downstream of a collimator. \textsc{ICARUS} assumes that the electron bunch and laser pulse are modelled by 3D Gaussian distributions, and can account for both circularly and linearly polarised incident photons. It is assumed that the collimator is placed far enough from the interaction that the source size of the interaction can be viewed as a point source (far-field collimation) However, \textsc{ICARUS} is currently only written to calculate for a head-on ($\phi = 0$) geometry.

Using a result of Sun et al \cite{sun2009characterizations,sun2011theoretical}, the distribution of a ICS scattered $\gamma$-ray beam produced by a head-on collision of an electron bunch and laser pulse is given by \textcolor{blue}{**HERE -- ON THIS SECTION**}
\begin{multline}
\frac{dN_{\gamma}}{d\Omega_{c} dE_{\gamma}} = N_{e}N_{L}\int \frac{d\sigma}{d\Omega}\delta\left(\bar{E_{\gamma}}-E_{\gamma}\right)c\left(1+\beta\right)f_{e}\left(x,y,z,x',y',p,t\right)\\ \times f_L\left(x,y,z,k,t\right)dx'~dy'~dp~dk~dV~dt,
\label{eq:central_distribution_sun}
\end{multline}
where the differential solid angle of the scattered photons incident on the collimator is $d\Omega_{c} = dx_{c}dy_{c}/L^{2}$ assuming $dx_{c} \ll L$, as shown in Fig.~\ref{fig:solid_angle_collimator_geometry}, with $x_{c}$ and $y_{c}$ the $x$ and $y$ positions at the collimator and $L$ the source-to-collimator distance. $d\sigma/d\Omega$ is the differential cross section of the ICS interaction, $\delta\left(\bar{E_{\gamma}}-E_{\gamma}\right)$ is a delta function which encapsulates energy conservation in the process and is integrated using the momentum integration variable $dp$ (after conversion), with $\bar{E_{\gamma}}$ the maximum possible energy a $\gamma$-ray may have for a scattering angle $\theta$,  $E_{\gamma}$ the actual $\gamma$-ray energy. The $c\left(1+\beta\right)$ term is a conversion from position to time and $N_{e}f_{e}$ and $N_{L}f_{L}$ are the phase-space density distributions of the electron bunch (Eq.~\ref{eq:electron_gaussian_intensity_distribution}) and laser pulse (Eq.~\ref{eq:laser_gaussian_intensity_distribution}) as modelled by Gaussian distributions, $dx'$ and $dy'$ are the divergence integration variables in $x$ and $y$ respectively, $dp$ is momentum of the electron integration variable, $dk$ is the wavenumber of the laser pulse integration variable, $dV$ is used to integrate the volume of the laser pulse--electron bunch interaction (source size in $x$, $y$ and $z$) and  $dt$ is the interaction time integration variable. 
\begin{figure}[!h]
\centering
\includegraphics[width=\textwidth]{Figures/Optimisation_and_Characterisation_of_Inverse_Compton_Scattering_Sources/collimator_geometry_solid_angle.pdf}
\caption{Schematics and geometry of inverse Compton scattering modelled in the \textsc{ICARUS} code, based on Sun et al's model \cite{sun2009characterizations,sun2011theoretical}. Left: A photon (red) is scattered from an electron bunch--laser pulse interaction (purple) with polar scattering angle $\theta$ and incident on a collimator (grey) a distance $L$ downstream. Scattered photons from the interaction are produced within in a cone of polar angle $\theta=1/\gamma$. For the photon of interest (red) $\theta < \theta_{\mathrm{col}}$, with the collimation angle $\theta_{\mathrm{col}}$, therefore the scattered photon passes through the face of the collimator at position $P = \left(x_{c},y_{c},L\right)$, with aperture radius $a$. Assuming an infinitesimal variation in angle $d\theta$, the differential solid angle of photons are contained within an are $dx_{c}dy_{c}$ at the face of the collimator. Right: An electron (red) with angular divergence $x'$ interacts with an incident photon photon (not shown) at position $O$ and a photon is scattered (green) which passes through a collimator placed a distance $L$ downstream. The photon is scattered with a polar angle $\theta_{x}$ in the horizontal plane and passes through the collimator at position $P$.}
\label{fig:solid_angle_collimator_geometry}
\end{figure}

The differential cross section for a head-on ($\phi=0$) collision in this model is given by
\begin{equation}
\frac{d\sigma}{d\Omega} = 8\pi r_{e}^{2}\left\{\frac{1}{4}\left[\frac{4\gamma^{2}E_{L}}{\bar{E_{\gamma}}\left(1+\gamma^{2}\theta^{2}\right)}+\frac{\bar{E_{\gamma}}\left(1+\gamma^{2}\theta^{2}\right)}{4\gamma^{2}E_{\gamma}}\right]-2\cos^{2}\left(\tau-\phi_{f}\right)\frac{\gamma^{2}\theta^{2}}{\left(1+\gamma^{2}\theta^{2}\right)^{2}}\right\}\left(\frac{\bar{E_{\gamma}}}{4\gamma E_{L}}\right)^{2},
\label{eq:sun_differential_cross_section}    
\end{equation}
where $\bar{E_{\gamma}}$ is the scattered photon energy for a particular scattering angle $\theta$ in the small angle approximation for a head-on ($\phi=0$) collision (Eq.~\ref{eq:small_angle_scattered_photon_energy}),
\begin{equation}
\bar{E_{\gamma}} = \frac{4\gamma^{2}E_{L}}{1+\gamma^{2}\theta^{2}+\frac{4\gamma E_{L}}{m_{e}c^{2}}}.
\label{eq:sun_Egamma_bar}    
\end{equation}
The angular divergences of the scattered photons $x'$ and $y'$ and their initial horizontal $x$ and vertical $y$ position can be expressed in terms of the projection of the scattering angle of the produced radiation in each plane $\theta_{x}'$ and $\theta_{y}'$ ($\theta = \sqrt{\theta_{x}^{2}+\theta_{y}^{2}}$) using the relations
\begin{align}
\theta_{x}' + x' &= \frac{x_{c}-x}{L}, & \theta_{y}' + y' &= \frac{y_{c}-y}{L}
\label{eq:sun_angular_divergence}    
\end{align}
which arise from the geometric constraints, shown in Fig.~\ref{fig:solid_angle_collimator_geometry}, of a photon passing through a far-field ($L \gg \sqrt{x_{c}^{2}+y_{c}^{2}}$) collimator at a position $\left(x_{c},y_{c}\right)$ on the collimator face. Here, the angular divergences of the laser pulse have been neglected. The model can be simply extended for collimator misalignment, through addition of a simple error term \cite{sun2009characterizations} $x_{\mathrm{err}}$ or $y_{\mathrm{err}}$, where the angular divergences become
\begin{align}
\theta_{x}' + x' &= \frac{x_{c}-x-x_{\mathrm{err}}}{L}, &
\theta_{y}' + y' &= \frac{y_{c}-y-y_{\mathrm{err}}}{L}
\label{eq:sun_collimator_misallignment}    
\end{align}

Applying (Eq.~\ref{eq:sun_angular_divergence}) and integrating (Eq.~\ref{eq:central_distribution_sun}) with respect to $dV$, the laser pulse--electron bunch interaction volume or overlap, and $dt$ the interaction time, whilst expanding the differential solid angle $d\Omega = dx_{c}dy_{c}/L^{2}$, the Gaussian density distributions of the electron bunch (Eq.~\ref{eq:electron_gaussian_intensity_distribution}) and laser pulse (Eq.~\ref{eq:laser_gaussian_intensity_distribution}), yields
\begin{multline}
\frac{dN_{\gamma}}{dE_{\gamma}dx_{c}dy_{c}} = \frac{N_{e}N_{L}}{\left(2\pi\right)^{3}z_{R}\sigma_{p}\sigma_{k}L^{2}}\int \frac{k}{\sqrt{\zeta_{x}\zeta_{y}}\sigma_{\theta_{x}}\sigma_{\theta_{y}}}\frac{d\sigma}{d\Omega}\delta\left(\bar{E_{\gamma}}-E_{\gamma}\right)\left(1+\beta\right) \\\times\exp\left[-\frac{\left(\theta_{x}-x_{c}/L\right)^{2}}{2\sigma_{\theta_{x}}^{2}}-\frac{\left(\theta_{y}-y_{c}/L\right)^{2}}{2\sigma_{\theta_{y}}^{2}}-\frac{\left(p-p_{0}\right)^{2}}{2\sigma_{p}^{2}}-\frac{\left(k-k_{0}\right)^{2}}{2\sigma_{k}^{2}}\right]~d\theta_{x}~d\theta_{y}~dp~dk,
\label{eq:sun_volume_time_integral}    
\end{multline}
where $z_{R}$ is the Rayleigh range of the laser pulse (Eq.~\ref{eq:rayleigh_range}), and the $\zeta_{x/y}$, $\sigma_{\theta_{x/y}}$ and $\xi_{x/y}$ parameters in each plane are given by
\begin{align}
\zeta_{x} &= 1+\frac{2k\beta_{x}\epsilon_{x}}{z_{R}}, & \sigma_{\theta_{x}} &= \sqrt{\frac{\epsilon_{x}\xi_{x}}{\beta_{x}\zeta_{x}}}, & \xi_{x} &= 1+\left(\alpha_{x}-\frac{\beta_{x}}{L}\right)^{2}+\frac{2k\beta_{x}\epsilon_{x}}{z_{R}}, \nonumber\\
\zeta_{y} &= 1+\frac{2k\beta_{y}\epsilon_{y}}{z_{R}}, & \sigma_{\theta_{y}} &= \sqrt{\frac{\epsilon_{y}\xi_{y}}{\beta_{y}\zeta_{y}}}, & \xi_{y} &= 1+\left(\alpha_{y}-\frac{\beta_{y}}{L}\right)^{2}+\frac{2k\beta_{y}\epsilon_{y}}{z_{R}}, 
\label{eq:zeta_sigmatheta_xi_parameters_sun}
\end{align}
where $p_{0}$ is the reference momentum of the electron bunch and $k_{0}$ is the centroid wavenumber of the laser pulse (the wavenumber of the fundamental harmonic). Note that there is an algebraic error in Sun et al's \cite{sun2009characterizations,sun2011theoretical} derivation; Sun et al's pre-factor states that $dN/dE\propto L^{2}$. The error occurs because of a mishandling of the detector solid angle. Clearly, this should be $dN/dE\propto 1/L^{2}$, since if the source to collimator distance $L$ is increased the number of photons through the collimator should decrease. This is corrected within the \textsc{ICARUS} code. 

The delta function, encompassing the energy conservation of the interaction, can be rewritten in terms of the Lorentz factor
\begin{equation}
\delta\left(\bar{E_{\gamma}}-E_{\gamma}\right) = -\delta\left(\gamma-\bar{\gamma}\right)\frac{\left(1+\bar{\gamma}^{2}\theta^{2}+\frac{4\bar{\gamma}E_{L}}{m_{e}c^{2}}\right)^{2}}{8\bar{\gamma}E_{L}\left(1+\frac{2\bar{\gamma}E_{L}}{m_{e}c^{2}}\right)},
\label{eq:sun_electron_energy_delta_function}    
\end{equation}
where $\bar{\gamma}$ is given by
\begin{equation}
\bar{\gamma} = \frac{2E_{\gamma}E_{L}}{m_{e}c^{2}\left(4E_{L}-E_{\gamma}\theta^{2}\right)}\left[1+\sqrt{1+\frac{4E_{L}-E_{\gamma}\theta^{2}}{4E_{L}^{2}E_{\gamma}/\left(m_{e}c^{2}\right)^{2}}}\right].    
\end{equation}

Substituting for $\delta\left(\bar{E_{\gamma}}-E_{\gamma}\right)$ using (Eq.~\ref{eq:sun_electron_energy_delta_function}) and simply exchanging the electron bunch momentum variable from $dp$ to $d\gamma$, (Eq.~\ref{eq:sun_volume_time_integral}) is integrated with respect to $d\gamma$ to introduce the electron bunch energy spread variation, which becomes
\begin{multline}
\frac{dN_{\gamma}}{dE_{\gamma}} = \frac{r_{e}^{2}N_{e}N_{L}}{4\pi^{3}L^{2}\hbar c z_{R}\sigma_{\gamma}\sigma_{k}}\int_{k_{\mathrm{min}}}^{k_{\mathrm{max}}}\int_{-\theta_{x,\mathrm{max}}}^{\theta_{x,\mathrm{max}}}\int_{-\theta_{y,\mathrm{max}}}^{\theta_{y,\mathrm{max}}}\int_{y_{\mathrm{min}}}^{y_{\mathrm{max}}}\int_{x_{\mathrm{min}}}^{x_{\mathrm{max}}}\frac{1}{\sqrt{\zeta_{x}\zeta_{y}}\sigma_{\theta_{x}}\sigma_{\theta_{y}}}\frac{\bar{\gamma}}{1+2\gamma E_{L}/m_{e}c^{2}} \\
\times\left\{\frac{1}{4}\left[\frac{4\bar{\gamma}^{2}E_{L}}{E_{\gamma}\left(1+\bar{\gamma}^{2}\theta^{2}\right)}+\frac{E_{\gamma}\left(1+\bar{\gamma}^{2}\theta^{2}\right)}{4\bar{\gamma}^{2}E_{L}}\right]-2\cos^{2}\left(\tau-\phi_{f}\right)\frac{\bar{\gamma}^{2}\theta^{2}}{\left(1+\bar{\gamma}^{2}\theta^{2}\right)^{2}}\right\} \\
\times\exp{\left[-\frac{\left(\theta_{x}-x_{c}/L\right)^{2}}{2\sigma_{\theta_{x}}^{2}}-\frac{\left(\theta_{y}-y_{c}/L\right)^{2}}{2\sigma_{\theta_{y}}^{2}}-\frac{\left(\gamma-\gamma_{0}\right)^{2}}{2\sigma_{\gamma}^{2}}-\frac{\left(k-k_{0}\right)^{2}}{2\sigma_{k}^{2}}\right]}dx_{c}~dy_{c}~d\theta_{y}~d\theta_{x}~dk,
\label{eq:ICARUS_equation}
\end{multline}
where the cross section has been expanded in terms of the $\bar{\gamma}$ parameter with a normalisation factor in the third term $\bar{\gamma}/\left(1+2\gammaE_{L}/m_{e}c^{2}\right)$, $\gamma_{0}$ is the centroid Lorentz factor of the electron bunch, $\sigma_{\gamma} = \sigma_{e}/m_{e}c^{2}$ is the spread of the Lorentz factor of the electron bunch. Integral limits have been imposed on each of the integrations, as discussed below. 

The integral over the horizontal collimator aperture $dx_{c}$ (and vertical collimator aperture $dy_{c}$) can be carried out with the limits $x_{\mathrm{min}}$ and $x_{\mathrm{max}}$ ($y_{\mathrm{min}}$ and $y_{\mathrm{max}}$), which vary dependent on the collimator shape (rectangular or circular) and based on the specified collimator dimensions. For the circular collimation case, the relationship $R\geq\sqrt{x_{c}^{2}+y_{c}^{2}}$ must be obeyed with $R$ the radius of the collimator. For example, in the horizontal $x$ plane $x_{\mathrm{min}} = -R$ and $x_{\mathrm{max}} = R$, such that the limits are equal to the radius of the collimator. However this means that in the $y$ plane, the collimator position integration limits are a function of $x_{c}$: $y_{\mathrm{min}} = -\sqrt{R^{2}-x_{c}^{2}}$ and $y_{\mathrm{max}} = \sqrt{R^{2}-x_{c}^{2}}$. For the case of a rectangular collimator these limits can be treated independently.  
Integrals over the projection of the scattering angles in each plane are carried out using the limits
\begin{align}
\theta_{x,\mathrm{max}} &= \sqrt{\frac{4E_{L}}{E_{\gamma}}-\theta_{y}^{2}}, & \theta_{x,\mathrm{min}} &= -\sqrt{\frac{4E_{L}}{E_{\gamma}}-\theta_{y}^{2}} \nonumber\\ 
\theta_{y,\mathrm{max}} &= \sqrt{\frac{4E_{L}}{E_{\gamma}}}, & \theta_{y,\mathrm{min}} &= -\sqrt{\frac{4E_{L}}{E_{\gamma}}},  
\end{align}
which constrains the angular cone the radiation is produced into to a maximum of a $1/\gamma$ cone in two dimensions as shown in Fig.~\ref{fig:solid_angle_collimator_geometry}. The collimation angle $\theta_{\mathrm{col}}$ typically limits $E_{\gamma}$ so that the cone is smaller than $1/\gamma$. As the limits of the $\theta_{x}$ integral are dependent on $\theta_{y}$, the order of integration is constrained and $\theta_{y}$ must be evaluated before $\theta_{x}$. The order of integration is similarly constrained for the integration over a circular collimator aperture -- $x_{c}$ must be integrated before $y_{c}$.

The wavenumber $k$ of the incident laser photon in Sun et al's model \cite{sun2009characterizations,sun2011theoretical} is integrated from $0$ to $\infty$, to reflect a summation over all possible laser harmonics. However, this is impractical in a real simulation and unnecessary for laser driven sources because the fundamental harmonic is the laser wavelength of interest and other harmonics will have a negligibly weak contribution to the spectra for $a_{0} \ll 1$ or be excluded completely due to re-circulation in a Fabry-Perot optical cavity. Therefore, the limit for the integration of the wavenumber of the incident laser is set to $k\pm3\sigma_{k}$, representing the 3$\sigma_{k}$ spectral bandwidth tail of the laser pulse, as the laser pulse is modelled using a Gaussian distribution.  

The polarisation term $2\cos^{2}\left(\tau-\phi_{f}\right)$ in (Eq.~\ref{eq:ICARUS_equation}) represents the polarisation of the scattered photon, as defined in Section~\ref{sec:electron_photon_interaction_cross_section}. The polarisation term must be modified as the azimuthal scattering angle $\phi_{f}$ is not explicitly integrated; instead the azimuthal scattering angle dependence is incorporated into the integration of the projection of the scattering angles ($\theta_{x}$ and $\theta_{y}$) in each plane. Therefore, the azimuthal scattering angle $\phi_{f}$ in (Eq.~\ref{eq:ICARUS_equation}) is replaced by $\phi_{f} = \cos^{-1}\left(\theta_{x}/\theta\right)$. The $x$ plane projection of the scattering angle $\theta_{x}$ is selected over the $y$ plane due to order of integration constraints. However, when integrated over the full azimuthal scattering angle ($0 \leq \phi_{f} \leq 2\pi$) -- which occurs during the spectrum code simulation -- the polarisation term has no effect, as expected from the derivation of the total electron--photon interaction cross section (Eq.~\ref{eq:compton_cross_section}) in Section~\ref{sec:electron_photon_interaction_cross_section}. 

\begin{figure}[!h]
\centering
\includegraphics[width=0.7\textwidth]{Figures/Optimisation_and_Characterisation_of_Inverse_Compton_Scattering_Sources/CaseA_example.pdf}
\caption{Example spectrum (spectral density against scattered photon energy) produced using \textsc{ICARUS} for the case A parameters defined in Section~\ref{sec:benchmarking_cases_characterisation_optimisation}. A total of 100 scattered photon energy intervals in the range $E_{\gamma} = 16.9$~\si{\mega\electronvolt} to $E_{\gamma} = 17.6$~\si{\mega\electronvolt} are calculated. The peak spectral density occurs when photons are backscattered ($\theta=0$) and corresponds to a scattered photon energy given by (Eq.~\ref{eq:compton_edge_energy}) -- the Compton edge energy. }
\label{fig:example_ICARUS_caseA}
\end{figure}
The \textsc{ICARUS} spectrum code is a \textsc{Mathematica} script with pre-input of ICS source parameters (electron bunch and laser pulse parameters), often from the optimisations later in this chapter, and post-processing which produces plots of spectra from the developed model and calculates the spectral yield (collimated flux) from the produced spectrum. An example \textsc{ICARUS} spectrum is shown in Fig.~\ref{fig:example_ICARUS_caseA}, using electron bunch and laser parameters presented later in this chapter in Tables~\ref{tab:char_opt_electron_bunch_parameters} and \ref{tab:char_opt_laser_pulse_parameters}. The spectral yield (collimated flux) is calculated by 
\begin{equation}
\mathcal{F}_{\mathrm{col}} = R_{AC}f\mathcal{F}_{\mathrm{\textsc{ICARUS}}}    
\label{eq:spectral_yield}
\end{equation}
where $R_{AC}$ is the crossing angle luminosity reduction factor (Eq.~\ref{eq:angular_crossing_factor}), which adjusts the head-on spectrum flux for the crossing angle of the interaction because \textsc{ICARUS} only simulates head-on interactions, $f$ is the repetition rate of interactions of the source, included because the \textsc{ICARUS} simulation only simulates a single electron bunch--laser pulse interaction and $\mathcal{F}_{\mathrm{\textsc{ICARUS}}}$ is the spectral yield (collimated flux) of a single electron bunch--laser pulse interaction. -- the area under the \textsc{ICARUS} spectrum. The spectral yield can be applied more generally to calculate the collimated flux from any spectrum code. 

The \textsc{ICARUS} simulation evaluates (Eq.~\ref{eq:ICARUS_equation}) at a series of scattered photon energy points to determine the number of photons produced at scattered photon energy intervals thereby building up a spectrum. For example, in Fig~\ref{fig:example_ICARUS_caseA} the spectrum is produced for a scattered photon energy range of $\Delta E_{\gamma} = 0.6$~\si{\mega\electronvolt}, where the model (Eq.~\ref{eq:ICARUS_equation}) is evaluated for 100 scattered photon energy points (a calculation every 6~\si{\kilo\electronvolt}). Simulation time increases linearly with the number of scattered photon energy points used in generating the \textsc{ICARUS} spectrum.

The maximum scattered photon energy to sample for the spectrum calculation is calculated using (Eq.~\ref{eq:scattered_photon_energy}) with $E_{e}+3\sigma_{e}$ as the electron bunch energy because of the Gaussian distribution of electron energies with non-negligible energy spread. However, this neglects the laser pulse spectral bandwidth because the scattered photon energy (Eq.~\ref{eq:scattered_photon_energy}) has a squared dependence on the electron bunch energy ($E_{\gamma}\propto E_{e}^{2}$) whereas there is a linear dependence on the incident photon energy ($E_{\gamma}\propto E_{L}$). Consequently, the scattered photon energy varies less with incident photon energy than electron energy and therefore it is a reasonable to neglect the spectral bandwidth of the laser pulse unless $\Delta E_{L}/E_{L} \gg \Delta E_{e}/E_{e}$. The minimum sampled scattered photon energy for a Gaussian electron bunch and laser pulse interaction is approximated arbitrarily as
\begin{equation}
E_{\gamma,\mathrm{min}} \approx E_{\gamma}\left[1-3\left(\frac{\Delta E_{\gamma}}{E_{\gamma}}\right)_{\mathrm{FWHM}}\right],
\label{eq:ICARUS_minimum_energy}
\end{equation}
where the Compton edge energy reduced is reduced by three full-width half-maximum bandwidths (Eq.~\ref{eq:FWHM_bandwidth}) of the ICS source being simulated. For example, the FWHM bandwidth of the case A spectrum in Fig.~\ref{fig:example_ICARUS_caseA} is 0.0118, so the minimum energy simulated in the spectrum is 16.9~\si{\mega\electronvolt}, with a 17.5~\si{\mega\electronvolt} Compton edge energy. Alternatively, a user specified maximum and minimum scattered photon energy can be provided.

For the complex integration involved in evaluating the central equation of \textsc{ICARUS} (Eq.~\ref{eq:ICARUS_equation}), quasi-Monte Carlo (QMC) integration methods are used alongside parallel-processing, where each node calculates an individual scattered photon energy point in the simulation. The oscillatory behaviour or `roughness' -- quickly varying amplitude of the spectral density -- is present in the shape of the \textsc{ICARUS} spectra because of the highly oscillatory integrals involved in the electron bunch energy spread term of (Eq.~\ref{eq:ICARUS_equation}) with small electron bunch energy spread
\begin{equation}
\frac{dN_{\gamma}}{dE_{\gamma}} &\propto \exp\left[\frac{\left(\gamma-\gamma_{0}\right)^{2}}{2\sigma_{\gamma}^{2}}\right],
\label{eq:ICARUS_electron_energy_spread_term}
\end{equation}
which causes errors in the quasi-Monte Carlo integration. The `roughness' or bumps seen in the \textsc{ICARUS} spectrum in Fig.~\ref{fig:example_ICARUS_caseA} appear unphysical and, upon further investigation, seem related to the setting of the quasi-Monte Carlo integration routine used in the simulation. 

\section{Benchmarking Cases for Characterisation and Optimisation}
\label{sec:benchmarking_cases_characterisation_optimisation}

Three ICS source benchmarking cases are specified which are designed to be characteristic of three accelerator types that are considered as drivers of ICS sources. The three test case ICS sources use the `re-circulated pulse' approach, using a Fabry-Perot optical cavity to re-circulate the laser pulse for interaction with a high repetition rate accelerator, which means low laser pulse energies are used, and that the flux of a single interaction is small. Further details can be found in Section~\ref{sec:lasers_fabry_perot}, and the interaction scheme matches that shown in Figure~\ref{fig:2_mirror_4_mirror}. These three cases include: a high energy ERL driven $\gamma$-ray ICS source (Case A) -- a precursor to the DIANA design in Chapter~\ref{DIANA_Inverse_Compton_Source_Design} -- a storage ring $\gamma$-ray ICS source based on the the MAX-III storage ring  \cite{owen2013nonequilibrium,sjostrom2009max} (Case B), and a low energy high repetition rate linac ICS source based on the Old Dominion University x-ray ICS source design \cite{krafft2016laser,deitrick2017inverse,deitrick2018high} (Case C). For each case a single set of laser parameters is used, as discussed in Section~\ref{sec:opt_char_laser}, based on the cERL Fabry-Perot optical re-circulation cavity \cite{akagi2016narrow}. Inclusion of a Fabry-Perot cavity in the case A ERL is as described for the DIANA ERL ICS design (Chapter~\ref{DIANA_Inverse_Compton_Source_Design}). A Fabry-Perot cavity can be incorporated within a straight -- as is the case for the DIANA ERL -- in a design like the MAX-III storage ring in case B, though proposal of a detailed optics scheme for ICS interaction are beyond the scope of this work. The laser system envisioned in the ODU ICS source \cite{krafft2016laser,deitrick2017inverse,deitrick2018high} could simply be replaced by the Fabry-Perot cavity as there is a reasonable distance ($L\sim0.5$~\si{\meter}) between the last final focus quadrupole and the interaction point where a Fabry-Perot optical cavity with a 3~\si{\meter} total path length could be installed \cite{deitrick2018high}.

\subsection{Electron Bunch Parameters}

The electron bunch parameters at the interaction point for each of the three accelerator cases are shown in Table~\ref{tab:char_opt_electron_bunch_parameters}; the parameters of the existing accelerators have not been modified.
\begin{table}[!h]
\centering
\caption{Electron bunch parameters at the interaction point. Parameters are given for three cases: a state-of-the-art 1~\si{\giga\electronvolt} ERL -- a preliminary design of the DIANA ERL (Chapter~\ref{DIANA_Inverse_Compton_Source_Design}) -- (Case A), the MAX-III storage ring operated at 700~\si{\mega\electronvolt} \cite{sjostrom2009max,hansson2011imaging,rosborg2012electron} (Case B) and the designed ODU ICS 25~\si{\mega\electronvolt} high repetition rate linac \cite{krafft2016laser,deitrick2017inverse,deitrick2018high} (Case C). Parameters have not been modified with respect to pre-existing accelerators.}
\vspace{3mm}
\resizebox{\columnwidth}{!}{
\begin{threeparttable}
\begin{tabular}{lcccc}
\hline\hline
Parameter & Case A (DIANA) & Case B (MAX-III) & Case C (ODU ICS) & Unit \\
\hline
Kinetic Energy, $E_{e}$ & 1000 & 700 & 25 & \si{\mega\electronvolt} \\
Repetition Rate, $f$ & 100 & 83.33 & 100 & \si{\mega\hertz} \\
Bunch Charge, $Q$ & 100 & 3000 & 10 & \si{\pico\coulomb} \\
Norm. Trans. Emittance, $\epsilon_{n,x}/\epsilon_{n,y}$ & 0.50/0.50 & 18.78/0.233 & 0.10/0.10 & \si{\milli\meter}-\si{\milli\radian} \\
Bunch Length, $\sigma_{e,z}$ & 1.00 & 27.9 & 0.38 & \si{\milli\meter} \\
Relative Energy Spread, $\Delta E_{e}/E_{e}$ & $10^{-4}$ & $6.07\times 10^{-4}$ & $3.00\times 10^{-4}$ & \\
\hline
Scattered Photon Energy~\tnote{*}, $E_{\gamma}$ & 17.55 & 8.65 & 0.0116 & \si{\mega\electronvolt} \\
\hline\hline
\end{tabular}
\begin{tablenotes}
\item[*]{Assuming head-on ($\phi=0$), backscattering ($\theta=0$) interaction with an Nd:YAG laser ($\lambda=1064$~\si{\nano\meter})}
\end{tablenotes}
\end{threeparttable}}
\label{tab:char_opt_electron_bunch_parameters}
\end{table}

The three electron bunch cases in Table~\ref{tab:char_opt_electron_bunch_parameters} have electron bunch energies chosen to reflect the typical electron bunch energy regimes of x-ray ($E_{e} =$10's~\si{\mega\electronvolt}) and $\gamma$-ray ($E_{e} =$100's~\si{\mega\electronvolt}) production typical of many proposed and operating ICS sources. Within the three electron bunch cases, each of the main accelerator driver options for ICS sources are represented: an ERL, a storage ring and a linac; therefore this should provide a fair overview of the applicability of the developed characterisation and optimisation methods to a wide range of ICS source designs. A large range in electron bunch energy is spanned (25--1000~\si{\mega\electronvolt}) as well as a large variation in bunch charge (10~\si{\pico\coulomb}--3~\si{\nano\coulomb}), tranverse normalised emittance (0.1--18.78~\si{\milli\meter}-\si{\milli\radian}) and bunch length (0.38--26.7~\si{\milli\meter}) which is adequate for evaluating the optimisation and characterisation methods and their efficacy in modelling a range of ICS sources. Each of these sources has a high $\sim100$~\si{\mega\hertz} repetition rate; therefore they are suitable for design of high flux ICS sources using the `re-circulated pulse' approach with Fabry-Perot cavities described in Section~\ref{sec:lasers_fabry_perot}. The characterisation and optimisation methods are designed toward the `re-circulated pulse' approach explained in Section~\ref{sec:lasers_fabry_perot}, because this scheme is the focus in this thesis but they can also be applied to any linear ($a_{0}\ll 1$) ICS source. 

Case A electron bunch parameters are typical of a world leading 1~\si{\giga\electronvolt} 3-turn energy recovery linac, and are based on a precursor to the conceptual DIANA ERL and ICS source that is presented in Chapter~\ref{DIANA_Inverse_Compton_Source_Design}. The DIANA ERL design is based upon several next generation \si{\giga\electronvolt} ERL designs, such as a recent ERL based EUV-FEL design \cite{akkermans2017compact}, the PERLE energy recovery linac \cite{angal2018perle} and the ER@CEBAF ERL project \cite{meot2016er,bogacz2016er}. The full justification for the case A electron bunch parameters is near-identical to the justification of the DIANA ICS source parameters, so the reader is directed to the explanation in Chapter~\ref{DIANA_Inverse_Compton_Source_Design}.      

The electron beam parameters of case B are based upon the electron bunch parameters of the MAX-III storage ring synchrotron light source \cite{sjostrom2009max,hansson2011imaging,rosborg2012electron}. Whilst the existing accelerator does not have an ICS interaction point, one could be accomodated by focusing optics and a Fabry-Perot cavity implemented in a straight section in the 36~\si{\metre} circumference. Operation of MAX-III as an ICS source has been suggested by Yu et al \cite{yu2009lattice,owen2013nonequilibrium}; many other synchrotron light sources have been used as an ICS source such as HI$\gamma$S at the Duke University storage ring \cite{weller2009research}, NewSUBARU \cite{utsunomiya2015gamma} and a similar design to a MAX-III ICS source has been proposed by Pan et al \cite{pan2019design}. The parameters for case B are based on Sj\"{o}strom et al \cite{sjostrom2009max} using emittance measurements by Hansson et al \cite{hansson2011imaging} because these parameters are most typical of existing storage ring based $\gamma$-ray ICS sources such as NewSUBARU \cite{utsunomiya2015gamma} and HI$\gamma$S \cite{weller2009research}.  MAX-III has numerous other configurations and operating modes \cite{sjostrom2009max}, such as non-equilibrium operation \cite{owen2012modular}, but these have not been evaluated. Note that the Case B 83.33~\si{\mega\hertz} repetition rate is lower than the 100~\si{\mega\hertz} repetition rate of case A and C \cite{sjostrom2009max,rosborg2012electron}. At a current of 250~\si{\milli\ampere}, with a 36~\si{\meter} circumference and repetition rate of 83.33~\si{\mega\hertz} \cite{sjostrom2009max,rosborg2012electron} there are a total of 10 bunches each with a 3~\si{\nano\coulomb} bunch charge .

Case C is based upon the design of the ODU compact linac \cite{krafft2016laser,deitrick2017inverse,deitrick2018high}, which is designed for use as a compact x-ray ICS source because of the lower energy electron beam ($E_{e} = 25$~\si{\mega\electronvolt}). The ODU ICS source is a high repetition rate ($f = 100$~\si{\mega\hertz}) linac, and is therefore comparable to the ERL and storage ring ICS approaches included here and could utilise a high average power Fabry-Perot cavity for the production of x-rays. The design study for this linac is completed with start-to-end simulations and therefore beam parameters are comprehensively presented \cite{deitrick2017inverse,deitrick2018high}, which are used in this case. Small emittance and energy spread parameters of the ODU linac are attractive for development of a narrowband ICS source, hence the ODU ICS souce design is suitable for inclusion within this study.  

\subsection{Laser Pulse Parameters}

The laser parameters for the test case ICS sources, based on the electron bunch parameters in Table~\ref{tab:char_opt_electron_bunch_parameters}, are kept constant with the exception of an adjusted repetition rate for the MAX-III case B parameters where the repetition rate is reduced to 83.33~\si{\mega\hertz}. Constant laser pulse parameters for each case, shown in Table~\ref{tab:char_opt_laser_pulse_parameters}, mean the effect of the electron bunch on the ICS spectrum and optimisation is more readily interpreted. The laser parameters are based upon an Nd:YAG laser ($\lambda = 1064$~\si{\nano\meter}) re-circulated in a 4-mirror Fabry-Perot optical cavity based on the cERL ICS demonstration at KEK \cite{akagi2016narrow}. The Nd:YAG laser is selected for its reasonable incident photon energy ($E_{L}=1.17$~\si{\electronvolt}), which enables scattering of high energy x-rays and $\gamma$-rays, as well as the commercially available narrow spectral bandwidth of $\Delta E_{L}/E_{L} = 4.70\times 10^{-4}$  ($\Delta\lambda = 0.5$~\si{\nano\meter}) \cite{thorlabs2021ndyag200}), which is ideal for a narrowband ICS source.

\begin{table}[!h]
\centering
\caption{Laser pulse parameters at the interaction point. Each accelerator electron bunch case in Table~\ref{tab:char_opt_electron_bunch_parameters} is assumed to interact with identical laser pulse parameters at the IP.}
\vspace{3mm}
\begin{threeparttable}
\begin{tabular}{lcc}
\hline\hline
Parameter & Quantity & Unit \\
\hline
Wavelength, $\lambda_\textrm{laser}$ & 1064 & nm\\
Photon energy, $E_\textrm{laser}$ & 1.17 & eV\\
Pulse energy  & 0.1 & \si{\milli\joule}\\
Number of photons, $N_{\textrm{laser}}$ & $5.34\times 10^{14}$\\ 
Repetition rate, $f$ & 100 (83.33)\tnote{*} & MHz\\
Spot size at the IP, $\sigma_{L}$ & 30 & \si{\micro\meter}\\
Crossing angle, $\phi$ & 5 & deg \\
Pulse length  & 10 & ps\\
Spectral Bandwidth, $\Delta E_{L}/E_{L}$ & 4.70$\times 10^{-4}$ &   \\
 % 0.5nm rms error on 1064nm
\hline\hline
\end{tabular}
\begin{tablenotes}
\item[*]{Adjusted to compensate for the lower case B (MAX-III \cite{sjostrom2009max,rosborg2012electron}) repetition rate.}
\end{tablenotes}
\end{threeparttable}
\label{tab:char_opt_laser_pulse_parameters}
\end{table}

The optical cavity envisioned to deliver these laser parameters has an average stored power of 10~\si{\kilo\watt} (Case B: 8.33~\si{\kilo\watt}), well below demonstrations at MuCLS \cite{eggl2016munich} and the state-of-the-art 670~\si{\kilo\watt} stored average power optical cavity \cite{carstens2014megawatt}. A reduced average stored power is considered here because this is the current highest average stored power ICS source demonstration on an ERL, from the cERL ICS souce \cite{akagi2016narrow}. Limitations are also imposed upon the stored power of a Fabry-Perot optical cavity due to mirror heating \cite{chaikovska2016high} and the proximity of strong magnetic fields \cite{gunther2019device}, which causes thermoelatic deformation of the cavity optical mirrors and a loss of stability. At a repetition rate of 100~\si{\mega\hertz} (83.33~\si{\mega\hertz}), the cavity path length is 3~\si{\meter} (3.6~\si{\meter})  which is tolerable for both misalignment errors \cite{zomer2009polarization} and mirror heating considerations with a single stored 0.1~\si{\milli\joule} laser pulse.  

\section{Benchmarking of the Characterisation Methods}
\label{sec:benchmarking_of_the_characterisation_methods}

\subsection{ICARUS Spectrum Code}
\label{sec:ICARUS_benchmarking}

In addition to \textsc{ICARUS}, several other codes are available that can calculate spectra of ICS sources, using either semi-analytical or Monte Carlo approaches. Two codes have been considered for benchmarking the \textsc{ICARUS} code: \textsc{CAIN} \cite{chen1995cain}, a Monte Carlo electromagnetic interactions code, and \textsc{ICCS3D} a semi-analytical inverse Compton scattering code. \textsc{CAIN} simulation is currently viewed as the `standard' method of simulating ICS source spectra; however, the \textsc{ICCS3D} code has demonstrated advantages in the simulation of re-circulated ICS sources with collimation, as discussed in this section. 

% CAIN - what is it + how does it work + what effects
The \textsc{CAIN} code is a Monte Carlo  code designed to simulate all electromagnetic interactions but subroutines can limit the simulated interactions to purely inverse Compton scattering interactions. A Gaussian or uniform laser distribution is generated -- specified using a superposition of plane waves -- and interacted with an electron bunch generated from a Gaussian or uniform distribution or a distribution of uniformly weighted macroparticles. A Monte Carlo event generator, as specified in the \textsc{CAIN} manual \cite{yokoya2003users}, determines whether a photon is produced and then calculates a scattered photon energy and polar and azimuthal scattering angles that correspond to the scattered photon. \textsc{CAIN} is capable of simulating non-linear ICS interactions, unlike the \textsc{ICARUS} code in Section~\ref{sec:development_of_the_ICARUS_spectrum_code}, but \textsc{CAIN} is also limited to head-on ($\phi=0$) interactions. The emittance and divergence effects of the electron bunch are accounted for by \textsc{CAIN}, but neglected in the laser pulse. Collimation is not directly implemented within the \textsc{CAIN} code but can be implemented via post processing of the produced spectra. 

% NEED MORE -HOW DOES IT WORK? WHAT EFFECTS?
The \textsc{ICCS3D} semi-analytical spectrum code, a generalization of the \textsc{ICCS} code \cite{krafft2016laser,ranjan2018simulation}, computes scattered radiation from laser pulse--electron bunch interactions within the linear Compton regime ($a_{0}\ll 1$) and accounts for electron recoil. In \textsc{ICCS3D}, a 3D laser pulse model is used as described in Terzi\'c et al \cite{terzic2019improving}; instead of all electrons experiencing the same laser field strength $a_{0}$, as they do for a 1D plane wave, their effective laser field strength is dependent on the electron's distance from the centre of the laser spot. \textsc{ICARUS} models the interaction as a collision between a laser pulse and electron bunch whereas \textsc{ICCS3D} models the interaction as an electron radiating within the electromagnetic field of a laser pulse; these are two equivalent models with differing mathematical construction. To calculate anticipated spectral output, in addition to the laser parameters, \textsc{ICCS3D} can use either the parameters of the electron beam or an arbitrary electron distribution -- which could have been obtained by tracking through the electron accelerator -- allowing start-to-end simulation of ICS sources. The emission spectrum of a single electron is calculated using a 3D laser field model and the Klein-Nishina cross section, and the bunch spectrum is formed by summing over the emission of each individual electron. Collimation is implemented within \textsc{ICCS3D} by limiting the possible scattered photon scattering angles through which the emission is calculated. \textsc{ICCS3D} can calculate the spectrum of an ICS interaction in an angular crossing case, with square or circular collimation and accounts for the spectral bandwidth of the laser pulse, the electron bunch energy spread as well as the emittance and divergence of the electron bunch. Like the \textsc{ICARUS} code, \textsc{ICCS3D} can produce spectra for arbitrary polarisation of the incident photons.   

A semi-analytical ICS spectrum code is advantageous to Monte Carlo based techniques such as the \textsc{CAIN} spectrum code as collimation effects aren't taken into account directly within the simulation and are instead required as part of post-simulation analysis. There is no built in method of imposing a collimator within \textsc{CAIN} \cite{chen1995cain}. Because \textsc{CAIN} treats the incident laser pulse as a superposition of plane waves, the effect of the spectral bandwidth of the laser pulse isn't properly accounted for. Incorporating the effect of the incident laser pulse spectral bandwidth is necessary for the ICS interaction because it can be the main contributor to the bandwidth of the resulting spectrum, as is frequently the case in the `single shot' ICS source approach (see Section~\ref{sec:lasers_fabry_perot}). Inherently, in Monte Carlo simulation rare events in nature will be as rare in the simulation, therefore statistics in situations where low scattered photon counts are expected are poor; for example in the tails of the distribution, at very narrow apertures \cite{ranjan2018simulation} and in re-circulated ICS sources where low flux interactions are conducted at high repetition rate. However, \textsc{CAIN} can model non-linear ($a_{0} \sim 1$) ICS interactions unlike \textsc{ICCS3D} and \textsc{ICARUS}, which are limited to the linear regime.  

The \textsc{ICCS3D} code \cite{krafft2016laser,ranjan2018simulation} was selected to benchmark the \textsc{ICARUS} spectrum code as \textsc{ICCS3D} also aims to predict spectra of ICS sources in the linear regime with a similar semi-analytical approach. \textsc{ICCS3D} provides a good benchmarking standard because  \textsc{ICCS3D} has also previously been benchmarked against the 1D model by Sun et al \cite{sun2009energy} (see Krafft et al \cite{krafft2016laser} Fig.~2--4) and the \textsc{CAIN} Monte Carlo code (see Ranjan et al\cite{ranjan2018simulation} Fig.~7). Consequently, benchmarking \textsc{ICARUS} against \textsc{ICCS3D} demonstrates that \textsc{ICARUS} would perform reasonably against \textsc{CAIN}. Through benchmarking of \textsc{ICARUS} against \textsc{ICCS3D}, \textsc{ICCS3D} has also been improved via introduction of a new integration method to handle laser pulse durations on the order of 10's~\si{\pico\second}.

\textsc{ICARUS} spectra have been produced for the three configurations of ICS source (case A, B and C) outlined in Section~\ref{sec:benchmarking_cases_characterisation_optimisation}, as shown in Figure~\ref{fig:ICARUS_optimised_benchmarking}. The head-on ($\phi=0$) spectra in Fig.~\ref{fig:ICARUS_optimised_benchmarking} are produced using the optimised electron bunch $\beta$-functions and collimation parameters from a 0.5\% \textit{rms} bandwidth (2\% \textit{rms} bandwidth for case B) simplex optimisation (detailed in Section~\ref{sec:NRB_optimisation}), as shown in Table~\ref{tab:single_point_optimisations}, because collimation and $\beta$-functions at the IP are not defined in Table~\ref{tab:char_opt_electron_bunch_parameters}. Spectra have also been produced as shown in Fig.~\ref{fig:ICARUS_optimised_benchmarking}, courtesy of B. Terzi\'{c}, using \textsc{ICCS3D} with identical parameters for cases A, B and C.       

\begin{figure}[!h]
\centering
\includegraphics[width=\textwidth]{Figures/Optimisation_and_Characterisation_of_Inverse_Compton_Scattering_Sources/ICARUS_ICCS3D_cases_comparision.pdf}
\caption{Comparison of ICS head-on ($\phi=0$) single electron bunch--laser pulse interaction spectra using circular collimation for each case in Table~\ref{tab:char_opt_electron_bunch_parameters}, produced by the semi-analytical codes \textsc{ICARUS} (red) and \textsc{ICCS3D} (blue) for 0.5\% \textit{rms} (2\% case B) bandwidth, with configuration optimised by the single point simplex elliptical beam optimisation  (see Section~\ref{sec:NRB_optimisation}). All \textsc{ICARUS} spectra are produced using 100 points across the energy range. Top Left: Case A. Top Right: Case B. Bottom Left: Case C.}
\label{fig:ICARUS_optimised_benchmarking}
\end{figure}

The \textsc{ICARUS} and \textsc{ICCS3D} spectra for cases A, B and C in Fig.~\ref{fig:ICARUS_optimised_benchmarking} show good agreement. The spectra in each case produced via \textsc{ICARUS} and \textsc{ICCS3D} are near-identical in shape, with identical peak spectral density and Compton edge energies (Eq.~\ref{eq:compton_edge_energy}) -- the scattered photon energy at maximum (or peak) spectral density -- in both of the spectra for each case. Consequently, the spectral yield (collimated flux) is in good agreement between the two codes -- an agreement in collimated flux of $< 2\%$ is noted for each of the benchmarking cases.

% Explain features of spectra
The spectra have a high energy tail resulting from the energy spread of the electron bunch and laser pulse spectral bandwidth at low spectral densities. Maximum spectral density is the location of the Compton edge, with scattered photon energy given by (Eq.~\ref{eq:compton_edge_energy}), which corresponds to the centroid energy of the electron bunch and laser pulse and occurs in the back-scattered direction ($\theta=0$). The collimator truncates the spectrum, so we do not see the full ICS source spectrum as shown in Fig.~\ref{fig:cross_section_scattered_photon_energy}. Emittance of the electron bunch and collimation result in the shape of the low energy tail of the spectrum. The `roughness' of the spectrum is related to numerical integration errors, as explained in Section~\ref{sec:development_of_the_ICARUS_spectrum_code}.

\subsection{Analytical Collimated Flux}
\label{sec:analytical_collimated_flux_benchmarking}

The analytical collimated flux (Eq.~\ref{eq:collimated_flux}) derived in Section~\ref{sec:analytical_collimated_flux} has been compared against numerous methods including the analytical formulation by Curatolo et al \cite{curatolo2017analytical} as well as the collimated flux (spectral yield) calculated from the \textsc{ICARUS} and \textsc{ICCS3D} \cite{krafft2016laser,ranjan2018simulation} spectrum codes. The collimated flux is calculated by Curatolo et al, in conventional units as 
\begin{equation}
\mathcal{F}_{\mathrm{col}} = 6.25\times 10^{8}\frac{E_{\mathrm{pulse}}\left(\mathrm{\si{\joule}}\right)Q\left(\si{\pico\coulomb}\right)f\left(\si{\hertz}\right)}{E_{L}\left(\si{\electronvolt}\right)\left[\sigma_{e}^{2}\left(\si{\micro\meter}\right)+\sigma_{L}^{2}\left(\si{\micro\meter}\right)\right]}\times\frac{\left(1+\sqrt[3]{X}\Psi^{2}/3\right)\Psi^{2}}{\left[1+\left(1+X/2\right)\Psi^{2}\right]\left(1+\Psi^{2}\right)},
\label{eq:curatolo_collimated_flux}
\end{equation}
where all symbols are consistent with those in (Eq.~\ref{eq:collimated_flux}), $\Psi-\gamma\theta_{\mathrm{col}}$ is the acceptance angle of the collimator and $\sigma_{e} = \sqrt{\sigma_{x}^{2}+\sigma_{y}^{2}}$ is the \texit{rms} spot size of the (round) electron bunch at the IP, with $\sigma_{x/y}$ the \textit{rms} spot size in each plane.

The collimated flux calculation by Curatolo et al \cite{curatolo2017analytical} (Eq.~\ref{eq:curatolo_collimated_flux}) uses the Berestetskii, Pitaevskii and Lifshitz \cite{berestetskii1982quantum} differential cross section (Eq.~\ref{eq:differential_cross_section_Y_invariant}) like the derivation presented here in Section~\ref{sec:analytical_collimated_flux} however, an explicit derivation is not shown or referenced in that paper. Upon inspection of (Eq.~\ref{eq:curatolo_collimated_flux}), it is evident that this is only valid for a head-on ($\phi=0$) interaction and doesn't account for the  hourglass effect \cite{furman1991hourglass,miyahara2008luminosity} (outlined in Section~\ref{sec:geometric_luminosity_reduction}). Therefore, to align the Curatolo et al calculations (Eq.~\ref{eq:curatolo_collimated_flux}) with (Eq.~\ref{eq:collimated_flux}), the angular crossing and hourglass effect luminosity reduction factor $R_{ACHG}$ \cite{miyahara2008luminosity} (Eq.~\ref{eq:miyahara_combined_reduction}) must be introduced.

The collimated flux calculation methods are compared using the benchmarking cases in Section~\ref{sec:benchmarking_cases_characterisation_optimisation} with the electron bunch parameters in Table~\ref{tab:char_opt_electron_bunch_parameters} and laser pulse parameters in Table~\ref{tab:char_opt_laser_pulse_parameters}. The calculations have been conducted using the simplex single point bandwidth optimisation, as outlined in Section~\ref{sec:NRB_optimisation}, for a 0.5\% \textit{rms} bandwidth (2\% \textit{rms} bandwidth case B) which yields the values for the $\beta$-functions at the IP and collimation angle shown later in this chapter in Table~\ref{tab:single_point_optimisations}. The results of the collimated flux calculations, by each method, for each of the benchmarking cases A, B and C are shown in Table~\ref{tab:collimated_flux_calculations}

\begin{table}[!h]
\centering
\caption{Calculations of collimated flux in a 0.5\% \textit{rms} bandwidth (Case B 2\% \textit{rms} bandwidth) for each of the benchmarking cases in Table~\ref{tab:char_opt_electron_bunch_parameters}, using elliptical beam simplex optimised interaction point parameters as shown in Table~\ref{tab:single_point_optimisations}, via a variety of methods. Laser parameters shown in Table~\ref{tab:char_opt_laser_pulse_parameters} are kept constant for each benchmarking case.}
\vspace{3mm}
\begin{threeparttable}
\begin{tabular}{lccc}
\hline\hline
 & \multicolumn{3}{c}{Collimated Flux (ph/\si{\second})} \\
\hline
Method & Case A & Case B & Case C \\
\hline 
Curatolo et al~\tnote{*} (Eq.~\ref{eq:curatolo_collimated_flux}) \cite{curatolo2017analytical} & $6.03 \times 10^{8}$ & $0.60\times 10^{10}$ & $0.74\times 10^{8}$ \\
Analytical (Eq.~\ref{eq:collimated_flux}) & $8.86 \times 10^{8}$ & $1.04 \times 10^{10}$ & $1.17\times 10^{8}$ \\
\textsc{ICARUS}~\tnote{$\dagger$} (Eq.~\ref{eq:spectral_yield}) & $8.76\times 10^{8}$ & $1.04\times 10^{10}$ & $1.14\times 10^{8}$ \\
\textsc{ICCS3D}~\tnote{$\dagger$} (Eq.~\ref{eq:spectral_yield}) \cite{krafft2016laser,ranjan2018simulation} & $8.75\times 10^{8}$ & $1.02\times 10^{10}$ & $1.16\times 10^{8}$  \\
\hline\hline
\end{tabular}
\begin{tablenotes}
\item[*]{The Curatolo et al collimated flux (Eq.~\ref{eq:curatolo_collimated_flux}) has been multiplied by the combined angular crossing and hourglass effect luminosity reduction factor (Eq.~\ref{eq:miyahara_combined_reduction}) for comparison with other collimated flux calculations.}
\item[$\dagger$]{The \textsc{ICARUS} and \textsc{ICCS3D} spectra used in the calculation of the collimated flux (spectral yield) are shown in Fig.~\ref{fig:ICARUS_optimised_benchmarking}.}
\end{tablenotes}
\end{threeparttable}
\label{tab:collimated_flux_calculations}
\end{table}

The collimated flux calculations in Table~\ref{tab:collimated_flux_calculations} show good agreement between the spectrum code calculations, as \textsc{ICARUS} and \textsc{ICCS3D} agree to within $< 2\%$ in all cases. Good agreement was shown qualitatively between these two codes in Fig.~\ref{fig:ICARUS_optimised_benchmarking}, therefore good quantitative agreement was expected. Between the \textsc{ICARUS} code and the analytical collimated flux calculation (Eq.~\ref{eq:collimated_flux}), a maximum discrepancy of 1.14\% is observed. Percentage scale discrepancies between these two calculations can be expected because the analytical calculation neglects the energy spread of the electron bunch and the spectral bandwidth of the laser pulse and the \textsc{ICARUS} code is not exact because of oscillatory integration errors, as discussed in Section~\ref{sec:ICARUS_benchmarking}. 

However, the collimated flux calculation by Curatolo et al \cite{curatolo2017analytical} (Eq.~\ref{eq:curatolo_collimated_flux}) shows large differences from both the analytical collimated flux derived in Section~\ref{sec:analytical_collimated_flux} and the spectrum codes. Table~\ref{tab:collimated_flux_calculations} shows the Curatolo et al collimated flux (Eq.~\ref{eq:curatolo_collimated_flux}) is consistently reduced with reference to the analytical (Eq.~\ref{eq:collimated_flux}) and spectrum code collimated flux, which is also observed in Fig.~\ref{fig:curatolo_collimated_flux_comparison}. For example, in Table~\ref{tab:collimated_flux_calculations}, a factor $\sim 1.75$ reduction in collimated flux is observed between (Eq.~\ref{eq:collimated_flux}) and (Eq.~\ref{eq:curatolo_collimated_flux}) for case B. A comparison of the collimated  flux derived here (Eq.~\ref{eq:collimated_flux}) and that by Curatolo et al \cite{curatolo2017analytical} (Eq.~\ref{eq:curatolo_collimated_flux}) as a function of scattering angle is shown in Fig.~\ref{fig:curatolo_collimated_flux_comparison}. The discrepancy must originate in the angular term of (Eq.~\ref{eq:curatolo_collimated_flux})
\begin{equation*}
\mathcal{F}_{\mathrm{col}}\propto\frac{\left(1+\sqrt[3]{X}\Psi^{2}/3\right)\Psi^{2}}{\left[1+\left(1+X/2\right)\Psi^{2}\right]\left(1+\Psi^{2}\right)}    
\end{equation*}
because the head-on luminosity is equivalent in both equations, the geometric luminosity reduction is applied identically and the cross section variation of Case A (the maximum electron bunch energy case, $E_{e}=1$~\si{\giga\electronvolt}) is small -- a 1.2\% reduction in the ICS cross section (Eq.~\ref{eq:compton_cross_section_classical_limit}). The cross section variation with electron energy is illustrated in Fig.~\ref{fig:cross_section_electron_energy}. However, the origin of the discrepancy in the angular term of the Curatolo et al \cite{curatolo2017analytical} calculation (Eq.~\ref{eq:curatolo_collimated_flux}) remains unknown because no adequate derivation is shown within the literature.   

To further investigate the angular dependency of the collimated flux calculations the collimated flux, via the two analytical methods (Eqs.~\ref{eq:collimated_flux}, \ref{eq:curatolo_collimated_flux}) and the \textsc{ICARUS} spectrum code, has been calculated as a function of the acceptance angle in Fig.~\ref{fig:curatolo_collimated_flux_comparison}. The benchmarking cases from Table~\ref{tab:char_opt_electron_bunch_parameters} are used, where the interaction $\beta$-functions are those from the 0.5\% \textit{rms} bandwidth (case B 2\% \textit{rms} bandwidth) elliptical beam simplex optimisation, and the collimation angle is varied. Laser parameters remain constant, as given in Table~\ref{tab:char_opt_laser_pulse_parameters}. 
\begin{figure}[!h]
\centering
\includegraphics[width=\textwidth]{Figures/Optimisation_and_Characterisation_of_Inverse_Compton_Scattering_Sources/Fcol_PSI_Cases_Curatolo_Analytical_ICARUS.pdf}
\caption{Comparison of the derived analytical collimated flux (Eq.~\ref{eq:collimated_flux}) (blue) with the Curatolo et al collimated flux calculation \cite{curatolo2017analytical}  (Eq.~\ref{eq:curatolo_collimated_flux}) (red) and the results of the \textsc{ICARUS} spectrum code (black) within an acceptance angle $0 \geq \Psi \geq 1$ (a $1/\gamma$ cone). The benchmarking cases are presented in Table~\ref{tab:char_opt_electron_bunch_parameters}, optimised for a 0.5\% \textit{rms} bandwidth (Case B 2\% \textit{rms} bandwidth) using the elliptical beam simplex optimisation with parameters shown in Table~\ref{tab:single_point_optimisations}. The \textsc{ICARUS} spectrum code data has been adjusted for an angular crossing (Eq.~\ref{eq:angular_crossing_factor}) and the Curatolo et al calculation (Eq.~\ref{eq:curatolo_collimated_flux}) has been adjusted for the hourglass effect and angular crossing (Eq.~\ref{eq:miyahara_combined_reduction}). Top Left: Case A. Top Right: Case B. Bottom Left: Case C.}
\label{fig:curatolo_collimated_flux_comparison}
\end{figure}

In Fig.~\ref{fig:curatolo_collimated_flux_comparison}, there is a strong agreement between the analytical collimated flux and that derived from the \textsc{ICARUS} spectrum code, therefore (Eq.~\ref{eq:scattered_photon_energy}) can efficiently predict the collimated flux across the whole range of scattering angles. However, at large acceptance angles in case A and C the \textsc{ICARUS} values differ from the analytical calculation. The discrepancy is due to the settings of the integration routine which have been setup for quick but less precise simulation and can result in a small integration error, as previously shown in Fig.~\ref{fig:ICARUS_sim_qmc_points}. The scattered photon energy range increases with collimation angle because of the energy--angle correspondence  in the scattered photon energy (Eq.~\ref{eq:scattered_photon_energy}) which, with a constant 100 simulation points, dilutes the spectrum as the step energy between simulation points is greater. The collimated flux calculation by Curatolo et al \cite{curatolo2017analytical} (Eq.~\ref{eq:curatolo_collimated_flux}) disagrees with (Eq.~\ref{eq:collimated_flux}) and the \textsc{ICARUS} scattering code beyond $\psi>0.1$ in each case in Fig.~\ref{fig:curatolo_collimated_flux_comparison}. The discrepancy in the Curatolo et al calculation increases as a function of acceptance angle, which suggests that a small angle approximation could occur within the derivation of (Eq.~\ref{eq:curatolo_collimated_flux}), however it is unverifiable because (Eq.~\ref{eq:curatolo_collimated_flux}) is not derived in the literature.  

The analytical calculation (Eq.~\ref{eq:collimated_flux}) is sufficient to calculate the yield of \textsc{ICARUS} spectrum code within energy spread and spectral bandwidth tolerances and consequently is a valid alternative to spectrum code collimated flux calculations. The analytical collimated flux calculation is advantageous in terms of simulation time, as the collimated flux is calculated analytically on a sub-second timescale whereas the spectrum codes require up to 10 hours to run. Consequently, the analytical collimated flux is a useful calculation method for computing the many collimated flux values required in the optimisations detailed in the rest of this chapter.   

\section{Motivation for Narrowband Optimisation of ICS Sources}
\label{sec:motivation_optimisation}

Currently, many ICS sources are designed to maximise uncollimated flux via matching the transverse \textit{rms} spot size of the laser pulse to that of the electron bunch ($\sigma_{L}\approx\sigma_{\mathrm{electron}}$) \cite{akagi2016narrow,deitrick2018high,jacquet2015radiation,drebot2019brixs,pan2019design,dupraz2020thomx}, termed here transverse profile matching (TPM). However, TPM is deficient because ICS sources have two aims: to produce a large quantity of photons at the sample, where the scattered photon beam is collimated, and to minimise the energy spread (bandwidth) of the photons. The former, a high flux of photons, is desirable because a high flux decreases data acquisition times and improves the signal-to-noise ratio of measurements; the latter is advantageous due to the ability to target certain processes with small energy bandwidths, for example a specific resonance in an isotope like $^{235}\mathrm{U}$ for a nuclear resonance fluorescence investigation \cite{hayakawa2010nondestructive}. Users typically require a compromise between bandwidth and flux for precise investigations of specific energy dependent phenomena whilst conducting measurements on a reasonable timescale and with adequate statistics. Therefore, within the following optimisations we aim to maximise the collimated flux (Eq.~\ref{eq:collimated_flux}) whilst minimising, or limiting to some user specified value, the \textit{rms} bandwidth (Eq.~\ref{eq:RMS_bandwidth}) of the ICS source. All optimisations are performed with the assumption of a circular collimator, as the focus of this study is on narrow bandwidth and studies by Hajima \cite{hajima2021bandwidth} demonstrate circular collimation is the optimum approach for narrow bandwidth; the analytical collimated flux (Eq.~\ref{eq:collimated_flux}) is also derived for circular collimation.

Firstly, within this half of the chapter, the choice of variables of the optimisations are justified and the types of optimisation (single bandwidth and tuning curve) are explained. Then two optimisation methods are developed and tested: a round beam optimisation (RB) (Section~\ref{sec:RB_optimisation}) and an elliptical beam simplex optimisation (EB simplex). Round beam optimisations simplify the electron bunch dynamics so the $\beta$-functions are identical in each plane ($x$ and $y$), whereas elliptical beam optimisations treat each transverse plane individually. These optimisation methods are then evaluated and compared in Section~\ref{sec:evaluation_of_optimisation_methods} using the previously defined benchmarking cases in Section~\ref{sec:benchmarking_cases_characterisation_optimisation}. The developed optimisations are applied throughout the ICS source designs in Chapters~\ref{CBETA_Inverse_Compton_Scattering_Source_Design} and \ref{DIANA_Inverse_Compton_Source_Design}.    

\subsection{Variable Selection}
\label{sec:variable_selection}

Inspection of the \textit{rms} bandwidth for ICS sources has shown that the emittance (Eq.~\ref{eq:emittance_term}) and collimation (Eq.~\ref{eq:collimation_term}) terms typically dominate the bandwidth of an ICS source. The dominant terms of the bandwidth are dependent upon the transverse $\beta$-functions at the IP $\beta^{*}_{x/y}$ in each plane, the emittance (emittance term) in each plane and the collimation angle $\theta_{\mathrm{col}}$ (collimation term) respectively. Similarly, the collimated flux (Eq.~\ref{eq:collimated_flux}) of an ICS source is dependent upon the $\beta$-functions at the IP (via the electron beam spot size) and the collimation angle. Varying the $\beta$-functions at the IP is a better variable choice than the emittance because the $\beta$-functions can be varied through adjustment of the electron beam final focus whereas the transverse emittance is dependent on the electron photo-injector and the collective effects experienced by the bunch throughout the accelerator. Collimation of the produced radiation is external to the electron bunch--laser pulse interaction, therefore collimation angle is easily adjusted via a selection of collimators, a variable aperture collimator or by adjusting the source-to-collimator distance and hence is a good optimisation variable. Therefore, the $\beta$-functions at the IP and collimation angle ($\beta^{*}_{x}$, $\beta^{*}_{y}$, $\theta_{\mathrm{col}}$) are selected as optimisation variables. These variables can be tuned for any accelerator type, hence the ICS source optimisations presented here are generalised to any accelerator driver of an ICS source, not just an ERL.

Transverse profile matching is insufficient for optimising an ICS source for narrow bandwidth because the collimation angle is left as the only free parameter. If the emittance term (Eq.~\ref{eq:emittance_term}) is dominant in the bandwidth when $\sigma_{\mathrm{electron}} \approx \sigma_{L}$ then the collimation angle can not further reduce the bandwidth because it can only vary the collimation term (Eq.~\ref{eq:collimation_term}). The flux of an ICS source may also be poorly optimised by TPM when the laser spot size is large, the $\sigma_{\mathrm{electron}} \approx \sigma_{L}$ under-focuses the electron beam and decreases the luminosity (Eq.~\ref{eq:headon_luminosity}).

Other parameters could be selected as optimisation variables such as the transverse spot sizes of the laser pulse at the IP, however this is more complex because Fabry-Perot optical cavity design is severely constrained, as discussed in Section~\ref{sec:lasers_fabry_perot}. Electron bunch longitudinal phase space could also be considered for optimisation of the collimated flux and bandwidth because the bandwidth (Eq.~\ref{eq:RMS_bandwidth}) is dependent on the electron bunch energy spread and the collimated flux is dependent on the electron bunch length -- via the angular crossing luminosity reduction (Eq.~\ref{eq:angular_crossing_factor}). However, the collimated flux advantage of the reducing the electron bunch length is small and longitudinal optimisation is also more complex because of the RF system -- requiring a more complete accelerator design. Transverse laser pulse and longitudinal electron bunch optimisations are suitable subjects for future work but are rejected here.  

\subsection{Single Bandwidth and Tuning Curve Optimisations}

Single bandwidth optimisations aim to maximise the collimated flux for a specific single bandwidth which can be supplied by the user of the optimisation. For example, single bandwidth optimisations could be used to determine the maximum flux of an ICS source that can be produced with a 1\% bandwidth of the scattered photon beam. Therefore, single bandwidth optimisations are useful when an experimental phenomena can only be resolved by using a certain maximum energy bandwidth (spread) of the scattered photons, such as determining single nuclear resonance fluorescence lines in uranium-235 \cite{hayakawa2010nondestructive}. The optimisation methods also return the values of the variables required to achieve the user specified bandwidth and therefore the configuration of the source required to achieve the best experimental parameters for the user is known.

Single bandwidth optimisations are achieved through using the bandwidth of the ICS source as a constraint on the optimisations. The bandwidth can be used to construct a constraint on the variables directly, as in the round beam optimisation detailed in Section~\ref{sec:RB_optimisation} or indirectly as a separate constraint as in the elliptical beam optimisation. To use the bandwidth as an indirect constraint the difference between the bandwidth achieved in the optimisation $\left(\Delta E_{\gamma}/E_{\gamma}\right)_{\mathrm{ach}}$ and the single bandwidth supplied by the user $\left(\Delta E_{\gamma}/E_{\gamma}\right)$ is zeroed
\begin{equation}
0 = \left(\frac{\Delta E_{\gamma}}{E_{\gamma}}\right)_{\mathrm{ach}} - \left(\frac{\Delta E_{\gamma}}{E_{\gamma}}\right)_{\mathrm{tar}}.
\label{eq:single_bandiwdth_constraint}
\end{equation}
This method and its implementation if further detailed in  in Section~\ref{sec:EB_optimisation}.

Tuning curve optimisations aim to show the maximum collimated flux that can be produced over a range in bandwidth values. For example, the maximum collimated flux that an ICS source could produce within a bandwidth range of $0 < \left(\Delta E_{\gamma}/E_{\gamma}\right) < 0.01$ i.e. from 0--1\% bandwidth. Therefore, tuning curves are advantageous in mapping the possible operational configurations of an ICS source, much like the peak brilliance--photon energy ($\mathcal{B}_{\mathrm{pk}}$--$E_{\gamma}$) tuning curves produced for synchrotron radiation sources (see Fig.~\ref{fig:light_source_tuning_curves}). The tuning curves are produced by multiple single bandwidth optimisations at intervals in a bandwidth range and produce a plot of the maximum collimated flux against bandwidth of an ICS source. Similarly, tuning curves of the variables can be produced that correspond to the collimated flux--bandwidth tuning curves since the configuration of the variables that relate to the maximum collimated flux are also returned in single bandwidth optimisations. Variable tuning curves show the trade-off required to design narrowband ICS sources. 

\section{Round Beam Optimisation}
\label{sec:RB_optimisation}

Here we develop a brute-force optimisation method to maximise flux within a selected \textit{rms} bandwidth for the simplified case of an electron beam with a round transverse profile where the normalised emittance and $\beta$-functions at the IP are assumed to be identical in both planes ($\epsilon_{nx} = \epsilon_{ny} = \epsilon_{n}$, $\beta_{x}^{*} = \beta_{y}^{*} = \beta^{*}$), named the round beam (RB) approximation. Consequently, bandwidth tuning is possible via two variables: selecting $\beta^{*}$ at the IP and by setting the collimation angle $\theta_{\mathrm{col}}$.  The round beam approximation simplifies the interaction dynamics from a more general model with three variables. RB optimisation shows simple optimisation can yield improvement in collimated flux, and is applicable in ICS sources with near-round electron bunch transverse profiles -- as in ERLs and linacs. A round electron beam may also be the optimum solution (maximal collimated flux, minimal bandwidth) because the transverse laser profile is also typically near-round in an optical cavity \cite{dupraz2020thomx} and therefore a round beam may provide the optimum overlap of the electron beam and laser pulse.

\subsection{Round Beam Optimisation Method}

For the RB optimisation, the \textit{rms} bandwidth is given by (Eq.~\ref{eq:RMS_bandwidth}) with the emittance term (Eq.~\ref{eq:emittance_term}) modified for the transversely round bunch case
\begin{equation}
\frac{\sigma_{\epsilon}}{E_{\epsilon}} = \frac{2\gamma\epsilon_{n}}{\left(1+X\right)\beta^{*}},
\label{eq:emittance_term_round_beam}    
\end{equation}
where $X$ is the recoil parameter (Eq.~\ref{eq:X_geometry}), $\epsilon_{n}$ is the transverse \textit{rms} normalised emittance of the electron bunch and $\beta^{*}$ is the $\beta$-function at the interaction point. As mentioned in Section~\ref{sec:variable_selection}, the collimation term Eq.~(\ref{eq:collimation_term}) and the emittance term Eq.~(\ref{eq:emittance_term_round_beam}) are dominant. Therefore, optimisation of bandwidth and collimated flux minimises the collimation and emittance terms whilst maximising collimated flux.

By using a larger $\beta^{*}$ and a small collimator aperture (collimation angle), the contribution of the collimation and emittance terms can be reduced so that they are negligible; thus the electron bunch (Eq.~\ref{eq:beam_energy_spread_term}) and laser pulse energy spread terms (Eq.~\ref{eq:laser_energy_spread_term}) dominate the bandwidth for accelerators with a sufficiently small emittance. Taking the limit of a small collimation angle ($\theta_{\mathrm{col}}\rightarrow 0$), and assuming the $\beta$-function at the IP can be made very large ($\beta^{*}\rightarrow 0$) this effectively places a lower limit on the bandwidth of an ICS source; it is limited by the energy spread of the electron beam $\Delta E_{e}/E_{e}$ and laser pulse spectral bandwidth $\Delta E_{\mathrm{laser}}/E_{\mathrm{laser}}$ as 
\begin{equation}
\left(\frac{\Delta E_{\gamma}}{E_{\gamma}}\right)_{\mathrm{min}} \approx \sqrt{\left[\left(\frac{2+X}{1+X}\right)\frac{\Delta E_{e}}{E_{e}}\right]^{2} + \left[\left(\frac{1}{1+X}\right)\frac{\Delta E_{L}}{E_{L}}\right]^{2}}.
\label{eq:bandwidth_limitation_minimum}
\end{equation}

Consequently, any bandwidth above the 
limit (Eq.~\ref{eq:bandwidth_limitation_minimum}) can theoretically be achieved by an ICS source by tuning of the collimation angle and $\beta$-function at the IP so that a desired bandwidth, $\Delta E_{\gamma}/E_{\gamma}$, is achieved. Since the collimation and emittance terms are typically dominant, all other terms can be excluded and the solutions are approximately bounded by
\begin{equation}
\frac{\Delta E_{\gamma}}{E_{\gamma}} > \sqrt{\left(\frac{ \sigma_{\theta}}{E_{\theta}}\right)^{2}+\left(\frac{\sigma_{\epsilon}}{E_{\epsilon}}\right)^{2}}.
\label{eq:bandwidth_limitation_maximum_approximation}
\end{equation}
The limit in (Eq.~\ref{eq:bandwidth_limitation_maximum_approximation}) can be re-cast in terms of $\beta^{*}$ through a rearrangement of (Eq.~\ref{eq:RMS_bandwidth}), with the emittance term in the transversely round bunch case (Eq.~\ref{eq:emittance_term_round_beam})
\begin{equation}
\beta^{*} \leq \frac{2\gamma\epsilon_{n}}{\left(1+X\right)\sqrt{\left(\frac{\Delta E_{\gamma}}{E_{\gamma}}\right)^{2}-\left[\left(\frac{\sigma_{\theta}}{E_{\theta}}\right)^{2}+\left(\frac{\sigma_{e}}{E_{e}}\right)^{2}+\left(\frac{\sigma_{L}}{E_{L}}\right)^{2}\right]}}.
\label{eq:beta_star_maximum limitation}
\end{equation}

The collimation angle is lower bounded as the collimation angle must be larger than zero ($\theta_{\mathrm{col}}>0$) -- or the collimator becomes an attenuator -- and is upper bounded through the assumption that the collimation term (Eq.~\ref{eq:collimation_term}) is dominant and the \textit{rms} bandwidth (Eq.~\ref{eq:RMS_bandwidth}) becomes
\begin{equation}
\left(\frac{\Delta E_{\gamma}}{E_{\gamma}}\right)_{\mathrm{rms}} \approx \left(\frac{\sigma_{\theta}}{E_{\theta}}\right),    
\label{eq:collimation_dominant}
\end{equation}
which, by expanding the collimation term (Eq.~\ref{eq:collimation_term}), is re-cast as an approximate collimation angle upper bound  
\begin{equation}
\theta_{\mathrm{col},\mathrm{max}} \approx \frac{1}{\gamma}\sqrt{\frac{2\sqrt{3}\left(\Delta E_{\gamma}/E_{\gamma}\right)_{\mathrm{rms}}\left[1+X\right]}{1-\sqrt{3}\left(\Delta E_{\gamma}/E_{\gamma}\right)_{\mathrm{rms}}}}
\label{eq:collimation_angle_upper_bound}    
\end{equation}
where all terms have previously been defined and $0<\theta_{\mathrm{col}}<\theta_{\mathrm{col},\mathrm{max}}$.
 
Bounding the bandwidth by these limits (Eq.~\ref{eq:bandwidth_limitation_minimum}) and  (Eq.~\ref{eq:bandwidth_limitation_maximum_approximation}) results in a range of $\beta^{*}$ and $\theta_{\mathrm{col}}$ that give a particular chosen bandwidth. The different $\beta^{*}$, $\theta_{\mathrm{col}}$ combinations each give a different collimated flux; the solution with the largest collimated flux is optimal. Calculation of the collimated flux (Eq.~\ref{eq:collimated_flux}) from every combination of $\beta^{*}$ and $\theta_{\mathrm{col}}$ within the upper (Eq.~\ref{eq:beta_star_maximum limitation}) and lower (Eq.~\ref{eq:bandwidth_limitation_minimum}) bounds is not practical. Instead, an array of collimation angles $\theta_{\mathrm{col}}$ from 0 to $1/\gamma$ is used ($0\leq\Psi\leq1$), stepped by $\Delta\theta_{\mathrm{col}}$, is used and the collimation angle and minimum $\beta^{*}$-function (for maximum flux) are calculated for each combination, with the maximum collimated flux solution returned 
\subsection{Implementation and Test Case}

Calculation of the collimated flux (Eq.~\ref{eq:collimated_flux}) from every combination of $\beta^{*}$ and $\theta_{\mathrm{col}}$ within the upper (Eq.~\ref{eq:beta_star_maximum limitation}) and lower (Eq.~\ref{eq:bandwidth_limitation_minimum}) bounds of the $\beta$-function and collimation angle (Eq.~\ref{eq:collimation_angle_upper_bound}) is not practical. Instead, an array of collimation angles $\theta_{\mathrm{col}}$ from 0 to the upper bound (Eq.~\ref{eq:collimation_angle_upper_bound}), stepped by $\Delta\theta_{\mathrm{col}}$, for 1000 steps is used and the collimation angle and minimum $\beta^{*}$-function (for maximum flux) are calculated for each combination, with the maximum collimated flux solution returned. The optimisation method is implemented as a script within \textsc{Mathematica} \cite{wolfram2021nmaximize}, that iterates over every point within this range and selects the solution with the maximum collimated flux. 

The method described above optimises the collimated flux within bandwidth limits to determine $\theta_{\mathrm{col}}$ and $\beta^{*}$. In addition, applying the method to a series of chosen bandwidths maps the optimum configurations of the ICS source, producing tuning curves of the collimated flux against bandwidth. For example, tuning curves in solution ($\mathcal{F}_{\mathrm{col}}$--$\left(\Delta E_{\gamma}/E_{\gamma}\right)_{\mathrm{rms}}$) and parameter ($\beta^{*}$--$\theta_{\mathrm{col}}$) space have been produced for benchmarking case A as shown in Fig.~\ref{fig:CaseA_RB_tuning_curve}, using electron bunch parameters in Table~\ref{tab:char_opt_electron_bunch_parameters} with laser parameters in Table~\ref{tab:char_opt_laser_pulse_parameters}. 
\begin{figure}[!h]
\centering
\includegraphics[width=\textwidth]{Figures/Optimisation_and_Characterisation_of_Inverse_Compton_Scattering_Sources/Case_A_RB_Tuning_Curves.pdf}
\caption{Case A (see Table~\ref{tab:char_opt_electron_bunch_parameters}) tuning curves in solution space (red) ($\mathcal{F}_{\mathrm{col}}$--$\left(\Delta E_{\gamma}/E_{\gamma}\right)_{\mathrm{rms}}$) and parameter space ($\beta^{*}$--$\theta_{\mathrm{col}}$) (blue), using the round beam optimisation method in the narrowband regime ($\DeltaE_{\gamma}/E_{\gamma}/E_{\gamma}$). Both the left and right plot are coupled together as the right plot is the parameter space ($\beta^{*}$--$\theta_{\mathrm{col}}$) of the $\mathcal{F}$--$\left(\Delta E_{\gamma}/E_{\gamma}\right)$ solution space. Left: Solution space plot showing the Pareto-optimal front. All solutions below the line (red, shaded) are possible. Right: Parameter space plot showing the variable front corresponding to the Pareto-optimal front. Wide-band solutions favour large collimation angle and small $\beta$-function at the IP (collimation dominated), narrow-band solutions favour small collimation angle and and large $\beta$-function at the IP (emittance dominated). All solutions above the line (blue, shaded) are possible. }
\label{fig:CaseA_RB_tuning_curve}
\end{figure}

Fig.~\ref{fig:CaseA_RB_tuning_curve} shows the collimated flux--\textit{rms} bandwidth tuning curve for the case A parameters within the \textit{rms} bandwidth range 0--1\%, the defined regime of narrowband operation. A small bandwidth limitation (Eq.~\ref{eq:bandwidth_limitation_minimum}) of $\sim$0.05\% exists due to the electron bunch energy spread and laser spectral bandwidth of the source. The collimated flux (Eq.~\ref{eq:collimated_flux}) increases linearly with increasing \textit{rms} bandwidth. Collimated fluxes below the line (i.e. less flux) can be produced at each bandwidth point in the range, however the tuning curve denotes the maximum collimated flux available. 

The case A $\beta^{*}$--$\theta_{\mathrm{col}}$ parameter space in Fig.~\ref{fig:CaseA_RB_tuning_curve} has a curved front, where the combinations of variables shown in the tuning curve correspond to the $\mathcal{F}_{\mathrm{col}}$--$\left(\Delta E_{\gamma}/E_{\gamma}\right)$ tuning curve. All solutions above the line in the $\beta^{*}$--$\theta_{\mathrm{col}}$ parameter space are possible though combinations of $\beta^{*}$ and $\theta_{\mathrm{col}}$ on the front shown are optimal. Combinations of larger $\beta^{*}$-functions and collimation angles (and vice-versa) may have the same \textit{rms} bandwidth but will have reduced collimated flux. The narrowest bandwidth solutions have higher $\beta$-function at the IP and smaller collimation angle whereas the wider-band solutions exist at smaller $\beta$-functions at the IP and larger collimation angles. 

In the small collimation angle and large $\beta$-function region of the tuning curve $(\theta_{\mathrm{col}}\lesssim 0.02$~\si{\milli\radian}) the emittance term (Eq.~\ref{eq:emittance_term_round_beam}) dominates unlike in the wider-band, larger collimation angle and smaller $\beta^{*}$-function region ($\theta_\mathrm{col} \gtrsim 0.02$~\si{\milli\radian}) where the collimation term dominates. A point exists in the tuning curve where emittance domination switches to collimation domination, which is not visible for case A but is visible in case C in Fig.~\ref{fig:case_C_optimisation_comparision}. 

\section{Elliptical Beam Simplex Optimisation}
\label{sec:EB_optimisation}

In an elliptical beam (EB) optimisation the transverse normalised emittances ($\epsilon_{nx}/\epsilon_{ny}$) and the $\beta$-functions at the IP ($\beta_{x}^{*}/\beta_{y}^{*}$) in each plane of the electron beam can differ, as well as the collimation angle $\theta_{\mathrm{col}}$. Three variables are used in the optimisation, unlike the two in the RB optimisation method in Section~\ref{sec:RB_optimisation}. An elliptical beam may provide the optimal solution (maximal collimated flux, minimal bandwidth) because ICS interactions can occur with a crossing angle $\phi$, where the geometry of the interaction is modified and an elliptical beam may provide a better overlap. Laser pulses, whilst often near-round, can be elliptical as in the cERL ICS source demonstration \cite{akagi2016narrow}, and then an elliptical electron beam could provide better overlap between electron bunch and laser pulse.     

The round beam approximation is also a poor approximation in some cases, for example in storage rings electron beams are typically `pancake' in shape where ($\epsilon_{nx} \neq \epsion_{ny}$) such as at the HI$\gamma$S storage ring driven $\gamma$-ray ICS source where the emittance is $\epsilon_{x} = 18$~\si{\nano\meter}--\si{\radian}, $\epsilon_{y} < 1~\si{\nano\meter}$--\si{\radian} \cite{wu1996performance,weller2009research}. Realistic electron beams are also rarely round in practice, therefore elliptical beam optimisation may offer a more precise result.

For an elliptical beam optimisation, the selected $\beta$-function variables can not be easily bound or calculated as a function of the bandwidth as in (Eq.~\ref{eq:beta_star_maximum_limitation}) of the round beam optimisation because the full emittance term (Eq.~\ref{eq:emittance_term}) can not be simply re-arranged. This also means no satisfactory upper bounds on the $\beta$-functions can be derived. Therefore, a different optimisation method to the RB optimisation in Section~\ref{sec:RB_optimisation} must be developed. Hence, an elliptical beam optimisation based on the simplex algorithm is developed in this section.  

\subsection{Simplex Method}
\label{sec:simplex_optimisation}

The downhill simplex method is a local, direct search optimisation method which is used to find a solution to the collimated flux (Eq.~\ref{eq:collimated_flux})--\textit{rms} bandwidth (Eq.~\ref{eq:}) trade-off. The downhill simplex optimisation method is a local minimisation routine as derivatives are effectively used to find minima \cite{jones2016design}, however a minimisation routine can be used to maximise the collimated flux if we use $-\mathcal{F}_{\mathrm{col}}$. A global solution is possible from local optimisation methods if the parameter space of the optimisation problem is not complex, else a local minima solution will be achieved; as both the bandwidth (Eq.~\ref{eq:RMS_bandwidth}) and collimated flux (Eq.~\ref{eq:collimated_flux}) are smoothly varying functions it is reasonable to assume the parameter space is not complex. In practice downhill simplex is efficient in finding global minima where few local minima exist in the parameter space \cite{wolfram2021nmaximize}. In this section the simplex methodology is outlined, its application to the optimsation problem for single bandwidth optimisation is explained, and extension to the tuning curve case is demonstrated. The simplex method will be explained using the \textsc{Mathematica} formalism \cite{wolfram2021nmaximize}, as \textsc{Mathematica}'s \textsc{NMaximise} function is used for the EB optimisation.

\begin{figure}[!h]
\centering
\includegraphics[width=\textwidth]{Figures/Optimisation_and_Characterisation_of_Inverse_Compton_Scattering_Sources/simplex_fixed_4.pdf}
\caption{Diagram of simplex optimisation, where each axis of the 3D plot corresponds to an optimisation variable ($\theta_{\mathrm{col}}$, $\beta_{x}$, $\beta_{y}$). Each set of axes shows a step of the simplex optimisation procedure which is iterated until a solution is found. Left: In the reflection step, a 3D polytope is generated with 4 points (blue + white) where the worst point $x_{4}$ (white) -- the poorest bandwidth and collimated flux point -- is replaced by a trial point $x_{t}$ (red) that is a reflection of the worst point. The trial point replaces the worst point in the new polytope. Middle Left: The expansion step, where if the best point $x_{1}$ (white) is replaced by a trial point $x_{t}$ (red) from the reflection, the trial point is expanded $x_{e}$ (green) in the direction of the reflection and a new polytope is formed. Middle Right: Contraction step, where the worst point $x_{4}$ (white) is replaced by a trial point $x_{t}$ (orange) closer to the midpoint of the polytope and the new polytope is formed. Right: Shrink step, if the worst point is the contracted point $x_{c}$ (orange), then the contraction is repeated and $x_{c}$ in the polytope is replaced by a shrink trial point $x_{s}$ (yellow) closer to the midpoint and a new polytope is formed. }
\label{fig:simplex_method_diagram}
\end{figure}
In downhill (minimisation) simplex optimisation, for an optimisation problem with $n$ variables, a set of $n+1$ points ($x_{1}$--$x_{n+1}$) are used to form the vertices of a polytope in an $n$-dimensional parameter space \cite{wolfram2021nmaximize}. A diagram illustrating the simplex optimisation method is shown in Fig.~\ref{fig:simplex_method_diagram}. For example, in the EB optimisation there are three variables so the polytope is a 4-vertex polyhedron in parameter space. Selection of the vertex points can be prescribed, but random selection allows the potential to fully investigate the parameter space \cite{koshel2002enhancement}. The objective function $f\left(x\right)$ -- in this case the negative collimated flux -- is calculated for each of these vertices and they are sorted into the form
\begin{equation}
f\left(x_{1}\right) \leq f\left(x_{2}\right) \leq \ldots \leq f\left(x_{n+1}\right),
\label{eq:simplex_polytope_objective_functions}
\end{equation}
where the objective functions are ordered from minimum objective function ($x_{1}$, best point) to maximum objective function ($x_{n+1}$, worst point). The worst point is then replaced by a trial point $x_{t}$, given by
\begin{equation}
x_{t} = c+s_{r}\left(c-x_{n+1}\right),
\label{eq:simplex_trial_point}    
\end{equation}
which denotes a reflection of the worst point $x_{n+1}$ through the midpoint of the polytope, with $s_{r} > 0$ the refection parameter, and the midpoint $c$ of the $n$-dimensional polytope is given by
\begin{equation}
c = \frac{1}{n}\sum_{i=1}^{n}x_{i}.
\label{eq:polytope_centroid_simplex}
\end{equation}
If $x_{t}$ minimises the objective better than the best point ($f\left(x_{t}\right) \leq f\left(x_{1}\right)$), then reflection is a successful objective function minimisation method. The polytope is then expanded in the reflected direction because this could provide further minimisation of the objective function. The new trial point $x_{e}$ is given by
\begin{equation}
x_{e} = c+s_{e}\left(x_{t}-c\right),
\label{eq:simplex_expansion}    
\end{equation}
where $s_{e} > 1$ is the expansion parameter. If $f\left(x_{e}\right) < f\left(x_{t}\right)$, $x_{e}$ is a better solution and replaces the worst point in the set $x_{n+1}$, otherwise $x_{t}$ replaces $x_{n+1}$. 

Contraction of the polytope could provide a better solution if the new trial point $x_{t}$ is worse than the the second worst point $f\left(x_{t}\right) \geq f\left(x_{n}\right)$, the new trial vertex becomes
\begin{equation}
x_{c} = 
\begin{cases}
c+s_{c}\left(x_{n+1}-c\right), & \text{if}  ~f\left(x_{t}\right) \geq f\left(x_{n+1}\right), \\
c+s_{c}\left(x_{t}-c\right), & \text{if}  ~f\left(x_{t}\right) < f\left(x_{n+1}\right),
\end{cases}
\label{eq:simplex_contraction}
\end{equation}
where $0 < s_{c} < 1$ is the contraction parameter. A further contraction, named a shrink, is carried out if $f\left(x_{c}\right) < f\left(x_{n+1}\right) \land f\left(x_{t}\right)$ because the previous contraction was successful, with $x_{t}$ replacing $x_{n+1}$. A shrink would be of identical form to (Eq.~\ref{eq:simplex_contraction}) with the contraction parameter $s_{c}$ replaced by a shrink parameter $0 < s_{s} < 1$. The method is then iterated until the termination condition -- typically a number of iterations -- is exceeded or until the simplex optimisation converges. 

\subsection{Implementation and Test Case}

The optimisation uses the \textsc{NMaximise} maximisation routine in \textsc{Mathematica} \cite{wolfram2021nmaximize} where the collimated flux (Eq.~\ref{eq:collimated_flux}) is maximised for a user chosen bandwidth value. The optimisation is then extended to tuning curve optimisations by running the single bandwidth simplex optimation many times over a range of bandwidths; typically 200 bandwidth values in the narrowaband range $0 \leq \Delta E_{\gamma}/E_{\gamma} \leq 0.01$ (i.e. 0--1\%) are used.   

A set of constraints are imposed upon the simplex optimisation, as summarised in Table~\ref{tab:simplex_optimisation_settings}, where the variables are upper and lower bounded. For the collimation angle, the upper and lower bounds are unchanged from the round beam optimisation where $\theta_{\mathrm{col}} > 0$ and an approximate upper bound is provided by (Eq.~\ref{eq:collimation_angle_upper_bound}). The $\beta$-function at the IP upper bounds vary from the round beam optimisation because the bandwidth can not be simple re-arranged to for a $\beta$-function constraint and consequently arbitrary lower and upper bounds must be set. Lower bounds of $\beta_{x/y}^{*} > 1$~\si{\milli\meter} are set because even the tightest final focuses (typically used for particle colliders) are limited to the \si{\milli\meter}-scale; for example the KEK ATF2 \cite{okugi2016achievemen} used as a demonstrator for the international linear collider (ILC) \cite{yamamoto2021international} achieves $\beta$-functions at the IP of 0.48~\si{\milli\meter}. An upper bound of $\beta_{x/y}^{*} < 30$~\si{\meter} is arbitrarily set because this means large IP spot sizes of 10's~\si{\milli\meter} are present for each of the benchmarking cases in Table~\ref{tab:char_opt_electron_bunch_parameters}. For example, for case C where the emittance is smallest ($\epsilon_{nx/ny} = 0.1$~\si{\micro\meter}) an electron beam spot sizes at the IP of $\sigma_{x/y} = 12.35$~\si{\milli\meter} is found for $\beta_{x/y}^{*} = 30$~\si{\meter}. These constraints are imposed using a penalty method \cite{myers1973response}, where solutions a
\begin{table}[!h]
\centering
\caption{Simulation settings, including variable bounds, used for the elliptical beam simplex optimisations.}
\vspace{3mm}
\begin{tabular}{lccc}
\hline\hline
\multicolumn{4}{c}{Constraints} \\
\hline
Parameter & Lower Bound & Upper Bound & Unit \\ 
\hline
Collimation Angle, $\theta_{\mathrm{col}}$ & 0 & Eq.~\ref{eq:collimation_angle_upper_bound} & \si{\radian} \\
Horizontal $\beta$-function at IP, $\beta_{x}^{*}$ & $10^{-3}$ & 30 & \si{\meter} \\
Vertical $\beta$-function at IP, $\beta_{y}^{*}$ & $10^{-3}$ & 30 & \si{\meter} \\
Bandwidth Tolerance, $\Omega$ & & $10^{-6}$ & \\
\hline
\multicolumn{4}{c}{Simplex Settings} \\
\hline
Parameter & \multicolumn{3}{c}{Value} \\
\hline
ReflectRatio, $s_{r}$ & \multicolumn{3}{c}{1.0} \\
ExpandRatio, $s_{e}$ & \multicolumn{3}{c}{2.0} \\
ContractRatio, $s_{c}$ & \multicolumn{3}{c}{0.5} \\
ShrinkRatio, $s_{s}$ & \multicolumn{3}{c}{0.5} \\
No. Iterations, $N_{\mathrm{it}}$ & \multicolumn{3}{c}{200} \\
\hline\hline
\end{tabular}
\label{tab:simplex_optimisation_settings}
\end{table}

A further constraint is imposed using the bandwidth of the ICS source, where the bandwidth achieved by the simplex optimisation $\left(\Delta E_{\gamma}/E_{\gamma}\right)_{\mathrm{ach}}$ is forced to converge to a bandwidth chosen by a user $\left(\Delta E_{\gamma}/E_{\gamma}\right)_{\mathrm{tar}}$. Hence, the bandwidth constraint is of the form
\begin{equation}
\Omega > \left|\left(\frac{\Delta E_{\gamma}}{E_{\gamma}}\right)_{\mathrm{tar}}-\left(\frac{\Delta E_{\gamma}}{E_{\gamma}}\right)_{\mathrm{ach}}\right|,    
\label{eq:simplex_epsilon_constraint}
\end{equation}
where $\Omega$ is a tolerance on the bandwidth. A bandwidth tolerance of $\Omega = 10^{-6}$ is used because we typically optimise to narrow bandwidths of $\Delta E_{\gamma}/E_{\gamma} \sim 10^{-4}$ (i.e. 1\%)  and $\Omega = 10^{-6}$ means a 1\% error in optimising to this bandwidth. The method we have used to impose this constraint on the simplex optimisation follows the $\epsilon$-constraint multi-objective optimisation method first described by Haimes \cite{haimes1971modeling}, where an objective -- the bandwidth of the ICS source -- is re-cast into a constraint, whilst we maximise the most important objective the collimated flux. This is a commonly used method in multi-objective optimisation and its formalities are described further by Marler and Arora \cite{marler2004survey}. This constraint then operates like the variable constraints within \textsc{NMaximise}, where the penalty method is used to exclude solutions that lie outside of the constraint.

The simplex EB optimisation uses the default values of the constants $s_{r},~s_{e}, ~s_{c},~s_{s}$, named ReflectRatio, ExpandRatio, ContractRatio, ShrinkRatio within \textsc{Mathematica}, which are tabulated in Table~\ref{tab:simplex_optimisation_settings}. The default values are used because benchmarking has shown that varying combinations of these parameters produce broadly similar results (a variation $ < 2$\%) and optimal settings vary between optimisation cases. A termination condition of $N_{\mathrm{it}} = 200$ iterations is arbitrarily set because this value is large enough to allow all of the single bandwidth optimisations in Section~\ref{sec:single_bandwidth_optimisations} to converge and is around the default 100 iterations \cite{wolfram2021nmaximize}. However, the EB simplex optimisation is limited by this arbitrary termination condition, which may mean that a true optimal solution is not found if finding it exceeds the iteration limit.

The results of the simplex elliptical beam tuning curve optimisations for benchmarking case A (see Section~\ref{sec:benchmarking_cases_characterisation_optimisation}) are shown in Fig.~\ref{fig:case_A_simplex_tuning_curves} for the narrowband ($\Delta E_{\gamma}/E_{\gamma} \leq1$\%) regime. The collimated flux--\textit{rms} bandwdith ($\mathcal{F}_{\mathrm{col}}$--$\left(\Delta E_{\gamma}/E_{\gamma}\right)_{\mathrm{rms}}$) tuning curve in Fig.~\ref{fig:case_A_simplex_tuning_curves} is well defined in the 0--1\% \textit{rms} bandwidth range. The minimum achieved bandwidth is around $\sim0.1$\%, as expected using (Eq.~\ref{eq:bandwidth_limitation_minimum}). A maximum collimated flux of $\sim 1.9\times 10^{9}$~ph/\si{\second} in a 1\% \textit{rms} bandwidth is predicted, identical to Fig.~\ref{fig:case_A_GA_tuning_curves}. The EB simplex method fails to converge for few bandwidth points in the tuning curve, hence the complexity of solution space is minimal, and the downhill simplex method is generally satisfactory for case A. 
\begin{figure}[!h]
\centering
\includegraphics[width=\textwidth]{Figures/Optimisation_and_Characterisation_of_Inverse_Compton_Scattering_Sources/Case_A_simplex_Tuning_Curves.pdf}
\caption{Case A simplex optimisation tuning curves in both solution and parameter space. Top Left: Pareto-optimal front of the collimated flux as a function of \textit{rms} bandwidth. Top Right: Simplex parameter space ($\beta_{x}^{*}$--$\beta_{y}^{*}$) variable front (green) corresponding to the Pareto-optimal front in solution space compared to the monotonically varying round beam variable front (black). Bottom Left: Parameter space ($\beta_{x}^{*}$--$\theta_{\mathrm{col}}$) corresponding to the Pareto-optimal front in solution space. Bottom Right: Parameter space ($\beta_{y}^{*}$--$\theta_{\mathrm{col}}$) corresponding to the Pareto-optimal front in solution space.}
\label{fig:case_A_simplex_tuning_curves}
\end{figure}

The $\beta_{x}^{*}$--$\beta_{y}^{*}$ parameter space tuning curve corresponding to the collimated flux--\textit{rms} bandwdith tuning curve shows that the optimal solution is an elliptical electron bunch profile at the IP because the $\beta_{x}^{*}$-function is typically larger than the $\beta_{y}^{*}$ function in Fig.~\ref{fig:case_A_simplex_tuning_curves}. A larger horizontal spot size is favoured because of the $\phi = 5$\si{\degree} crossing angle between the electron bunch and laser pulse in the $x$--$z$ plane. Both of the $\beta$-function at the IP against collimation angle parameter space plots in Fig.~\ref{fig:case_A_simplex_tuning_curves} display curved fronts similar to those observed in the RB case in Fig.~\ref{fig:CaseA_RB_tuning_curve}. However, the $\beta_{x}^{*}$--$\theta_{\mathrm{col}}$ plot shown numerous points which have failed to converge around $0.03~\textrm{\si{\milli\radian}} < \theta_{\mathrm{col}} < 0.09~\mathrm{\si{\milli\radian}}$. 

\section{Evaluation of Optimisation Methods} 
\label{sec:evaluation_of_optimisation_methods}

Optimisations have been performed for each of the cases in Table~\ref{tab:char_opt_electron_bunch_parameters} using the laser parameters in Table~\ref{tab:char_opt_laser_pulse_parameters} using the RB (Section~\ref{sec:RB_optimisation}), GA NRB and simplex NRB (Section~\ref{sec:NRB_optimisation} methods. Since case B electron bunch parameters don't fulfill the round beam optimisation criteria ($\epsilon_{nx}\neq\epsilon_{ny}$) it has been excluded from the RB method. Within this section, the optimisation methods will be compared through both single point and tuning curve optimisations, to evaluate the applicability of optimisation towards narrowband radiation production, quantify the advantage of NRB optimisation over RB optimisation and examine the feasibility of each optimisation method. 

Applying these optimisation techniques in practice requires further studies. For example, accelerator jitter may impede the the optimised electron bunch transverse profile at the interaction point from being produced and the effect of this error on source performance needs to be quantified. Errors such as misalignment of the incident laser pulse and collimator must also be studied. This is a subject for future work.  

\subsection{Single Bandwidth Optimisations}
\label{sec:single_bandwidth_optimisations}

Single point bandwidth optimisations of the benchmarking cases in Table~\ref{tab:char_opt_electron_bunch_parameters}, with laser parameters in Table~\ref{tab:char_opt_laser_pulse_parameters}, are used to compare optimisation methods (RB, NRB simplex and NRB GA) at small \textit{rms} bandwidths (0.5\% \texit{rms} bandwidth for case A and C and 2\% \texit{rms} bandwidth for case B) to quantify the advantage of using each method and for comparison against the standard transverse profile matching approach in Section~\ref{sec:transverse_profile_matching}. The single point bandwidth optimised results of each case are shown in Table~\ref{tab:single_point_optimisations}. 

Comparison between optimisation methods will determine if optimising the transverse dynamics for maximal collimated flux in a narrow bandwidth is a worthwhile strategy and ascertain the benefit of using a non-round beam optimisation, beyond the obvious advantage that RB optimisation is only applicable when certain criteria are satisfied. Through examination of the variables related to the optimal solution, ICS source design can be further understood and the question of whether a round transverse profile is optimal can be answered.      

\begin{table}[!h]
\centering
\caption{Single point optimisations for the transverse profile matching (TPM), round beam (RB), non-round beam simplex (NRB simplex) and non-round beam genetic algorithm (NRB GA) methods of all benchmarking cases. \textit{Rms} bandwidths of 0.5\% (2\%) are chosen to evaluate the single point optimisation methodologies for case A and C (case B). The optimisations are compared via the collimated flux, which is maximised, and the variables corresponding to the optimal solution.}
\vspace{3mm}
\begin{threeparttable}
\begin{tabular}{lccccc}
\hline\hline
Parameter & TPM & RB & NRB simplex & NRB GA & Unit \\
\hline
\multicolumn{6}{c}{Case A 0.5\% \textit{rms} bandwidth} \\
\hline
Collimated flux & $7.37\times 10^{8}$ & $8.62\times 10^{8}$ & $8.81\times 10^{8}$ & $8.86\times 10^{8}$ & ph/\si{\second} \\
$\beta_{x}^{*}$-function & 3.53 & 1.15 & 1.69 & 2.11 & \si{\meter}\\
$\beta_{y}^{*}$-function & 3.53 & 1.15 & 0.88 & 0.87 & \si{\meter}\\
Collimation angle & 0.068 & 0.066 & 0.066 & 0.066 & \si{\milli\radian}\\
\hline
\multicolumn{6}{c}{Case B 2\% \textit{rms} bandwidth} \\
\hline
Collimated flux & -- & -- & $9.19\times 10^{9}$ & $1.04\times 10^{10}$ & ph/\si{\second} \\
$\beta_{x}^{*}$-function & -- & -- & 3.46 & 11.05 & \si{\meter} \\
$\beta_{y}^{*}$-function & -- & -- & 0.10 & 0.20 & \si{\meter} \\
Collimation angle & -- & -- & 0.178 & 0.194 & \si{\milli\radian} \\ 
\hline
\multicolumn{6}{c}{Case C 0.5\% \textit{rms} bandwidth} \\
\hline
Collimated flux & $8.35\times 10^{7}$ & $1.16\times 10^{8}$ & $1.16\times 10^{8}$ & $1.17\times 10^{8}$ & ph/\si{\second} \\
$\beta_{x}^{*}$-function & 0.45 & 0.015 & 0.016 & 0.029 & \si{\meter} \\
$\beta_{y}^{*}$-function & 0.45 & 0.015 & 0.015 & 0.013 & \si{\meter} \\
Collimation angle & 2.63 & 2.62 & 2.63 & 2.64 & \si{\milli\radian}\\
\hline\hline
\end{tabular}
\begin{tablenotes}
\item[*]{Transverse profile matching settings aren't possible because the emittance term (Eq.~\ref{eq:emittance_term}) for $\sigma_{\mathrm{electron}} = \sigma_{L}$ leads to a larger than 2\% \textit{rms} bandwidth.}
\item[$\dagger$]{This case doesn't satisfy the round beam condition ($\epsilon_{nx} \neq \epsilon_{ny}$), therefore the RB optimisation can't be utilised.}
\end{tablenotes}
\end{threeparttable}
\label{tab:single_point_optimisations}
\end{table}

Comparing the collimated flux for case A in Table~\ref{tab:single_point_optimisations}, the collimated flux in the transverse profile matching case is exceeded by all optimised cases, with a 17.0\% increase in collimated flux from the RB optimisation and a 20.2\% increase in collimated flux from the GA NRB optimisation, with respect to the transverse profile matching case. There is a small 2.8\% increase in collimated flux of using a GA NRB optimisation in comparison to a RB optimisation, however a small increase expected as the optimised case A adheres to the round beam criteria well ($\epsilon_{nx}=\epsilon_{ny}=\epsilon_{n}$, $\beta_{x}^{*}\sim\beta_{y}^{*}$). The collimated flux advantage in the NRB method must result from the angular crossing luminosity reduction term (Eq.~\ref{eq:angular_crossing_factor}) because each transverse plane of the interaction is otherwise identical. The $\beta$-functions in each plane of this case differ, therefore the optimal transverse profile of the electron bunch is not round. The simplex NRB case produces a reduced collimated flux in comparison to the GA method; the variation in the solutions is observed in the $\beta_{x}^{*}$ variables, however there is also a negligible difference in collimation angle. 

Single point optimisation results for case B, in Table~\ref{tab:single_point_optimisations}, show both the transverse profile matching method and round beam optimisation could not be used. The transverse profile matching method leads to a large emittance term (Eq.~\ref{eq:emittance_term}) in the bandwidth, which means a \textit{rms} bandwidth of 2\% can't be achieved -- in fact an \textit{rms} bandwidth below $\sim$20\% is not achievable. The round beam optimisation is, as aforementioned, unsuitable due to the mismatched emittances in each plane ($\epsilon_{nx}\neq\epsilon_{ny}$). Comparison of the collimated flux of the GA NRB case to the simplex NRB case shows a collimated flux increase of 13.2\% by using the GA method. The large discrepancy between NRB optimisations may be evidence of the simplex method encountering a local minima; mismatched emittances ($\epsilon_{nx}\neq\epsilon_{ny}$) may complicate the solution space. The variability in NRB optimisation is much larger in case B than the other cases. Furthermore, there is a large discrepancy in the $\beta_{x}^{*}$ function at the IP, with the simplex method yielding a $\beta_{x}^{*}$ function $\sim3$ times lower than in the GA method with similar factor of 2 discrepancies in $\beta_{y}^{*}$ and a 8.9\% increase in the collimation angle with respect to the downhill simplex method. The transverse profile of the electron bunch in this case is highly asymmetric, for example for the GA NRB optimisation $\sigma_{\mathrm{electron},x} = 389.2$~\si{\micro\meter} and $\sigma_{\mathrm{electron},y} = 5.8$~\si{\micro\meter}.

For case C, Table~\ref{tab:single_point_optimisations} shows that 38.9\% and 40.1\% increases in collimated flux are achieved via using the round beam and GA non-round beam optimisation methods respectively over transverse profile matching -- a significant improvement. However, comparing the NRB solutions with the RB solution, no notable increase in collimated flux is achieved. The NRB optimisations have provided similar collimated flux to the RB optimisation because case C fits the round beam criteria particularly well; the emittance is very small, identical in each plane and the short bunch length suppresses the crossing angle effect where the $\beta_{x}^{*} > \beta_{y}^{*}$ bias is introduced. Very modest asymmetry in the transverse profile of the electron bunch in the case C GA NRB optimisation is observed, which reinforces the case A conclusion that a round electron bunch transverse profile in not optimal.      

To summarise, in Table~\ref{tab:single_point_optimisations} the GA NRB single point bandwidth optimisation method consistently provides the most optimal solution, though the simulation time is much longer: $t_{\mathrm{sim}}\approx 5$~\si{\minute} for simplex NRB optimisation compared with $t_{\mathrm{sim}}\approx 50$~\si{minute} for GA NRB optimisation. Though reduction of the 200 generation termination condition could reduce the time of the GA method, whilst reducing optimisation precision. The NRB optimisation method is found to be necessary in cases where the emittances are mismatched and therefore it is especially valuable for storage ring driven ICS sources because ERL and linac drivers conform more readily to the RB criteria. Case A and C showed that optimisation provided on the order of 10's\% more collimated flux than the transverse profile matching scheme, therefore optimisation is necessary for high flux, narrowband ICS based light source operation. The optimal transverse profiles of all the trialed cases in Table~\ref{tab:single_point_optimisations} are asymmetric, so NRB optimisation can be worthwhile when emittances are identical in each plane. The downhill simplex method may potential encounter local minima, as potentially observed in case B simplex NRB optimisation. The most obvious solution to the problem of local minima is to combine the simplex optimiser with a non-derivative-based heuristic optimisation method \cite{jones2016design}, therefore combining a limited (by no. generations) GA NRB method to find a global solution with a downhill simplex method to quickly fine tune the optimisation may be the best method of optimisation. A combined GA and downhill simplex method would be a suitable investigation for further study.

\subsection{Tuning Curve Optimisations}

Comparison of tuning curve optimisations via differing methods for the three test cases in Table~\ref{tab:char_opt_electron_bunch_parameters} allows evaluation of the optimisation methods efficacy at mapping out the capabilities of an ICS source. Comparison of different ICS sources would also be furthered by comparison of their tuning curves, not just single configurations, as is best practice in other light source facilities such as synchrotron light sources and free electron lasers. Solution and parameter space tuning curves, such as those in Figs.~\ref{fig:}, would also enable ICS source users to find optimal configurations for experiments. 

A comparison of the tuning curve optimisation methods for the case A parameters in Table~\ref{tab:char_opt_electron_bunch_parameters} is shown in Fig.~\ref{fig:case_A_optimisation_comparison}. The case A solution space tuning curve ($\mathcal{F}_{\mathrm{col}}$--$\left(\Delta E_{\gamma}/E_{\gamma}\right)_{\mathrm{rms}}$) shows a small increase in collimated flux for the NRB solutions in comparison to the RB method, the difference in collimated flux between these methods increases with a wider bandwidth. As expected, the minimum achievable bandwidth (Eq.~\ref{eq:bandwidth_limitation_minimum}) of each of the tuning curves produced by the three methods agrees as this is limited by the energy spread of the electron bunch and laser pulse spectral bandwidth.

\begin{figure}[!h]
\centering
\includegraphics[width=\textwidth]{Figures/Optimisation_and_Characterisation_of_Inverse_Compton_Scattering_Sources/CaseAoptcomp.pdf}
\caption{Comparison of the three optimisation methods: RB (black), GA NRB (red) and simplex NRB (blue), used in maximal collimated flux narrowband ICS optimisation for the case A parameters (see Table~\ref{tab:char_opt_electron_bunch_parameters}). $\beta$-functions at the IP in each plane and collimation angle are varied. Top Left: Solution space ($\mathcal{F}_{\mathrm{col}}$--$\left(\Delta E_{\gamma}/E_{\gamma}\right)_{\mathrm{rms}}$) Pareto-optimal fronts. Top Right: Parameter space ($\beta_{x}^{*}$--$\beta_{y}^{*}$) variable fronts corresponding to the Pareto-optimal fronts in solution space. Bottom Left: Parameter space ($\beta_{x}^{*}$--$\theta_{\mathrm{col}}$) variable fronts corresponding to the Pareto-optimal fronts in solution space. Bottom Right: Parameter space ($\beta_{y}^{*}$--$\theta_{\mathrm{col}}$) variable fronts corresponding to the Pareto-optimal fronts in solution space.}
\label{fig:case_A_optimisation_comparison}
\end{figure}

The $\beta$-function at the IP parameter space tuning curve in Fig.~\ref{fig:case_A_optimisation_comparison} ($\beta_{x}^{*}$--$\beta_{y}^{*}$) demonstrates that the non-round transverse profile solution is favoured in case A because the optimal NRB solution differs from the RB solution. The simplex optimisation results show good agreement with the genetic algorithm results, with the latter showing a spread of possible solutions which are less concentrated beyond $\beta_{x}^{*} = 15$~\si{\meter}. The simplex optimisation shows a strongly linear relationship between the $\beta$-functions. 

The two $\beta_{x/y}^{*}$--$\theta_{\mathrm{col}}$ parameter space plot for the case A electron bunch parameters both have the characteristic `elbow' shape, however the NRB GA results are stratified. The NRB simplex method agrees well with the NRB GA method. Anomalous points are visible in the simplex data set for small $\beta^{*}$, which conform to the round transverse profile optimisation solution and could be local minima solutions. The $\beta_{x}^{*}$--$\theta_{\mathrm{col}}$ tuning curve of the NRB solution appears translated with respect to the RB solution, with larger $\beta_{x}^{*}$ functions for identical collimation angles. Good agreement is shown between the $\beta_{y}^{*}$--$\theta_{\mathrm{col}}$ parameter space tuning curves from the simplex and genetic algorithm methods, with less stratification in the $\beta_{y}^{*}$ parameter than in the $x$ plane $\beta$-function at the IP from the GA NRB optimisation, showing less sensitivity to $\beta_{x}^{*}$. Variation between the round beam and non-round beam optimisations is of a smaller magnitude than the $\beta_{x}^{*}$--$\theta_{\mathrm{col}}$ tuning curve with smaller $\beta_{y}^{*}$-functions favoured.   


The collimated flux--\textit{rms} bandwidth tuning curves for each optimisation method are shown in Fig.~\ref{fig:case_B_optimisation_comparison)} for a 0--2\% bandwidth range. The collimated flux--\textit{rms} bandwidth tuning curve shows that the NRB optimisation methods generally agree well. However, there are many anomalous local minima points within the simplex data set and the parameter space in case B may be more complicated than case A due to the asymmetric emittance. At the minimum bandwidth region of the tuning curve there is also a discrepancy between the simplex and genetic algorithm methods where the GA Pareto-optimal front becomes curved and non-linear whereas the simplex method remains linear. The discrepancy is likely caused because of few individuals at the narrowband end of the GA simulation -- more generations and more individuals may remedy this discrepancy.

\begin{figure}[!h]
\centering
\includegraphics[width=\textwidth]{Figures/Optimisation_and_Characterisation_of_Inverse_Compton_Scattering_Sources/CaseBoptcomp.pdf}
\caption{Comparison of the two optimisation methods: Simplex NRB (blue) and GA NRB (red), for the case B parameters (see Table~\ref{tab:char_opt_electron_bunch_parameters}). Top Left: Solution space ($\mathcal{F}_{\mathrm{col}}$--$\left(\Delta E_{\gamma}/E_{\gamma}\right)_{\mathrm{rms}}$) Pareto-optimal fronts. Top Right: Parameter space ($\beta_{x}^{*}$--$\beta_{y}^{*}$) variable fronts corresponding to the Pareto-optimal fronts in solution space. Bottom Left: Parameter space ($\beta_{x}^{*}$--$\theta_{\mathrm{col}}$) variable fronts corresponding to the Pareto-optimal fronts in solution space. Bottom Right: Parameter space ($\beta_{y}^{*}$--$\theta_{\mathrm{col}}$) variable fronts corresponding to the Pareto-optimal fronts in solution space.}
\label{fig:case_B_optimisation_comparison}
\end{figure}

The $\beta$-function at the IP parameter space ($\beta_{x}^{*}$--$\beta_{y}^{*}$) in Fig.~\ref{fig:case_B_optimisation_comparison)} shows that the solutions to the NRB optimisation are strongly biased toward a larger $\beta_{x}^{*}$ and smaller $\beta_{y}^{*}$. The tuning curves for the simplex and GA NRB optimisation methods agree well though there are some anomalous local minima points within the simplex dataset and some points are clustered around $\beta_{x}^{*} = 10$~\si{\meter}. Many local minima can be observed in the simplex NRB $\beta_{x}^{*}$--$\theta_{\mathrm{col}}$ parameter space tuning curves, however this broadly agrees with the GA NRB method. The $\beta_{y}^{*}$--$\theta_{\mathrm{col}}$ parameter space shows that $\beta_{y}^{*} < 1$~\si{\meter} are favoured past 50~\si{\micro\radian} collimation angle, though the simplex NRB method fails to define the large $\beta_{y}^{*}$-function region ($\theta_{\mathrm{col}} < 50$~\si{\micro\radian}) of the `elbow' shape plot unlike the NRB GA tuning curve optimisation. 

The case C collimated flux--\textit{rms} bandwidth tuning curves in Fig.~\ref{fig:case_C_optimisation_comparision} shows good agreement between all three methods (RB, simplex NRB, GA NRB), unlike in case A and B, because the emittance is very small ($\epsilon_{n} = 0.1$~\si{\milli\meter}--\si{\milli\radian}) and the electron bunch is short, so the angular crossing effect (Eq.~\ref{eq:angular_crossing_factor}) is minimal. Around $2.4\times 10^{8}$~ph/\si{\second} are available in a 1\% rms bandwidth. As in case B, a discrepancy is also present between the GA NRB and the simplex NRB optimisation in the narrowband region $\left(\Delta E_{\gamma}/E_{\gamma}\right)_{\mathrm{rms}} < 0.15$\% of the tuning curve. The discrepancy occurs because of a lack of individuals in the GA NRB optimisation achieving the narrowest bandwidths.       

\begin{figure}[!h]
\centering
\includegraphics[width=\textwidth]{Figures/Optimisation_and_Characterisation_of_Inverse_Compton_Scattering_Sources/CaseCoptcomp.pdf}
\caption{Comparison of the three optimisation methods: RB (black), GA NRB (red) and simplex NRB (blue), for the case C parameters (see Table~\ref{tab:char_opt_electron_bunch_parameters}). Top Left: Solution space ($\mathcal{F}_{\mathrm{col}}$--$\left(\Delta E_{\gamma}/E_{\gamma}\right)_{\mathrm{rms}}$) Pareto-optimal fronts. Top Right: Parameter space ($\beta_{x}^{*}$--$\beta_{y}^{*}$) variable fronts corresponding to the Pareto-optimal fronts in solution space. Bottom Left: Parameter space ($\beta_{x}^{*}$--$\theta_{\mathrm{col}}$) variable fronts corresponding to the Pareto-optimal fronts in solution space. Bottom Right: Parameter space ($\beta_{y}^{*}$--$\theta_{\mathrm{col}}$) variable fronts corresponding to the Pareto-optimal fronts in solution space.}
\label{fig:case_C_optimisation_comparision}
\end{figure}

The $\beta_{x}^{*}$--$\beta_{y}^{*}$ parameter space tuning curve shows that the GA NRB and simplex NRB optimisations favour $\beta_{x}^{*} > \beta_{y}^{*}$, however there are some GA NRB solutions clustered around $\beta_{x}^{*} = 14$~\si{\meter}, $\beta_{y}^{*} = 8$~\si{\meter} which differ from the linear trend. The clustered points correspond to the narrowest bandwidth solutions. In the $\beta_{x/y}^{*}$--$\theta_{\mathrm{col}}$ parameter space plots the 'elbow' shapes are very severe like the  case B $\beta_{y}^{*}$--$\theta_{\mathrm{col}}$ tuning curve in Fig.~\ref{fig:case_B_optimisation_comparison)} because the emittance in the direction of the $\beta$-function is very small. The gradient of the `elbow' shape $\beta^{*}$--$\theta_{\mathrm{col}}$ plots becomes steeper with small emittance. The simplex method appears to fail to map the steep, high $\beta$-function ($\beta_{x/y}^{*} > 3$~\si{\meter}), small collimation angle ($\theta_{\mathrm{col}} < 0.3$~\si{\milli\radian}) section of the parameter space tuning curves which correspond to the narrowest bandwidth solutions. The simplex NRB method performs poorly due to the sharp gradient, which results in local minima, as small increases in the $\beta$-functions yield small reductions in the bandwidth, which slows the convergence of the simplex optimisation. Similarly the NRB GA method struggles to map the parameter space, evidenced by the reduction in Pareto-optimal individuals around $3~\si{\meter} < \beta_{x}^{*} < 10$~\si{\meter}. 

In summary, the tuning curves in Figs.~\ref{fig:case_A_optimisation_comparison}, \ref{fig:case_B_optimisation_comparison)} and \ref{fig:case_C_optimisation_comparision} show that the optimisation methods developed within this chapter (RB, simplex NRB, GA NRB) are adept at mapping out the parameter space of an ICS source. Solution space tuning curves have shown the collimated flux increases near linearly with the \text{rms} bandwidth, and that there is a low bandwidth cut-off that can not be surpassed due to the energy spread of the electron bunch and spectral bandwidth of the laser pulse. Parameter space tuning curves, which correspond to the tuning curves in collimated flux--\textit{rms} bandwidth space, such as the $\beta_{x}^{*}$--$\beta_{y}^{*}$ tuning curves have demonstrated the optimal solution is to have a non-round transverse electron bunch at the interaction point. Whilst the $\beta_{x/y}^{*}$--$\theta_{\mathrm{col}}$ parameter space tuning curves for each case have shown that the narrowest bandwith ICS source configurations involve high $\beta$-functions at the IP and small collimation angles. The trade-off between the $\beta_{x/y}^{*}$-$\theta_{\mathrm{col}}$ creates an `elbow' shaped curve in the parameter space, which has a steeper gradient with increasingly small emittance.      

\section{Summary}

Within this chapter an analytical collimated flux calculation (Eq.~\ref{eq:collimated_flux}) has been developed which is beneficial over calculations such as the formulation by Curatolo et al (Eq.~\ref{eq:curatolo_collimated_flux}) \cite{curatolo2017analytical} because it fully accounts for a crossing angle between the laser pulse and electron bunch (in both cross section and geometric luminosity reduction) as well as the hourglass effect (Eq.~\ref{eq:furman_hourglass_reduction}). The derived collimated flux formula is generalised, but excludes the energy spread of the electron bunch and laser pulse spectral bandwidth. An ICS spectrum code \textsc{ICARUS} has been developed based on the improved and corrected model by Sun et al \cite{sun2009characterizations,sun2011theoretical} accounting for recoil of the electron bunch, emittance effects, collimation, energy spread of the electron bunch, spectral bandwidth of the laser pulse and is valid in the head-on ($\phi=0$) linear ($a_{0} \ll 1$) regime. The collimated flux calculation, Curatolo et al collimated flux calculation (Eq.~\ref{eq:curatolo_collimated_flux}), \textsc{ICARUS} spectrum code and \textsc{ICCS3D} spectrum code created by Krafft et al \cite{krafft2016laser,ranjan2018simulation} have been benchmarked against each other yielding good agreement between all except the formulation by Curatolo et al \cite{curatolo2017analytical}. Agreement between these codes allowed for the spectrum of an ICS source to be quantitatively characterised by these methods, with confidence in the spectral density units. Overall, this work has allowed for better characterisation of ICS sources, which is necessary in maximising their performance.

Studies have shown that simplistic methods of ICS source IP design like transverse profile matching are sub-optimal, consequently a series of optimisations have been developed toward maximising the collimated flux within a narrow bandwidth via adjusting the electron bunch $\beta$-functions at the IP and the collimation angle. Narrow bandwidth ICS sources provide good resolution of energy dependent phenomena for experiments and high collimated flux improves the signal-to-noise ratio and data acquisition times of measurements. The optimisation of the electron bunch transverse profile and the collimation angle by the the round beam (Section~\ref{sec:RB_optimisation}), simplex non-round beam (Section~\ref{sec:simplex_optimisation}) and genetic algorithm non-round beam (Section~\ref{sec:genetic_algorithm_optimisation}) methods has shown an advantage of up to $\sim 40$\% in collimated flux, and NRB optimisations have consistently out-performed RB optimisations. Two optimisation types have been developed: single point bandwidth optimisations and tuning curve optimisations. Single point optimisations allow the optimal configuration of an ICS source for a particular bandwidth to be found whereas tuning curves map the possible operational settings of an ICS source -- as is best practice for other accelerator light sources such as synchrotron light sources and free electron lasers. Tuning curve optimisations have shown there is a natural limit on the bandwidth of an ICS source because of the energy spread of the electron bunch and laser pulse spectral bandwidth, that the relationship between bandwidth and collimated flux is linear for an ICS source and that increasing the $\beta$-functions at the IP whilst minimising the collimation angle allows for the production of the narrowest bandwidth radiation. Advancements in both characterisation and optimisation enable ICS sources to be designed with higher collimated flux and narrower bandwidths, with better prediction of the spectral output. 

\end{document}