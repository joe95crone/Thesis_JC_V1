%%%%%%%%%%%%%%%%%%%%%%%%%%%%%%%%%%%%%%%%%%%%%%%%%%%%%%%%%%
%
% Doctoral Thesis Template @ The University of Manchester
% LaTeX Chapter Template
% Version 1 (23/07/2020)
% Joe Crone
%
% This template is based on:
% The University of Manchester, Presentation of Thesis Policy
% Research Office Graduate Education Team
% June 2017
% http://www.regulations.manchester.ac.uk/pgr-presentation-theses/
%
%%%%%%%%%%%%%%%%%%%%%%%%%%%%%%%%%%%%%%%%%%%%%%%%%%%%%%%%%%
\documentclass[../main.tex]{subfiles}
\begin{document}

% Title
%--------------------------------------------------------
\chapter{Conclusion}
\label{Conclusion} % to reference use \ref{ChapterTemplate}

\section{Conclusions}
% reference it heavily to the three aims set out in intro
% set out what I have previously done + investigated
    % investigated ICS sources on ERLs via two designs
    % investigated optimisation and characterisation methods
    % investigated their applicability vs other light sources - where are they relevant
    % investigated the applications they are most suited to
% re-iterate main results
    % developed new methods for calculating collimated flux (performance parameter) of ICS + spectrum code, which showed previous col flux methods were insufficient + benchmarked with other spec codes
    % narrow bandwidth radiation requires small emittance, and correctly chosen collimation + spot size at IP + optimised (via 3 methods) to provide best solutions which are non-round (elliptical electron beams at IP), producing tuning curves as common practice for other light sources
    % first pass design of an ICS source for CBETA multi-turn ERL and an applications beamline (other uses)
    % compared with synchrotron sources to show ICS sources become feasible beyond 300 keV
    % designed a gamma ray ICS source for a future conceptual ERL - capable of producing highest demonstrated flux in smallest demoed bandwidth, compared with other radiation production methods to show beneficial by monochromaticity, investigated several potential applications noting the advantages of photonuclear production of medical isotopes

This thesis explores the application of inverse Compton scattering sources to energy recovery linacs via the design of two ERL driven ICS sources: the CBETA ICS source, producing hard x-rays up to 402.5~\si{\kilo\electronvolt} based on the recently commissioned CBETA ERL \cite{bartnik2020cbeta}, and the DIANA ICS source, designed for production of up to 20.11~\si{\mega\electronvolt} $\gamma$-rays. Several methodologies have been developed to characterise and optimise an ICS source, as presented in Chapter~\ref{Optimisation_and_Characterisation_of_Inverse_Compton Scattering_Spectra}, based on the theory presented within Chapters~\ref{Energy_Recovery_Linac_Design} and \ref{Photon_Production_by_Inverse_Compton_Scattering}. Developed methods include optimisations of the electron beam and collimation parameters at the ICS source interaction point to maximise collimated flux whilst minimising bandwidth, an analytical derivation of the collimated flux produced by an ICS source and an semi-analytical spectrum code \textsc{ICARUS}, written as part of this work. These methods are applied to both the CBETA ICS source design in Chapter~\ref{CBETA_Inverse_Compton_Scattering_Source_Design} and the DIANA ICS source design in Chapter~\ref{DIANA_Inverse_Compton_Source_Design}.

Within Chapter~\ref{CBETA_Inverse_Compton_Scattering_Source_Design}, the design of an ICS source in the hard x-ray regime and an ICS beamline for bypassing the CBETA FFA electron beam transport are shown. The CBETA ICS source is then compared with other x-ray ICS sources and with synchrotron light sources. Finally, a set of possible applications of a high flux, narrowband hard x-ray source like the CBETA ICS source are explored. Similarly, the design of a $\gamma$-ray ICS source upon the conceptual DIANA ERL is presented in Chapter~\ref{DIANA_Inverse_Compton_Source_Design}, where first-pass design parameters are presented and spectral output is predicted. The DIANA ICS source is compared to other ICS sources and the bremsstrahlung method of $\gamma$-ray production. Suitable applications of a narrowband, high flux $\gamma$-ray source are then investigated, with focus on nuclear resonance fluorescence and photonuclear radioisotope production. Consequently, the material presented in the preceding chapters has satisfied the scope of the investigation outlined in Section~\ref{sec:thesis_layout_scope}: to investigate possible ERL based ICS source configurations and compare these with other accelerator types, to find the optimum configuration for production of high flux, narrowband radiation and to evaluate the most fitting applications for ERL driven ICS sources in the context of other accelerator light sources.

As shown in Chapters~\ref{Photon_Production_by_Inverse_Compton_Scattering} and \ref{Optimisation_and_Characterisation_of_Inverse_Compton Scattering_Spectra}, ICS sources alleviate the requirement of monochromators necessary for narrow bandwidth in other accelerator driven light sources, such as synchrotron light sources, because in an ICS source $E_{\gamma} = f\left(\theta\right)$ so monochromation is achieved via simple collimation. It can also be noted that high average beam current, small emittance and a small interaction spot size of the electron bunch are crucial for high flux, narrowband ICS sources. An ERL electron beam can provide all of these as evidenced by the ICS source designs in Chapters~\ref{CBETA_Inverse_Compton_Scattering_Source_Design} and \ref{DIANA_Inverse_Compton_Source_Design}. However, other approaches such as non-equilibrium or low emittance storage rings provide similar parameters; further investigation is required to determine if ERLs are the ideal accelerator choice for ICS sources. 

The spectrum code \textsc{ICARUS} has been developed in Chapter~\ref{Optimisation_and_Characterisation_of_Inverse_Compton Scattering_Spectra} and benchmarked against the \textsc{ICCS3D} spectrum code for several cases in Figs.~\ref{fig:ICARUS_optimised_benchmarking} and \ref{fig:CBETA_spectrum_benchmarking} with good agreement throughout. The analytical collimated flux calculation derived in Chapter~\ref{Optimisation_and_Characterisation_of_Inverse_Compton Scattering_Spectra} is consistent with the \textsc{ICARUS} semi-analytical spectrum code also developed in this chapter, demonstrated by the agreement in Fig.~\ref{fig:curatolo_collimated_flux_comparison} which also shows that the calculation of Curatolo et al \cite{curatolo2017analytical} is deficient. Chapter~\ref{Optimisation_and_Characterisation_of_Inverse_Compton Scattering_Spectra} also demonstrates that the bandwidth and collimated flux of an ICS source can be improved by proper trade-off of the collimation angle and $\beta$-functions (spot size) of the electron bunch at the interaction point. For a narrow 0.5\% bandwidth, up to 40\% increase in collimated flux is predicted in Table~\ref{tab:collimated_flux_calculations} using the elliptical electron beam optimisation. Similar increases in collimated flux are also available with the round beam optimisation, however some cases may only be optimised using the elliptical beam optimisation because of asymmetric emittance ($\epsilon_{nx} \neq \epsilon_{ny}$). An elliptical spot size of the electron bunch at the IP is typically favoured, as evidenced by Table~\ref{tab:single_point_optimisations} and Figs.~\ref{fig:case_A_optimisation_comparison}, \ref{fig:case_B_optimisation_comparison} and \ref{fig:case_C_optimisation_comparision}, because of the crossing angle imposed between the electron bunch and the counter-propagating laser pulse. 

The CBETA ICS source design demonstrates that ICS sources out-perform (in terms of flux and average brilliance) synchrotron light sources beyond scattered photon energies of 300~\si{\kilo\electronvolt}, as shown in Fig.~\ref{fig:ICS_vs_SPRING8_Undulator_Flux}. Therefore, the CBETA ICS source could be used to increase the photon energies available at Cornell University from the CHESS synchrotron. Bandwidths of 0.5\% \textit{rms} are possible for the CBETA ICS source -- narrower than the demonstrated bandwidth of any x-ray ICS source --- whilst a high flux is maintained, as evident from Table~\ref{CBETA_spectral_output}. Table~\ref{tab:xray_ICS_comparison} shows the CBETA ICS source is also capable of higher flux than other previously demonstrated x-ray ICS sources, demonstrating that the multi-turn ERL approach, with reasonable laser and optical cavity specifications, is a viable approach to high flux, narrow bandwidth ICS sources. With an adjustable collimator system (or set of collimators) and the variable final focus implemented in the ICS bypass the collimated flux and bandwidth of the CBETA ICS source are tunable to experimental requirements, evidenced by the tuning curve in Fig.~\ref{fig:CBETA_Tuning_Curve}.   

Design of the DIANA ICS source demonstrates that $\gamma$-rays with a maximum energy of 20.11~\si{\mega\electronvolt} can be produced with a high flux of up to $6.08\times 10^{10}$~ph/\si{\second}, where $1.30\times 10^{9}$~ph/\si{\second} are generated in a narrow 0.5\% \textit{rms} bandwidth, as presented in Table~\ref{tab:DIANA_spectral_output}. The maximum flux of the DIANA ICS source is comparable to the proposed (state-of-the-art) ELI-NP-GBS \cite{elinp2019vega,tanaka2020current} ICS source and could out-perform the current highest flux demonstration HI$\gamma$S \cite{weller2009research} by two orders of magnitude, as shown in Table~\ref{tab:gammaray_ICS_comparison}. Whilst bremsstrahlung sources such as ARIEL \cite{dilling2013ariel,lebois2011simulations} are expected to exceed ICS sources in $\gamma$-ray flux, the continuous nature of the bremsstrahlung spectrum and subsequent monochromation requirements mean bremsstrahlung is not suitable for narrowband radiation production. Finally, the DIANA ICS source design has been investigated for photonuclear radioisotope production of samarium-153; the DIANA ICS source could produce moderate specific activities of samarium-153 (around 60--70~\si{\mega\becquerel}/\si{\milli\gram}) with shelf-life advantages over existing production methods because no samarium-154 impurities arise in photonuclear production. 

\section{Future Work}

\subsection{Extension of Codes}
% extension of understanding to the non-linear regime?
% generalisation of ICARUS code to all crossing angles + variety of pulse + bunch shapes
% improvement of optimisation codes (GA + simplex) + longitudinal or laser optimisation
% jitter + error studies
Currently, all of the derivations, optimisations and spectrum code in this thesis are only relevant within the linear regime of ICS ($a_{0}\ll 1$) i.e for non-intense laser pulses. However, state-of-the-art conventional laser technologies are capable of producing intense laser pulses up to $a_{0}\sim227$ \cite{yoon2021realization} ($\lambda = 800$~\si{\nano\meter}, $I = 1.1\times 10^{23}$~\si{\watt}/\si{\centi\meter}$^{2}$) and high intensity laser pulses can be used to generate short wavelength radiation with higher order harmonics \cite{babzien2006observation,seipt2011nonlinear}. Higher harmonic generation reduces the electron energy required for production of photons at shorter wavelengths and consequently an ICS source can be made more compact. Therefore, extension of the codes and flux derivations to the non-linear regime would enable prediction of spectra in non-linear ICS sources. 

The \textsc{ICARUS} code, developed for this work, could be improved via extension to arbitrary crossing angles as the spectrum code is currently only valid for the head-on case ($\phi=0$). Extension to arbitrary crossing angles would enable \textsc{ICARUS} to better calculate spectra from ICS sources where laser pulse--electron bunch interactions occur in Fabry-Perot optical cavities, which typically impose a small crossing angle. \textsc{ICARUS} could be further improved by allowing selection of several distribution types such as uniform, Lorentzian, top-hat (laser) and arbitrary distributions to model the laser pulse and electron bunch as currently only Gaussian bunches and pulses are modelled.

Furthermore, investigation of laser pulse and electron beam jitter and final electron beam focus errors are required to verify the feasibility of the optimisations in Chapter~\ref{Optimisation_and_Characterisation_of_Inverse_Compton Scattering_Spectra}. Jitter in position and spot size of the laser pulse and electron bunch at the interaction point may not accommodate the degree of laser pulse and electron bunch control at the IP required for effective IP optimisations. Optimisations could also be extended for longitudinal phase space or laser pulse spot size in each plane trade-offs at the IP though, as mentioned in Chapter~\ref{Optimisation_and_Characterisation_of_Inverse_Compton Scattering_Spectra}, these present additional challenges.   

\subsection{ERL Driven ICS Source Designs}
% generalisation of CBETA ICS bypass to all energies, not just the maximum 150 MeV
% further development of the DIANA ERL, through development of a lattice model 

The CBETA ICS source design in Chapter~\ref{CBETA_Inverse_Compton_Scattering_Source_Design} could be extended via generalisation of the ICS bypass design in Section~\ref{sec:bypass_design} to each of the lower nominal electron energies (42, 78, 114~\si{\mega\electronvolt}) re-circulated in the ERL, not just the 150~\si{\mega\electronvolt} maximum energy. Design of the bypass for these energies is more complicated because the beam traverses the CBETA return loop twice at these energies (both accelerating and decelerating beams). However, a multi-energy bypass design would allow for multi-colour photon production from CBETA. The CBETA ICS bypass design could also be iterated upon with tracking and collective effects studies, building upon understanding of the existing CSR and BBU limitations of the CBETA ERL \cite{lou2019beam,lou2020coherent}.

The DIANA ERL is a conceptual design, therefore the ICS source design in Chapter~\ref{DIANA_Inverse_Compton_Source_Design} can be improved via a full lattice optics design. Lattice design would provide better validation of the current electron beam design parameters and also allow for development of electron beam final focus optics integrated within the accelerator. From a lattice design, the spectral output of the DIANA ICS source (flux, brilliance etc.) could be better predicted, and optimisations of the interaction point could be re-performed for iteration of the ICS source design. Applications of the DIANA ICS source require further investigation, such as design of particular NRF experiments that are applicable to DIANA and the exploration of other candidates for photonuclear radioisotope production.     

\subsection{Demonstration of an ICS Source}
% design of a practical ICS demonstration on CLARA FEBE (being conducted) 
Future work toward an ICS source demonstration utilising the full energy beam exploitation (FEBE) compact linear accelerator for research and applications (CLARA) linac \cite{angal2020design} is in progress. Production of 1.11--1.48~\si{\mega\electronvolt} $\gamma$-rays, with a moderate flux of $\sim 10^{7}$~ph/\si{\second} is predicted. Practical aspects neglected in current work, such as jitter and misalignment studies, should be conducted for this experiment. Component specifications of the interaction laser, collimator and detector are required for the ICS experiment demonstration and the best approaches for these are being decided. A CLARA FEBE ICS source demonstration could be used to verify some methodologies and approaches within this thesis, for example the \textsc{ICARUS} spectrum code will be used to predict the detected spectrum. First pass experiments, such as demonstration of low energy NRF, are also being investigated for the FEBE CLARA ICS source. An internal technical note is currently being prepared on this topic.

\end{document}