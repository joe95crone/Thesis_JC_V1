%%%%%%%%%%%%%%%%%%%%%%%%%%%%%%%%%%%%%%%%%%%%%%%%%%%%%%%%%%
%
% Doctoral Thesis Template @ The University of Manchester
% LaTeX Chapter Template
% Version 1 (23/07/2020)
% Joe Crone
%
% This template is based on:
% The University of Manchester, Presentation of Thesis Policy
% Research Office Graduate Education Team
% June 2017
% http://www.regulations.manchester.ac.uk/pgr-presentation-theses/
%
%%%%%%%%%%%%%%%%%%%%%%%%%%%%%%%%%%%%%%%%%%%%%%%%%%%%%%%%%%
\documentclass[../main.tex]{subfiles}
\begin{document}

% Title
%--------------------------------------------------------
\chapter{Photon Production by Inverse Compton Scattering}
\label{Photon_Production_by_Inverse_Compton_Scattering} % to reference use \ref{ChapterTemplate}

\section{Electron - Photon Interactions}
\label{sec:electron_photon_interactions}
\textcolor{blue}{**BLURB ABOUT PHOTON - ELECTRON INTERACTIONS**}
% AHHHHHH THIS IS SHIT

While scattering interactions, specifically inverse Compton scattering interactions, are possible from all particles, the work presented here is only concerned with electron - photon interactions, the most common form of inverse Compton scattering.  Electron - photon interactions are advantageous over other particle - photon interactions due to their low rest mass $m_{e}$, resulting in a typically large Lorentz factor, which allows for the production of high energy photons as the scattered photon energy $E_{\gamma}$ is proportional to the Lorentz factor squared ($E_{\gamma}\propto\gamma^{2}$).    

\textcolor{blue}{**EXPLANATION OF THOMSON, COMPTON AND INVERSE COMPTON**}

Thomson scattering \cite{thomson1904xxxiv}, where an incident photon accelerates a charged particle and the charged particle emits a photon of identical energy, we assume the collision is elastic and therefore $p_{1} = p_{2}$ i.e that the recoil of the electron is negligible. The criterion for the photon - electron interaction to be modelled by Thomson scattering is when the energy of the photon is much smaller than the rest mass of the electron. Thomson's electron - photon interaction model was furthered by the work of A.H. Compton \cite{compton1923quantum}. Compton scattering is the extension of Thomson scattering to the inelastic scattering case ($p_{1} \neq p_{2}$), where the electron recoils and the photon is scattered with a different wavelength to the wavelength of the incident photon.

Within this section we are concerned with inverse Compton scattering, the process of scattering a photon from a relativistically-moving  electron in which the scattered photon energy is double Doppler shifted, the scattered photon parasitically gains energy from the recoiling photon.  

\textcolor{blue}{**GENERAL FORM OF THE PHOTON-ELECTRON INTERACTION + FEYNMAN DIAGRAMS**}

Quantum electrodynamic photon - electron interactions, in general, can be represented by the two leading order Feynman diagrams as shown in Fig.~\ref{fig:ICS_Feynman_diagrams}. 

\begin{figure}[!h]
    \centering\includegraphics[width=0.7\textwidth]{Figures/Photon_Production_by_Inverse_Compton_Scattering/Berestetskii_ICS_Feynman.pdf}
    \caption{Two tree-level Feynman diagrams, which contribute to the matrix element of the (inverse) Compton scattering process. Left: The scattered photon is emitted with the annihilation of the incident photon, and the incident photon is absorbed with the production of the recoiling electron. Right: The incident photon and electron are absorbed, and the scattered photon is emitted with the recoiling electron. \cite{berestetskii1982quantum}}
    \label{fig:ICS_Feynman_diagrams}
\end{figure}

The general form of the electron - photon interactions can therefore be specified by
\begin{equation}
p_{1} + k_{1} = p_{2} + k_{2},
\label{eq:ICS_process}
\end{equation}

where $p_{1}$ is the four-momenta of the incident electron, $k_{1}$ is the four-momenta of the incident photon, $p_{2}$ is the four-momenta of the recoiling electron, and $k_{2}$ is the four-momenta of the scattered photon. The geometry of the inverse Compton scattering interaction in a 2D plane is shown in Fig.~\ref{fig:scattered_photon_kinematics}.

\begin{figure}[!h]
    \centering
    \includegraphics[width=\textwidth]{Figures/Photon_Production_by_Inverse_Compton_Scattering/scatteringkinematicsdiagram.pdf}
    \caption{Geometry of the inverse Compton scattering event at the interaction point. This geometry follows the geometry prescribed by Sun et al \cite{sun2009energy}. Here $\theta$ is the scattering angle of the photon, with $\phi' = \pi -\phi$ the angle between the incident electron and incident photon, where $\phi$ is the crossing angle of the electron and photon and $\theta' = \pi - \theta - \phi$ the angle between the incident and scattered photon. }
    \label{fig:scattered_photon_kinematics}
\end{figure}

\textcolor{blue}{**4-MOMENTA EQUATIONS**}

Motivated by the geometry of the inverse Compton scattering process as shown in Fig. \ref{fig:scattered_photon_kinematics}, we can write the initial photon $p_{1}$ and electron $k_{1}$ four-momenta
\begin{align}
p_{1} = \gamma m_{e}\left(c,\boldsymbol{v_{i}}\right),
\label{eq:initial_four_vectors} \\
k_{1} = \frac{E_{L}}{c}\left(1,\hat{n}_{i}\right), 
\end{align}

where $E_{L}$ is the energy of the incident photon, $\hat{n}_{i}$ is the unit displacement three-vector of the incident photon, $\gamma$ is the Lorentz factor, $m_{e}$ is the mass of the electron, $c$ the speed of light in a vacuum and $\boldsymbol{v_{i}}$ is the velocity three-vector of the incident electron with magnitude $v_{i}$. Similarly, we can present the final electron and scattered photon states with four-momenta 
\begin{align}
p_{2} = \gamma' m_{e}\left(c,\boldsymbol{v_{f}}\right), \\
k_{2} = \frac{E_{\gamma}}{c}\left(1,\hat{n}_{f}\right), 
\label{eq:final_four_vectors} 
\end{align}

where $E_{\gamma}$ is the scattered photon energy, $\hat{n}_{f}$ is the unit displacement three-vector of the scattered photon, $\gamma'$ is the Lorentz factor of the recoiling electron and $\boldsymbol{v_{f}}$ is the velocity three-vector of the recoiling electron with magnitude $v_{f}$.

\textcolor{blue}{**DERIVATION OF MANDELSTAM VARIABLES + LORENTZ INVARIANTS }

Using our four-momenta definitions (Eq.~\ref{eq:initial_four_vectors}-\ref{eq:final_four_vectors}), a  series of kinematic invariants (Mandelstam variables \cite{mandelstam1958determination}) can be determined for the photon-electron scattering process \cite{berestetskii1982quantum}
\begin{align}
s = \left(p_{1}+k_{1}\right)^{2} = \left(p_{2}+k_{2}\right)^{2},
\label{eq:s_Mandelstam} \\
t = \left(p_{1}-p_{2}\right)^{2} = \left(k_{2}-k_{1}\right)^{2},
\label{eq:t_Mandelstam} \\
u = \left(p_{1}-k_{2}\right)^{2} = \left(p_{2}-k_{1}\right)^{2}.
\label{eq:u_Mandelstam}
\end{align}

The kinematic invariants can be used to form more convenient Lorentz invariants
\begin{align}
X = \frac{s-\left(m_{e}c\right)^{2}}{\left(m_{e}c\right)^{2}},
\label{eq:X_Mandelstam} \\
Y = \frac{\left(m_{e}c\right)^{2}-u}{\left(m_{e}c\right)^{2}},
\label{eq:Y_Mandelstam}
\end{align}
which, when transformed into the geometry as shown in Fig.~\ref{fig:scattered_photon_kinematics}, can be re-wrote as
\begin{align}
X = \frac{2\gamma E_{L}\left(1-\beta\cos\phi'\right)}{m_{e}c^{2}},
\label{eq:X_geometry} \\
Y = \frac{2\gamma E_{\gamma}\left(1-\beta\cos\theta\right)}{m_{e}c^{2}},
\label{eq:Y_geometry}
\end{align}
where $\beta = v/c$, the Lorentz speed factor.

\textcolor{blue}{Do I show the full derivation of the Mandelstam variables? I have done this but still...}
\textcolor{blue}{**STATEMENT OF RECOIL REGIME}

If we take the head-on case ($\phi = 0$) and  assume the incident electron is ultra-relativistic ($\beta \rightarrow 1$) then the $X$ simplifies to 
\begin{equation}
X = \frac{4\gamma E_{L}}{m_{e}c^{2}}.
\label{eq:X_headon}
\end{equation}

The Lorentz invariant $X$ can also be termed the recoil parameter, which denotes the magnitude of the recoil of the incident electron. For the case $X \ll 1$, the recoil is small and inverse Compton scattering reduces to Thomson scattering, the interaction becomes inelastic. To illustrate, using (Eq.~\ref{eq:X_headon}) a 1\% recoil effect ($X = 0.01$) in head-on inverse Compton scattering geometry with a infrared laser (Nd:YAG, $\lambda = 1064$~\si{\nano\metres}) results in a total electron beam energy of $E_{e} = 558$~\si{\mega\electronvolt}, for a visible green laser (2nd Harmonic Nd:YAG, $\lambda = 532$~\si{\nano\metres}) a total electron beam energy $E_{e} = 279$~\si{\mega\electronvolt}. Therefore, as the sources designed in this Thesis typically use infrared lasers, only in designs for $\gamma$-ray ICS's (where the electron beam energy is 100's~\si{\mega\electronvolt}) are recoil effects expected to be non-negligible.     

\section{Non-linear Inverse Compton Scattering}

The inverse Compton scattering process becomes non-linear when the the scattering is performed with a particularly intense source of incident photons. In inverse Compton scattering sources, where the scope of this work lies, the source of photons is commonly a laser pulse. The intensity of the incident photon source is typically characterised using the normalised laser vector potential $a_{0}$ which can be defined for both a Gaussian (Eq.~\ref{eq:a0_gaussian}) and flat-top (Eq.~\ref{eq:a0_flat_top}) pulse \cite{terzic2019improving}
\begin{gather}
a_{0} = \frac{e\sqrt{2c\mu_{0}}}{2\pi m_{e}c^{2}}\lambda\sqrt{\frac{E_{\mathrm{pulse}}}{\left(2\pi\right)^{3/2}\sigma_{L}^{2}t_{\mathrm{pulse}}}},
\label{eq:a0_gaussian} \\
a_{0} = \frac{e\sqrt{2c\mu_{0}}}{2\pi m_{e}c^{2}}\lambda\sqrt{\frac{E_{\mathrm{pulse}}}{2\pi\sigma_{L}^{2}t_{\mathrm{pulse}}}},
\label{eq:a0_flat_top}
\end{gather}
where $e$ is the charge of an electron, $\mu_{0}$ is the permeability of free space, $E_{\mathrm{pulse}}$ is the energy of the incident laser (or photon) pulse, $\sigma_{L}$ is the transverse \textit{rms} spot size (radius) of the laser pulse and $t_{pulse}$ is the \textit{rms} pulse duration. The normalised laser potential is sometimes expressed more empirically \textcolor{blue}{citation?} by $a_{0} \approx 0.855\times 10^{-9} \lambda\left[\mathrm{\mu m}\right]\sqrt{I\left[\mathrm{W/cm^{2}}\right]}$, where $I$ is the intensity of the incident photon pulse.

When $a_{0} \ll 1$ the inverse Compton scattering interaction is termed linear, once this limit is encroached non-linear effects begin to have an effect on an inverse Compton scattering source. Common effects are ponderomotive broadening of the resulting spectra, harmonic generation of higher energy photons and multi-photon inverse Compton scattering \textcolor{blue}{Do these all require citations?}. Ponderomotive broadening and harmonic generation effects are possible with $a_{0}<1$ as their onset is also driven by other factors such as pulse length and shape, but for multi-photon inverse Compton scattering $a_{0}>1$. Unless explicitly stated, all equations within this work are derived for the linear regime as this is the regime the ICS sources presented are designed to operate.

% Ponderomotive Broadening
Ponderomotive broadening is a form of spectral broadening caused by the ponderomotive force of the incident photon pulse acting upon the electron bunch. During a laser pulse - electron beam interaction, the electron is decelerated, then accelerated by the ponderomotive force of the laser
pulse. These velocity shifts lead to frequency shifts in the
emitted radiation, increasing the width of the observed
spectrum \cite{krafft2004spectral}. The broadening effect is visible in the radiation emitted from each laser harmonic. Ponderomotive broadening is therefore a detrimental effect to the pursuit of quasi-monochromatic high-intensity photon beams, which are most favoured by radiation users. 

% Ponderomotive Broadening Mitigation
Hence, mitigation of ponderomotive effects is an extensively studied topic \cite{ghebregziabher2013spectral,terzic2014narrow,seipt2015narrowband,rykovanov2016controlling,terzic2016combining,terzic2019improving}. Strategies to mitigate ponderomotive broadening frequently involve the 'chirping' or frequency modulation of the laser pulse, in which the laser pulse shape is controlled in order to minimize the deceleration and acceleration of the interacted electrons. This is analagous to undulator tapering in free electron lasers The local $a_{0}$ of the laser pulse is modulated by the shape of the pulse in order to accomplish control of the ponderomotive force. 

The laser chirp can be produced with conventional stretcher/compressor and pulse-shaper combinations \cite{ghebregziabher2013spectral}, i.e using existing conventional laser optics or via an FEL oscillator, where the substatial bunch length of the driving electron bunches are of a length where the RF-curvature energy spread is considerable, which means that the resulting laser pulse emitted by the FEL will
also be chirped \cite{terzic2014narrow}. Solutions have been derived for a range of pulse shapes and modulation schemes up to a full 3D laser pulse description \cite{terzic2019improving}. \textcolor{blue}{Has this been demonstrated?}


% Harmonic Generation
\textcolor{blue}{What is Harmonic Generation? May need some work! APS papers aren't working right now!}
Harmonic generation is the non-linear inverse Compton scattering process in which the incident electron is accelerated by the electromagnetic field of the incident laser pulse. This anharmonic acceleration of the electron as it interacts with the radiation field \cite{englert1983second} causes non-sinusoidal transverse oscillations of the electron and induced figure-8 motion, which introduces an overall redshift in the radiation
spectrum, with the concomitant emission of higher order harmonics \cite{sakai2015observation}. The electrodynamics of harmonic generation by free electrons was first described quantum mechanically by Brown and Kibble \cite{brown1964interaction,kibble1965frequency} with a following classical interpretation by Sarrachik and Schappert \cite{sarachik1970classical}. 

The harmonic generation effect at the second harmonic was said to be demonstrated in 1983 \cite{englert1983second}, however the authors could not confirm whether this was harmonic generation or multi-photon inverse Compton scattering. Subsequently, confirmation of harmonic generation was first achieved at Brookhaven National Laboratory \cite{babzien2006observation,kumita2006observation}...\textcolor{blue}{need access to the paper} 

% Multi-Photon Compton Scattering
\textcolor{blue}{What is multi-photon inverse Compton scattering?}
Another non-linear process viable when $a_{0} < 1$ is multi-photon inverse Compton scattering. Multi-photon inverse Compton scattering occurs when the incident photon pulse is of sufficient intensity that two or more incident photons interact with an individual electron simultaneously. As we have two identical photons interacting simultaneously, the total incident photon energy is doubled and consequently the scattered photon energy is doubled.

This process was first conclusively demonstrated in an inverse Compton scattering source by C. Bula et al \cite{bula1996observation} at the Final Focus Test Beam at SLAC \cite{burke1994results} where inverse Compton scattering was observed with 4 incident photons.



\section{Derivation of the Scattered Photon Energy}
\label{sec:derivation_of_the_scattered_photon_energy}
\textcolor{blue}{**DERIVATION OF THE ICS SCATTERED PHOTON ENERGY**}

The relation in (Eq.~\ref{eq:ICS_process}) can be modified in order to calculate the scattered photon energy of an electron - photon interaction. Using the four-momenta in (Eq.~\ref{eq:initial_four_vectors}-\ref{eq:final_four_vectors}) we can create Lorentz invariant quantities from the Minkowski norms of the four-momenta 
\begin{gather}
p_{1\mu}p_{1}^{\mu} = \gamma^{2}m_{e}^{2}\left(v^{2}-c^{2}\right) = -m_{e}^{2}c^{2},
\label{eq:lorentz_invariants1} \\
k_{1\mu}k_{1}^{\mu} = 0.
\label{eq:lorentz_invariants2}
\end{gather}

We can multiply (Eq.~\ref{eq:ICS_process}) by the four-momentum  of the scattered photon and apply (Eq.~\ref{eq:lorentz_invariants2}) the Lorentz invariant 
\begin{gather}
k_{2}^{\mu}\left(p_{1\mu} + k_{1\mu}\right) = k_{2}^{\mu}\left(p_{2\mu} + k_{2\mu}\right), \\
k_{2}^{\mu}p_{1\mu}+k_{2}^{\mu}k_{1\mu} = k_{2}^{\mu}p_{2\mu}.
\label{eq:apply_photon_pfinal}
\end{gather}

Similarly we can construct another equation by inspecting the square of the conservation of four-momentum
\begin{gather}
\left(p_{1}+k_{1}\right)_{\mu}\left(p_{1}+k_{1}\right)^{\mu} = \left(p_{2}+k_{2}\right)_{\mu}\left(p_{2}+k_{2}\right)^{\mu}, \\
p_{1\mu}p_{1}^{\mu}+k_{1\mu}p_{1}^{\mu}+k_{1\mu}p_{1}^{\mu}+k_{1\mu}k_{1}^{\mu} = p_{2\mu}p_{2}^{\mu}+k_{2\mu}p_{2}^{\mu}+p_{2\mu}k_{2}^{\mu}+k_{2\mu}k_{2}^{\mu},
\label{eq:apply_conservation_squared}
\end{gather}
utilising the commutation of four-vectors ($p_{\mu}k^{\mu} = k_{\mu}p^{\mu}$) and our Lorentz invariants (Eq.~\ref{eq:lorentz_invariants1}, \ref{eq:lorentz_invariants2}) we can simplify this further 
\begin{equation}
p_{1\mu}k_{1}^{\mu} = p_{2\mu}k_{2}^{\mu}.
\label{eq:end_conservation_squared}
\end{equation}
Subbing (Eq.~\ref{eq:end_conservation_squared}) into (Eq.~\ref{eq:apply_photon_pfinal}) yields
\begin{equation}
k_{2}^{\mu}p_{1\mu}+k_{2}^{\mu}k_{1\mu} = k_{1}^{\mu}p_{1\mu}
\label{eq:substitution_four_vector}
\end{equation}

Which in three-vector notation is shown as
\begin{equation}
\frac{E_{L}E_{\gamma}}{c^{2}}\left(\hat{n}_{i}\cdot\hat{n}_{f}-1\right)+\frac{E_{\gamma}}{c}\gamma m_{e}\left(\hat{n}_{f}\cdot \boldsymbol{v_{i}}-c\right) = \frac{E_{L}}{c}\gamma m_{e}\left(\hat{n}_{i}\cdot \boldsymbol{v_{i}} -c\right)
\label{eq:three_vector_solution}
\end{equation}

However, (Eq.~\ref{eq:three_vector_solution}) should be presented in terms of the angles in Fig.~\ref{fig:scattered_photon_kinematics}. The dot products within this formula can be replaced using projections to introduce the angular dependencies
\begin{gather}
\hat{n}_{i}\cdot\hat{n}_{f} = \cos\theta',
\label{eq:projection_angle_incident_scattered_photon}\\
\hat{n}_{f}\cdot \boldsymbol{v_{i}} = v_{i}\cos\theta,
\label{eq:projection_scattering_angle}\\
\hat{n}_{i}\cdot \boldsymbol{v_{i}} = v_{i}\cos\phi'.
\label{eq:projection_pi_minus_crossing_angle}
\end{gather}

Upon introducing the projections (Eq.~\ref{eq:projection_angle_incident_scattered_photon}, \ref{eq:projection_scattering_angle}, \ref{eq:projection_pi_minus_crossing_angle}), (Eq.~\ref{eq:three_vector_solution}) becomes
\begin{equation}
E_{L}E_{\gamma}\left(\cos\theta'-1\right)+E_{\gamma}\gamma m_{e}c^{2}\left(\frac{v_{i}}{c}\cos\theta-1\right) = E_{L}\gamma m_{e}c^{2}\left( \frac{v_{i}}{c}\cos\phi'-1\right)
\label{eq:three_vector_solution_projections}
\end{equation}
Using the Lorentz speed factor $\beta = v/c$ and the total electron beam energy $E_{e} = \gamma m_{e}c^{2}$ and rearranging we arive at the general linear, recoil-corrected form for the scattered photon energy resulting from inverse Compton scattering 
\begin{equation}
E_{\gamma} = \frac{\left(1-\beta\cos\phi'\right)E_{L}}{1-\beta\cos\theta+\left(1-\cos\theta'\right)\frac{E_{L}}{E_{e}}}. 
\label{eq:scattered_photon_energy}
\end{equation}
  
In the head-on case ($\phi=0$), (Eq.~\ref{eq:scattered_photon_energy}) can be simplified to 

\begin{equation}
E_{\gamma} = \frac{\left(1+\beta\right)}{1-\beta\cos\theta-\left(\beta-E_{L}/E_{e}\right)\cos\theta},
\label{eq:headon_scattered_photon_energy}
\end{equation}

further simplification by the small angle approximation ($\theta \ll 1$) and for an ultra-relativistic electron beam ($\beta \rightarrow 1$) yields

\begin{equation}
E_{\gamma} \approx \frac{4\gamma^{2}E_{L}}{1+\gamma^{2}\theta^{2}+X},    
\label{eq:small_angle_scattered_photon_energy}
\end{equation}
where the recoil term $X$ is given by (Eq.~\ref{eq:X_headon}). Extending this to backscattering of the incident photon ($\theta = 0$), the scattered photon energy becomes

\begin{equation}
E_{\gamma} = \frac{4\gamma^{2}E_{L}}{1+X},
\label{eq:headon_backscattering_scattered_photon_energy}
\end{equation}
which is referred to in the literature \cite{krafft2010compton} as the Compton edge. The $\gamma^{2}$ factor within this equation refers to the double Doppler shift experienced by the incident photon as it is scattered.

\section{Electron - Photon Interaction Cross Section}

\textcolor{blue}{**START FROM FIRST PRINCIPLES** \\ **PUT IN THE POLARISATION STUFF TOO**}

\textcolor{blue}{**THE CROSS SECTION FORMULA** \\ \textit{Where does this need to go, need to fix an order.}}

The cross section for the inverse Compton scattering interaction \cite{berestetskii1982quantum} is given by \textcolor{blue}{is this excluding non-linear effects?}

\begin{equation}
 \sigma = \frac{2\pi r_{e}^{2}}{X}\left[\frac{1}{2}+\frac{8}{X}-\frac{1}{2\left(1+X\right)^{2}}+\left(1-\frac{4}{X}-\frac{8}{X^{2}}\right)\log{\left(1+X\right)}\right],
 \label{eq:compton_cross_section}
\end{equation}
where $r_{e}$ is the classical radius of the electron. If we take the limit of this in the classical limit ($X \to 0$) we obtain

\begin{equation}
\lim_{X \to 0} \sigma = \frac{8\pi r_{e}^{2}}{3}\left(1-X\right) = \sigma_{T}\left(1-X\right),
\label{eq:compton_cross_section_classical_limit}
\end{equation}
where $\sigma_{T}$ is the Thomson cross section. Where the interaction is firmly in the classical regime ($X \ll 1$), in which inverse Compton scattering becomes Thomson scattering and there is elastic scattering, the Compton cross section recovers the Thomson scattering cross section, $\sigma = \sigma_{T}$. If we take the ultra-relativistic limit ($X \to \infty$) of the cross section, the cross section becomes

\begin{equation}
\lim_{X \to \infty} \sigma = \frac{2\pi r_{e}^{2}}{X}\left(\log{X}+\frac{1}{2}\right).
\label{eq:compton_cross_section_ultrarelativistic_limit}
\end{equation}

\textcolor{blue}{**WOULD IT BE WORTHWHILE HAVING A GRAPH HERE TO SHOW CROSS SECTION VS ENERGY**}

Following the derevation by Berestetskii et al \cite{berestetskii1982quantum} and using the notation of Sun and Wu \cite{sun2011theoretical}, the inverse Compton scattering cross section is given by

\begin{equation}
\frac{d\sigma}{d\Omega} = \frac{8r_{e}^{2}}{X^{2}}\left\{\left[1+P_{t}\left(2\tau-2\phi_{f}\right)\right]\left[\left(\frac{1}{X}-\frac{1}{Y}\right)^{2}+\frac{1}{X}+\frac{1}{Y}\right]+\frac{1}{4}\left(\frac{X}{Y}+\frac{Y}{X}\right)\right\}\left(\frac{E_{\gamma}}{mc^{2}}\right)^{2},    
\end{equation}
where $d\Omega = \sin\theta d\theta d\phi_{f}$ is the differential solid angle, $P_{t}$ is the degree of linear polarisation, $\tau$ is the azimuthal angle of the linear polarization with respect to the $x$ axis, $\phi_{f}$ is the azimuthal angle of the scattering plane and the Lorentx invariant quatity $Y$ is given by

\begin{equation}
Y = \frac{2\gamma E_{\gamma}\left(1-\beta\cos\theta\right)}{mc^{2}}.
\label{eq:cross_section_Y}    
\end{equation}

For a circular or unpolarised incident laser this is negligible ($P_{t}=0$), however this is non-zero for linear polarised cases ($P_{t}\neq0$). Therefore, the distribution of scattered photons is azimuthally symetric for the circularly or unpolarized case polarised case but azimuthally modulated and asymmetric for the linear polarisation case \cite{sun2011theoretical}.  

\section{Luminosity and Flux}
\label{sec:luminosity_and_flux}
\textcolor{blue}{**DERIVATION FROM LARMOR THEOREM TO HEAD-ON LUMINOSITY**}
Applying Larmor's theorem \cite{larmor1897lxiii,purcell1965electricity} in the relativistic generalization \cite{jackson1999classical} and following the derivation by Krafft and Priebe \cite{krafft2010compton}, the luminosity of an ICS source in the head-on ($\phi=0$) case can be derived. Larmor's theorem of radiated power in the relativistic generalisation is 
\begin{equation}
P_{rad} = \frac{\gamma^{4}e^{2}}{6\pi \epsilon_{0}c^{3}}\lvert\mathbf{\dot{v}}\rvert^{2} = \gamma\sigma\epsilon_{0}c\lvert\left(\mathbf{E}+\mathbf{v}\times\mathbf{B}\right)\mathbf{x}\left(t\right)\rvert^{2},
\label{eq:larmor_formula}    
\end{equation}
where $e$ is the charge of an electron,  $\epsilon_{0}$ is the permitivity of free space, $c$ is the speed of light in a vacuum, $\lvert\mathbf{\dot{v}}\rvert$ is the acceleration of the electron in the laboratory frame, $\sigma$ is the Klein-Nishsina cross section \textcolor{blue}{**REPLACE THIS WITH LINKS TO THE CROSS SECTION SECTION**}, $\mathbf{E}$ and $\mathbf{B}$ are respectively the electric field and magnetic flux density of the incident laser, $\lvert\mathbf{v}\rvert$ is the velocity of the electron as it traverses the laser pulse and $\mathbf{x}\left(t\right)$ is the first order approximation of the orbit of the electron as it traverses the incident laser pulse. 
The total energy radiated by the electron $U_{e^{-}}$ in an electron - plane wave photon pulse interaction is given by \cite{krafft2010compton}

\begin{equation}
U_{e^{-}} = \int P\left(t\right)dt = \gamma^{2}\left(1+\beta\right)^{2}\sigma\epsilon_{0}c\int\lvert\mathbf{E}\left(x,y,\left[\beta+1\right]ct\right)\rvert^{2}dt,
\label{eq:electron_radiated_energy}
\end{equation}
where the energy density of a plane wave laser is $\epsilon_{0}\lvert\mathbf{E}\rvert^{2}$. If we seek to generalise this to $U_{\gamma}$, the energy radiated by an electron bunch - photon pulse interaction, we must replace the energy density of a plane wave laser by the energy density distribution of a laser pulse and convolve this with an electron bunch intensity distribution. For a head-on case $\left(\phi=0\right)$, with the laser pulse Doppler shifted into the electron frame, the radiated energy of a electron bunch - laser pulse interaction is    

\begin{equation}
U_{\gamma} = \gamma^{2}\left(1+\beta\right)\sigma E_{L}\int c\left(1+\beta\right) n_{e}\left(\mathbf{r},\mathbf{p},t\right)n_{L}\left(\mathbf{r},\mathbf{k},t\right) d\mathbf{p}~d\mathbf{k}~dV~dt,
\label{eq:total_interaction_energy}
\end{equation}
where $n_{e}\left(\mathbf{r},\mathbf{p},t\right) = N_{e}f_{e}\left(\mathbf{r},\mathbf{p},t\right)$ is the  electron bunch intensity function as a function of the displacement vector $\mathbf{r}$ (integrated over volume $V$), momentum vector $\mathbf{p}$ and time $t$ with $N_{e}$ the number of electrons per bunch and $E_{L}n_{L} = E_{L}N_{L}f_{L}\left(\mathbf{r},\mathbf{k},t\right)$ the energy density distribution of the laser pulse where $E_{L}$ is the incident photon energy, $N_{L}$ is the number of photons in the incident laser pulse and $f_{L}\left(\mathbf{r},\mathbf{k},t\right)$ is the laser pulse intensity function as a function of the displacement vector $\mathbf{r}$, momentum vector $\mathbf{k}$ and time $t$. Within this derivation, the laser pulse and electron bunch are approximated by Gaussian intensity distributions, however any model such as flat-top laser pulses or Lorentzian distributed electron bunches could be substituted. The Gaussian intensity distributions of the electron bunch and laser pulse are given by \cite{sun2011theoretical}
\begin{gather}
f_{e}\left(\mathbf{r},\mathbf{p},t\right) = \frac{1}{\left(2\pi\right)^{3}\varepsilon_{x}\varepsilon_{y}\sigma_{p}\sigma_{z,e}}\exp\left[-\frac{\gamma_{x}x^{2}+\alpha_{x}xx'+\beta_{x}x'^{2}}{2\varepsilon_{x}}-\frac{\gamma_{y}y^{2}+2\alpha_{y}yy'+\beta_{y}y'^{2}}{2\varepsilon_{y}}\right.\\\left.-\frac{\left(p-p_{0}\right)^{2}}{2\sigma_{p}^{2}}-\frac{\left(z-ct\right)^{2}}{2\sigma_{z,e}^{2}}\right], \nonumber
\label{eq:electron_gaussian_intensity_distribution} \\
f_{L}\left(\mathbf{r},\mathbf{k},t\right) = \frac{1}{4\pi^{2}\sigma_{z,L}\sigma_{k}\sigma_{w}^{2}}\exp\left[-\frac{x_{L}^{2}+y_{L}^{2}}{2\sigma_{w}^{2}}-\frac{z_{L}+ct}{2\sigma_{z,L}^{2}}-\frac{\left(k-k_{0}\right)}{2\sigma_{k}^{2}}\right],
\label{eq:laser_gaussian_intensity_distribution}
\end{gather}
where $\varepsilon_{x/y}$ is the emittance of the electron beam in the $x$ and $y$ directions, $\sigma_{p}$ is the fractional momentum spread of the electron bunch, $\sigma_{z,e}$ is the \textit{rms} electron bunch length, $\beta_{x/y}$, $\alpha_{x/y}$ and $\gamma_{x/y}$ are the Twiss parameters in either the $x$ or $y$ direction, $x$ and $y$ are the positions of the electron in both directions, similarly $x'$ and $y'$ are the angular divergences in each direction, $p$ is the magnitude of the momentum of an individual electron and $p_{0}$ is the centroid electron momentum of the bunch. The pulse length of the laser pulse is $\sigma_{z,L}$, $\sigma_{k}$ is the fractional momentum spread of the incident laser pulse, $x_{L}$, $y_{L}$ and $z_{L}$ are the positions of an incident photon in each direction, $k$ is the wavenumber of an individual photon and $k_{0}$ is the centroid wavenumber of the laser pulse and $\sigma_{w}$ is the \textit{rms} transverse waist of the laser pulse given by the usual Gaussian optics formula  \cite{siegmann1986lasers}
\begin{equation}
\sigma_{w} = \sigma_{L}\sqrt{1+\frac{z_{L}}{z_{R}}},
\label{eq:laser_waist}    
\end{equation}
with $\sigma_{L}$ the transverse \textit{rms} spot size of the laser pulse and $z_{R} = 4\pi\sigma_{L}^{2}/\lambda$ the Rayleigh range \cite{siegmann1986lasers} of a laser pulse of wavelength $\lambda$.

The head-on Gaussian luminosity of the interaction can then be separated from the other terms in (Eq.~\ref{eq:total_interaction_energy}), assuming the interaction takes place at the wi=aists of both pulse and bunch, by splitting the spatial terms from the energy spread terms in the electron bunch $f_{e}\left(\mathbf{r},\mathbf{p},t\right) = f_{e}\left(\mathbf{r},t\right)f_{e}\left(\mathbf{p}\right)$ and laser pulse $f_{L}\left(\mathbf{r},\mathbf{k},t\right) = f_{L}\left(\mathbf{r},t\right)f_{L}\left(\mathbf{f}\right)$ of the intensity functions (Eq.~\ref{eq:electron_gaussian_intensity_distribution},~\ref{eq:laser_gaussian_intensity_distribution}) 
\begin{gather}
\mathcal{L}_{\mathrm{HEAD-ON}} = N_{e}N_{L}c\left(1+\beta\right)\int f_{e}\left(\mathbf{r},t\right)f_{L}\left(\mathbf{r},t\right)~dV~dt, \\
\mathcal{L}_{\mathrm{HEAD-ON}} = \frac{N_{e}N_{L}}{2\pi\sqrt{\sigma_{x,e}^{2}+\sigma_{x,L}^{2}}\sqrt{\sigma_{y,e}^{2}+\sigma_{y,L}}} = \frac{N_{e}N_{L}}{2\pi\sigma_{x}\sigma_{y}},
\label{eq:headon_luminosity}
\end{gather}
where $\sigma_{i}(i=x,y,z) = \sqrt{\sigma_{i,e}^{2}+\sigma_{i,L}^{2}}$ is the convolution of the laser pulse and electron bunch spot sizes at the IP. We can then neglect the energy spreads of the electron bunch and laser pulse as these have already been adequately taken into account by using the average cross-section $\sigma$, which already encapsulates the variation in interactions due to energy variations. Therefore, the energy radiated by the interaction of a Gaussian laser pulse and electron bunch becomes  

\begin{equation}
U_{\gamma} = \gamma^{2}\left(1+\beta\right)\sigma\frac{N_{e}N_{L}}{2\pi\sigma_{x}\sigma_{y}}E_{L},
\label{eq:total_interaction_energy_simplified}
\end{equation}
this formula also omits the hourglass effect of the two diverging beams and is also only true in the case of a head-on interaction at the waists of the laser pulse and electron bunch. The solutions for each of these cases is covered in the next section. The number of photons produced per interaction $N_{\gamma}$ can be found using the average scattered photon energy $\gamma^{2}\left(1+\beta\right)\hbar\omega$ \textcolor{blue}{should prove this elsewhere}
\begin{equation}
N_{\gamma} = \sigma\frac{N_{e}N_{L}}{2\pi\sigma_{x}\sigma_{y}^{2}+\sigma_{y}}.
\label{eq:no_photon_headon}
\end{equation}
The total flux of the photons is given by $\mathcal{F} = \sigma\mathcal{L}f$ where $\mathcal{L}$ is the luminosity of the source and $f$ is the repetition frequency of the ICS interaction. The often quoted result \cite{krafft2010compton,curatolo2017analytical}, of the flux in the head-on configuration can be replicated
\begin{equation}
\mathcal{F} = \frac{\sigma N_{e}N_{L}f}{2\pi\sigma_{x}\sigma_{y}}.
\label{eq:headon_flux}
\end{equation}

\section{Angular Crossing and Hourglass Effects}
\label{sec:angular_crossing_and_hourglass_effects}
\textcolor{blue}{**DERIVATION FOR THE ANGULAR CASE, Miyahara \cite{miyahara2008luminosity} or Suzuki \cite{suzuki1976general}**}

A generalization of the head-on ($\phi = 0$) luminosity in (Eq.~\ref{eq:head_luminosity}) to the case with an angular crossing ($\phi \neq 0$) can be performed, where the interaction geometry is modified and the transverse and longitudinal profiles of the interaction are adjusted. The effect of an angular crossing, which is assumed to be in the horizontal $x$ plane, can be expressed by a reduction factor $R_{AC}$ in two identical forms \cite{suzuki1976general,miyahara2008luminosity}

\begin{equation}
R_{AC} = \frac{\sigma_{x}\cos\phi}{\sqrt{\sigma_{x}^{2}\cos^{2}\phi+\sigma_{z}^{2}\sin^{2}\phi}} = \frac{1}{\sqrt{1+\left(\sigma_{z}^{2}/\sigma_{x}^{2}\right)\tan^{2}\phi}},
\label{eq:angular_crossing_factor}    
\end{equation}
where $\sigma_{z,L} = ct_{pulse}$ with $t_{pulse}$, the laser pulse duration. The luminosity for an angular crossing becomes $\mathcal{L} = R_{AC}\mathcal{L}_{\mathrm{HEAD-ON}}$.Therefore, the flux in the case of an angular crossing is given by 
\begin{gather}
\mathcal{F} = \sigma R_{AC}\mathcal{L}_{\mathrm{HEAD-ON}}f, \\
\mathcal{F} = \sigma\frac{N_{e}N_{L}f\cos\phi}{2\pi\sigma_{y}\sqrt{\sigma_{x}^{2}\cos^{2}\phi + \sigma_{z}^{2}\sin^{2}\phi}}.
\label{eq:crossing_angle_flux}    
\end{gather}

\textcolor{blue}{**DERIVATION FOR THE HOURGLASS EFFECT**}

Another deleterious effect on the luminosity is the hourglass effect, in which the effect of the divergence of two colliding beams is taken into account. As the electron bunch and laser pulse overlap, they diverge and their transverse profiles increase in size which in turn reduces the luminosity. The laser pulse typically diverges quicker than the electron bunch. 

The reduction factor of the hourglass effect $R_{HG}$ for the head-on case ($\phi = 0$) as described by M. Furman \cite{furman1991hourglass} for a beam - beam collision of asymmetric beams. Here it is altered for the electron bunch - photon pulse collision, where an electron beam divergence term in $t_{x/y}$ is replaced for a photon pulse divergence term, and it is converted to defined notation 

\begin{equation}
R_{HG} = \frac{1}{\sqrt{\pi}}\int_{-\infty}^{\infty}\frac{\exp\left(-t^{2}\right)}{\sqrt{\left(1+t^{2}/t_{x}^{3}\right)\left(1+t^{2}/t_{y}^{2}\right)}}dt,
\label{eq:furman_hourglass_reduction}    
\end{equation}
where $t$ is the integration variable and the $t_{x/y}$ parameters are given by

\begin{equation}
t_{x/y} = \sqrt{\frac{2\sigma_{x/y}^{2}}{\sigma_{z}^{2}\left(\sigma_{x/y,e}^{2}/\beta_{x/y}^{*2}+\sigma_{L}^{2}/z_{R}^{2}\right)}},
\label{eq:furman_txy_parameters}    
\end{equation}
where $\beta_{x/y}^{*}$ are the $\beta$-functions of the electron beam at the interaction point and $z_{R}=4\pi\sigma_{L}^{2}/\lambda$ is the Rayleigh range as described by A.E. Seigmann \cite{siegmann1986lasers}, with $\lambda$ the wavelength of the incident photon.

An analytical solution to (Eq.~\ref{eq:furman_hourglass_reduction}) can be found for the case of round electron beams ($\sigma_{x,e}=\sigma_{y,e}$), where $t_{x}=t_{y}=t_{RB}$
\begin{equation}
R_{HG} = \sqrt{\pi}t_{RB}\exp\left(t_{RB}^{2}\right)\left[1-\Phi\left(t_{RB}\right)\right],
\label{eq:furman_hourglass_reduction_analytical}    
\end{equation}
where $1-\Phi\left(t_{RB}\right)$ is the complementary error function of $t_{RB}$, with the error function $\Phi$ defined as
\begin{equation}
\Phi\left(x\right) = \frac{2}{\sqrt{\pi}}\int_{0}^{x}\exp\left(-t^{2}\right)dt,
\label{eq:error_function}    
\end{equation}
where $t$ is the integration variable. The non-round beam cases have to be evaluated using numerical integration after inspection of the underlying $R_{HG}$ functions.

The luminosity of a head-on ($\phi = 0$)collision taking the hourglass effect into account is $\mathcal{L} = R_{HG}\mathcal{L}_{HEAD-ON}$, following the methodology of the angular crossing reduction.

The work of Y. Miyahara \cite{miyahara2008luminosity} builds upon the isolated descriptions of the angular crossing by T. Suzuki \cite{suzuki1976general}, and the hourglass effect by M. Furman \cite{furman1991hourglas} by looking at the combined effect of these. The combined angular crossing and hourglass effect reduction factor $R_{ACHG}$ is defined as
\begin{equation}
R_{ACHG} = \int_{-\infty}^{\infty}\frac{H\exp\left(-hZ_{c}^{2}\right)}{\sqrt{\sigma_{x}^{2}+\langle U_{x}^2\rangle Z_{c}^{2}}\sqrt{\sigma_{y}^{2}+\langle U_{y}^{2}\rangle Z_{c}^{2}}}dZ_{c},
\label{eq:miyahara_combined_reduction}    
\end{equation}
where $Z_{c}$ is the integration variable, the parameters $H$ and $h$ are given by
\begin{gather}
H = \cos\phi\sqrt{\frac{\sigma_{x}^{2}\sigma_{y}^{2}}{\pi\sigma_{z}^{2}}},
\label{eq:miyahara_H_parameter} \\
h = \frac{\sin^{2}\phi}{\sigma_{x}^{2}+\langle U_{x}^{2}\rangle Z_{c}^{2}}+\frac{\cos^{2}\phi}{\sigma_{z}^{2}},
\label{eq:miyahara_h_parameter}
\end{gather}
and the divergence term $\langle U_{x/y}^{2}\rangle$ of the electron bunch - laser pulse interaction is
\begin{equation}
\langle U_{x/y}^{2}\rangle = \frac{\left(\sigma_{x/y,e}^{2}/\beta_{x/y}^{*2}\right)+\left(\sigma_{L}^{2}/z_{R}^{2}\right)}{2}.    
\end{equation}
The combined reduction factor $R_{ACHG}$ must be evaluated numerically as no analytical solutions exist for this function. The flux in the case of an angular collision with non-negligible divergence of electron bunches and laser pulses (hourglass effect) is given by

\begin{equation}
\mathcal{F} = \sigma R_{ACHG}\mathcal{L}_{\mathrm{HEAD-ON}}f.
\label{eq:flux_angular_crossing_hourglass}
\end{equation}

The luminosity of this result is given by $\mathcal{L} = R_{ACHG}\mathcal{L}_{\mathrm{HEAD-ON}}$, mirroring the form of the luminosity for the isolated angular crossing and hourglass effect. Naively, one would expect the luminosity to also be appropriately modelled by $\mathcal{L} = R_{AC}R_{HG}\mathcal{L}_{\mathrm{HEAD-ON}}$, however this is incorrect as a moderate crossing angle $\phi$ will shorten the interaction time between the laser pulse and electron beam such that the interaction length is of an order at which the electron bunch and laser pulse minimally diverge and the hourglass effect is negligible. Effectively a moderate crossing angle suppresses the hourglass effect.   

\textcolor{blue}{**DO I INCLUDE THE FULL INVESTIGATION HERE, i.e SHOW THE MIYAHARA REPLICATION}

\section{Source Size and Divergence}
\textcolor{blue}{**Source size derivation**}

The \textit{rms} source size of an inverse Compton scattering electron bunch--laser pulse interaction is 
\begin{equation}
\sigma_{\gamma,x/y} = \frac{\sigma_{e,x/y}\sigma_{L}}{\sqrt{\sigma_{e,x/y}^{2}+\sigma_{L}^{2}}}
\label{eq:source_size}
\end{equation}
where $\sigma_{e,x/y}$ is the \textit{rms} transverse electron bunch spot size in each plane and $\sigma_{L}$ is the \textit{rms} transverse laser pulse spot size.  

The \textit{rms} source angular divergence of an ICS interaction is
\begin{equation}
\sigma_{\gamma,x/y}' = \frac{\sigma_{e,x/y}'\sigma_{L}'}{\sqrt{\sigma_{e,x/y}'^{~2}+\sigma_{L}'^{~2}}},
\label{eq:source_divergence}
\end{equation}
where $\sigma_{e,x/y}' = \sqrt{\epsilon_{x}/\beta_{x}^{*}}$ is the \textit{rms} transverse electron bunch angular divergence and $\sigma_{L} = \sqrt{\lambda/4\pi}$ is the \textit{rms} transverse laser pulse angular divergence. REFERENCE?


\textcolor{blue}{**State how emission can occur from any spatial point, say how it is unlikely from the edges**}
\textcolor{blue}{**Explain how this isn't a problem transversely or longitudinally, what Balsa and I talked about**}



\section{Spectral Density}

\section{Bandwidth + Investigation}
\label{sec:bandwidth}

The bandwidth, the \textit{rms} energy spread of the scattered photons, is given in the recoil corrected non-linear regime ($a_{0}\ll 1$) \cite{ranjan2018simulation} by

\begin{equation}
\frac{\Delta E_{\gamma}}{E_{\gamma}} = \sqrt{\left(\frac{\sigma_{\theta}}{E_{\theta}}\right)^{2}+\left(\frac{\sigma_{e}}{E_{e}}\right)^{2}+\left(\frac{\sigma_{L}}{E_{L}}\right)^{2}+\left(\frac{\sigma_{\epsilon}}{E_{\epsilon}}\right)^{2}},
\label{eq:RMS_bandwidth}    
\end{equation}
where $\frac{\sigma_{\theta}}{E_{\theta}}$ is the collimation term, $\frac{\sigma_{e}}{E_{e}}$ is the electron beam energy spread term, $\frac{\sigma_{L}}{E_{L}}$ is the laser pulse energy spread term and $\frac{\sigma_{\epsilon}}{E_{\epsilon}}$ is the emittance terms. These terms are given by
\begin{gather}
\frac{\sigma_{\theta}}{E_{\theta}} = \frac{1}{\sqrt{12}}\frac{\psi^{2}}{1+X+\psi^{2}/2},
\label{eq:collimation_term} \\
\frac{\sigma_{e}}{E_{e}} = \frac{2+X}{1+X+\psi^{2}},
\label{eq:beam_energy_spread_term} \\
\frac{\sigma_{L}}{E_{L}} = \frac{1+\psi^{2}}{1+X+\psi^{2}},
\label{eq:laser_energy_spread_term} \\
\frac{\sigma_{\epsilon}}{E_{\epsilon}} = \frac{\sqrt{2}\gamma^{2}}{1+X}\sqrt{\frac{\epsilon_{x}^{2}}{\beta_{x}^{*2}}+\frac{\epsilon_{y}^{2}}{\beta_{y}^{*2}}}
\label{eq:emittance_term}
\end{gather}
where $\psi = \gamma\theta$ is the acceptance angle, $\epsilon_{x/y}$ is the emittance in the $x$ or $y$ direction and $\beta_{x/y}^{*}$ are the $\beta$ functions at the IP in both directions. This can be converted into the full width half maximum (FWHM) bandwidth 

\begin{equation}
\left(\frac{\Delta E_{\gamma}}{E_{\gamma}}\right)_{\mathrm{FWHM}} = 2\sqrt{2\ln{2}}\left(\frac{\Delta E_{\gamma}}{E_{\gamma}}\right)_{rms}
\label{eq:FWHM_bandwidth}
\end{equation}

\textcolor{blue}{**MENTION THE CURATOLO BANDWIDTH, HOW THIS DOESN'T MATCH UP-THE PLOT IN RANJAN PAPER** \\ **MENTION THE HAJIMA EXTENTION OF THIS TO RECTANGULAR COLLIMATION** }


\section{Peak and Average Brilliance}
\textcolor{blue}{**DERIVATION OF THE AVERAGE BRILLIANCE \\ Is this necessary?? Is it enough to say flux per area in phase space?}

Brilliance, also termed brightness, is a quantity originally used to quantify the 6D phase space density of particle beams \cite{courant1958theory}. This concept was adopted to characterize production of radiation from accelerator driven light sources such as bending magnet or wiggler synchrotron radiation sources and undulators \cite{kim1989characteristics} where the brilliance is defined as the flux into a cone of particular spectral bandwidth (usually 0.1\%) per unit phase space area. 

Average brilliance is time averaged, summing the contribution of all electron bunch--laser pulse interactions of the source, thereby accounting for the repetition rate of the interactions. Whereas, peak brilliance is calculated for a single bunch--pulse interaction subject to the interaction time. Experimentally, both peak and average brilliance have advantages ... \textcolour{blue}{NEED A GOOD ARGUMENT}. Linac driven ICS sources typically are designed to be single-shot, high peak brilliance sources whereas storage ring and recirculated electron beam based ICS sources take advantage of high repetition rates to be high average brilliance sources. Theoretically, ERL's with linac quality recirculated electron beams could dually provide high peak and average brilliance.   

Often ICS sources are viewed as analagous to undulator radiation, a 'laser undulator' system. Hence, the average brilliance of an ICS source is adapted from that of an undulator. The average brilliance of an undulator source assuming a Gaussian electron bunch and planar undulator field \cite{chao2013handbook} is given by 

\begin{equation}
\mathcal{B}_{U} = \frac{\Phi_{n}}{4\pi^{2}\Sigma_{x}\Sigma_{y}\Sigma_{x}'\Sigma_{y}'}
\label{eq:undulator_brightness}    
\end{equation}
where $\Phi_{n}$ is the total spectral flux of the $n$th undulator harmonic generate into a central cone of 0.1\% spectral bandwidth, $\Sigma_{x/y} = \sqrt{\sigma_{x/y}^{2}+\sigma_{R}^{2}}$ are the source sizes of the interaction in each plane with $\sigma_{x/y}$ the \textit{rms} source size of the electron beam and $\sigma_{R}$ the diffraction limited source size of a single electron emission \cite{kim1987brightness}, and $\Sigma_{x/y}' = \sqrt{\sigma_{x}'^{2}+\sigma_{R}'^{2}}$ the source size divergences with $\sigma_{x}'$ the \textit{rms} divergence of the electron beam and $\sigma_{R}'$ the \textit{rms} angular divergence of the single electron emission \cite{krinsky1983undulators}. 

The average brilliance for an inverse Compton scattering source is defined as the average flux per unit phase space area of the interaction per 0.1\% \textit{rms} bandwidth. The ICS average brilliance, is modified from the undulator brilliance (Eq.~\ref{eq:undulator_brightness}) through replacement of the single electron emission field for the electric field of a Gaussian laser pulse \cite{krafft2010compton,deitrick2018high}, is of the form   

\begin{equation}
\mathcal{B}_{\mathrm{avg}} = \frac{\mathcal{F}_{0.1\%}}{4\pi^{2}\sigma_{\gamma,x}\sigma_{\gamma,x}'\sigma_{\gamma,y}\sigma_{\gamma,y}'},
\label{eq:average_brightness}
\end{equation}
where $\mathcal{F}_{0.1\%}$ is the flux in a 0.1\%  \textcolor{blue}{FWHM?rms?} bandwidth, $\sigma_{\gamma,x/y}$ are the source sizes in each plane (Eq.~\ref{eq:source_size}) and $\sigma_{\gamma,x/y}'$ are the source angular divergences in each plane. The form of the source sizes in the ICS average brilliance 



\end{document}