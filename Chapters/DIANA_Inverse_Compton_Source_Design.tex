%%%%%%%%%%%%%%%%%%%%%%%%%%%%%%%%%%%%%%%%%%%%%%%%%%%%%%%%%%
%
% Doctoral Thesis Template @ The University of Manchester
% LaTeX Chapter Template
% Version 1 (23/07/2020)
% Joe Crone
%
% This template is based on:
% The University of Manchester, Presentation of Thesis Policy
% Research Office Graduate Education Team
% June 2017
% http://www.regulations.manchester.ac.uk/pgr-presentation-theses/
%
%%%%%%%%%%%%%%%%%%%%%%%%%%%%%%%%%%%%%%%%%%%%%%%%%%%%%%%%%%
\documentclass[../main.tex]{subfiles}
\begin{document}

% Title
%--------------------------------------------------------
\chapter{DIANA Inverse Compton Source Design}
\label{DIANA_Inverse_Compton_Source_Design} % to reference use \ref{ChapterTemplate}

\section{The DIANA Energy Recovery Linac}

DIANA, the Daresbury Industrial Accelerator for Nuclear Physics Applications is an applications centric conceptual 3-turn superconducting RF ERL designed for light source operations. The DIANA ERL is anticipated to provide a high brilliance electron beam at a maximum energy of $\sim$\si{\giga\electronvolt} with small relative energy spread ($< 10^{-4}$) and transverse emittance ($< 1$~\si{\milli\meter}-\si{\milli\radian}) pushing the average beam current to the 10's~\si{\milli\ampere} frontier.

High brilliance electron beams on the \si{\giga\electronvolt}-scale can facilitate applications such as a high power extreme ultraviolet (EUV) FEL and a $\gamma$-ray inverse Compton scattering source. An EUV FEL source would have far-reaching consequences for semiconductor lithography providing an unparalleled source of 13.5~\si{\nano\meter} EUV radiation (or some harmonic thereof). Whereas a $\gamma$-ray inverse Compton scattering source would have considerable impact upon nuclear physics and security. Within the scope of the authors current work, the focus is on the development of the latter application as well a progress toward a conceptual design of the DIANA ERL.

A $\gamma$-ray ICS source at moderate energies ($E_{\gamma} < 5$~\si{\mega\electronvolt}) could enable applications such as nuclear resonance fluorescence (NRF) for inspection of nuclear fuel rods, waste studies and detection of clandestine nuclear material to high energy ($E_{\gamma} > 5$~\si{\mega\electronvolt}) applications such as nuclear photonics and medical isotope production. Here special consideration is given to the photo-nuclear production of medical isotopes. 

\section{DIANA ERL ICS Electron Beam and Optical Cavity Laser Pulse Parameters}

The electron bunch parameters for the DIANA ERL ICS are presented in Table~\ref{tab:DIANA_electron_beam_design_parameters}. Baseline parameters are designed for the case of a small interaction point $\beta$-function and therefore a small electron spot size in order to maximise the uncollimated flux and brilliance of the ICS source. This is also useful for highlighting changes within source performance that only occur due to the varying kinetic energy of the electron bunch. A non-round beam optimised case is also shown for maximising the collimated flux into a narrow 0.5\% bandwidth, which is optimised using the \textcolor{blue}{which one?} methodology in Chapter~\ref{Optimisation_and_Characterisation_of_Inverse_Compton Scattering_Spectra}. 

\begin{table}[H]
\centering
\caption{Electron beam parameters forseen at the DIANA ICS source interaction point (IP). Baseline parameters assume a round transverse profile for the electron bunch whereas the optimised parameters are the result of a non-round beam optimisation \textcolor{blue}{what type?}. The given baseline parameters -- which assume the same $\beta^*$ at the IP -- allow a comparison of flux and bandwidth at different energies. The optimised values beneath those are designed to maximise the flux into a 0.5\% \textit{rms} scattered photon bandwidth through a trade-off of $\beta$-function of the electron bunch in each transverse plane and collimation angle.}
\begin{threeparttable}
\begin{tabular}{lccccc}
\hline\hline
Parameter & \multicolumn{3}{c}{Quantity} & Unit \\
\hline
Turn number & 1 & 2 & 3  \\
Injection Energy, $E_{\mathrm{inj}}$ & \multicolumn{3}{c}{7} & \si{\mega\electronvolt}\\
\tnote{$\dagger$}~Electron kinetic energy, $E_e$ & 362 & 717 & 1072 & \si{\mega\electronvolt}\\
Harmonic Frequency, $f$ & \multicolumn{3}{c}{125} & \si{\mega\hertz}\\
Bunch charge, $e N_e$ & \multicolumn{3}{c}{100} & \si{\pico\coulomb} \\
Beam current, $I$ & \multicolumn{3}{c}{12.5} & \si{\mill\ampere} \\
Transverse normalised \textit{rms} emittance, $\epsilon_{N}$ & \multicolumn{3}{c}{0.5} & \si{\milli\meter}-\si{\milli\radian}\\
\tnote{$\sharp$}~\textit{rms} bunch length, $\Delta \tau$ & \multicolumn{3}{c}{0.9 (3)} & \si{\milli\meter} (\si{\pico\second})\\
Bunch spacing, $t_{b}$ & \multicolumn{3}{c}{10} & \si{\pico\second} \\
RF frequency, $f_{RF}$ & \multicolumn{3}{c}{750} & \si{\mega\hertz} \\
\tnote{*}~Absolute energy spread, $\Delta E_{e}$ & \multicolumn{3}{c}{$\sim$10} & \si{\kilo\electronvolt} \\ 
\tnote{*}~Relative energy spread, $\left(\Delta E_{e}/E_{e}\right)$ & \multicolumn{3}{c}{$\sim10^{-5}$} & \\
\hline
\multicolumn{5}{c}{Baseline Parameters} \\
\hline
$\beta$-functions at the IP, $\beta_{x}^{*}$/$\beta_{y}^{*}$ & 0.2/0.2 & 0.2/0.2 & 0.2/0.2 & \si{\meter} \\
Electron bunch spot size, $\sigma_{e,x}$/$\sigma_{e,y}$ & 11.87/11.87 & 8.44/8.44 & 6.90/6.90 & \si{\micro\meter}\\
\hline\multicolumn{5}{c}{Optimised 0.5\% \textit{rms} Bandwidth} \\
\hline
$\beta$-functions at the IP $\beta_{x}^{*}$/$\beta_{y}^{*}$ & 1.33/0.298 & 2.62/0.587 & 3.90/0.874 & \si{\meter} \\
Electron bunch spot size, $\sigma_{e,x}$/$\sigma_{e,y}$ & 30.62/14.49 & 30.54/14.46 & 30.48/14.43 & \si{\micro\meter}\\
Collimation Angle, $\theta_{\mathrm{col}}$ & 0.180 & 0.091 & 0.061 & \si{\milli\radian} \\ 
\hline\hline
\end{tabular}
\begin{tablenotes}
\item[$\sharp$]{Taken from the ASML parameters.}
\item[*]{Estimated values.}
\item[$\dagger$]{Electron beam energies to accomplish $E_{\gamma}^{\mathrm{max}}$ = 20~\si{\mega\electronvolt} $\gamma$-rays. $\Delta E_{\mathrm{turn}}$ = 355~\si{\mega\electronvolt}.}
\end{tablenotes}
\end{threeparttable}
\label{tab:DIANA_electron_beam_design_parameters}
\end{table}

The envisioned parameters of the laser pulse provided by a Nd:YAG ($\lambda = 1064$~\si{\nano\meter}) laser and a four mirror Fabry-Perot optical cavity for use in the DIANA ICS source are shown in Table~\ref{tab:DIANA_laser_pulse_design_parameters}. Again, an Nd:YAG laser is selected for its picosecond domain and narrow spectral bandwidth. Only a single pulse is recirculated within the optical cavity which is designed to be operated with a fixed \textit{rms} laser pulse waist of radius 25~\si{\micro\meter}. A crossing angle of 5\si{\degree} must be imposed due to the geometry considerations of such a source.

These laser pulse parameters are based upon the demonstration of the Fabry-Perot optical cavity in the cERL ICS experiment \cite{akagi2016narrow}. However, they have been modified for a reduced 125~\si{\mega\hertz} repetition rate with an increased pulse energy of 100~\si{\micro\joule} resulting in an increased average stored power of 12.5~\si{\kilo\watt} recirculated in the optical cavity. Therefore, the parameters for the DIANA ICS source are slightly less conservative than those proposed for the CBETA ICS source in Table~\ref{tab:CBETA_laser_pulse_design_parameters} though the MuCLS demonstration of 70~\si{\kilo\watt} \cite{eggl2016munich} affords credibility to these design parameters.

\begin{table}[H]
\centering
\caption{Nd:YAG Gaussian laser pulse parameters at the CBETA ICS IP. The interacted laser pulse is produced via a Nd:YAG infrared laser and re-circulated in a bow-tie Fabry-Perot optical cavity.}
\begin{tabular}{lcc}
\hline\hline
Parameter & Quantity & Unit \\
\hline
Wavelength, $\lambda_\textrm{laser}$ & 1064 & \si{\nano\meter}\\
Photon energy, $E_\textrm{laser}$ & 1.17 & \si{\electronvolt}\\
Pulse energy, $E_{pulse}$  & 100 & \si{\micro\joule}\\
Number of photons, $N_{\textrm{laser}}$ & 5.34$\times 10^{14}$ & \\ 
Repetition rate, $f$ & 125 & \si{\mega\hertz}\\
Spot size at the IP, $\sigma_\textrm{laser}$ & 25 & \si{\micro\meter}\\
Crossing angle, $\phi$ & 5 & deg \\
Pulse length, $\tau_{\mathrm{laser}}$  & 10 & \si{\pico\second}\\
Relative energy spread, $\Delta E_\textrm{laser}/E_\textrm{laser}$ & $\times 10^{-5}$ &   \\
\hline\hline
\end{tabular}
\label{tab:DIANA_laser_pulse_design_parameters}
\end{table}

\section{ICS Source Spectral Output}

The anticipated spectral output of the DIANA ICS source, using the electron bunch and laser pulse parameters specified in Tables~\ref{tab:DIANA_electron_beam_design_parameters},~\ref{tab:DIANA_laser_pulse_design_parameters}, is presented in Table~\ref{tab:DIANA_spectral_output}. 

\begin{table}[H]
\centering
\begin{tabular}{lcccc}
\hline\hline
 & \multicolumn{3}{c}{Electron Kinetic Energy (\si{\mega\electronvolt})} & \\
 \cline{2-4}
 & 362 & 717 & 1072 & \\
\hline
$\gamma$-ray peak energy  & 2.33 & 9.06 & 20.11 & \si{\mega\electronvolt}\\
Source size ($x$/$y$)  & 10.72/10.72 & 8.00/8.00 & 6.65/6.65 & \si{\micro\meter} \\
Uncollimated flux  & 5.77$\times 10^{10}$ & 6.02$\times 10^{10}$ & 6.08$\times 10^{10}$ & ph/\si{\second}\\
Spectral density  & 2.48$\times 10^{5}$ & 6.65$\times 10^{4}$ & 3.03$\times 10^{4}$ & ph/\si{\second} \si{\electronvolt}\\
Average brilliance  & 5.64$\times 10^{12}$ & 2.05$\times 10^{13}$ & 4.45$\times 10^{13}$ & ph/\si{\second} \si{\milli\meter}$^{2}$\si{\milli\radian}$^{2}$ 0.1\% bw\\
Peak brilliance  & 5.60$\times 10^{17}$ & 2.22$\times 10^{18}$ & 4.99$\times 10^{18}$ & ph/\si{\second} \si{\milli\meter}$^{2}$ \si{\milli\radian}$^{2}$ 0.1\% bw\\
\hline
 & \multicolumn{3}{c}{0.5\% \textit{rms} bandwidth} & \\
\hline
Source Size ($x$/$y$) & 19.36/12.54 & 19.35/12.52 & 19.33/12.50 & \si{\micro\meter} \\ 
Collimated flux  & 1.30$\times 10^{9}$ & 1.29$\times 10^{9}$ & 1.29$\times 10^{9}$ & ph/\si{\second} 0.5\% bw \\
\hline\hline
\end{tabular}
\label{tab:DIANA_spectral_output}
\end{table}

\section{$\gamma$-ray ICS Source Comparison}

\section{Bremsstrahlung Source Comparison}

\section{DIANA ICS Applications}
\textcolor{blue}{**$\gamma$-ray applications from CBETA** \\ **SOME MAY BE USEFUL**}

The final ambitious application, nuclear resonance fluorescence (NRF), is a technique suitable for a future, higher-energy ERL based ICS, with an electron beam energy on the order of 350~\si{\mega\electronvolt} or above. This electron beam energy regime boosts the Compton back scattered photons into the regime of gamma rays with an energy of 2.2~\si{\mega\electronvolt} or above. These in turn would be used to excite nuclear levels identifying them with a energy sensitive solid state detector, achieving the nuclear sister spectroscopy to the atomic fluorescence spectroscopy mentioned in our first application. Such spectroscopy would be very useful in assaying nuclear materials, for example identification of manufacturing defects in fission fuel assemblies, nonproliferation security of spent fission fuel and identification of unknown legacy wastes~\cite{angal2018perle,angell2015demonstration,bolind2015states,geddes2017impact,kwan2011discrete}. Moving up to photon energies above $5$~\si{\mega\electronvolt} (requiring a \si{\giga\electronvolt}-scale electron ERL) would open up the nuclear transmutation reactions $(\gamma,p)$, $(\gamma,n)$, $(\gamma,f)$ with potentially far-reaching applications in waste transmutation \cite{ur2017optimization}, the understanding of fission dynamics \cite{bellia1983towards,bhike2017exploratory,finch2018monoenergetic} and bespoke medical isotope production from existing waste streams \cite{habs2011production}. 

\end{document}