%%%%%%%%%%%%%%%%%%%%%%%%%%%%%%%%%%%%%%%%%%%%%%%%%%%%%%%%%%
%
% Doctoral Thesis Template @ The University of Manchester
% LaTeX Chapter Template
% Version 1 (23/07/2020)
% Joe Crone
%
% This template is based on:
% The University of Manchester, Presentation of Thesis Policy
% Research Office Graduate Education Team
% June 2017
% http://www.regulations.manchester.ac.uk/pgr-presentation-theses/
%
%%%%%%%%%%%%%%%%%%%%%%%%%%%%%%%%%%%%%%%%%%%%%%%%%%%%%%%%%%
\documentclass[../main.tex]{subfiles}
\begin{document}

% Title
%--------------------------------------------------------
\chapter{DIANA Inverse Compton Source Design}
\label{DIANA_Inverse_Compton_Source_Design} % to reference use \ref{ChapterTemplate}


\section{DIANA ICS Applications}
\textcolor{blue}{**$\gamma$-ray applications from CBETA** \\ **SOME MAY BE USEFUL**}

The final ambitious application, nuclear resonance fluorescence (NRF), is a technique suitable for a future, higher-energy ERL based ICS, with an electron beam energy on the order of 350~\si{\mega\electronvolt} or above. This electron beam energy regime boosts the Compton back scattered photons into the regime of gamma rays with an energy of 2.2~\si{\mega\electronvolt} or above. These in turn would be used to excite nuclear levels identifying them with a energy sensitive solid state detector, achieving the nuclear sister spectroscopy to the atomic fluorescence spectroscopy mentioned in our first application. Such spectroscopy would be very useful in assaying nuclear materials, for example identification of manufacturing defects in fission fuel assemblies, nonproliferation security of spent fission fuel and identification of unknown legacy wastes~\cite{angal2018perle,angell2015demonstration,bolind2015states,geddes2017impact,kwan2011discrete}. Moving up to photon energies above $5$~\si{\mega\electronvolt} (requiring a \si{\giga\electronvolt}-scale electron ERL) would open up the nuclear transmutation reactions $(\gamma,p)$, $(\gamma,n)$, $(\gamma,f)$ with potentially far-reaching applications in waste transmutation \cite{ur2017optimization}, the understanding of fission dynamics \cite{bellia1983towards,bhike2017exploratory,finch2018monoenergetic} and bespoke medical isotope production from existing waste streams \cite{habs2011production}. 

\end{document}