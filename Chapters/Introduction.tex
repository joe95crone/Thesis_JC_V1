%%%%%%%%%%%%%%%%%%%%%%%%%%%%%%%%%%%%%%%%%%%%%%%%%%%%%%%%%%
%
% Doctoral Thesis Template @ The University of Manchester
% LaTeX Chapter Template
% Version 1 (23/07/2020)
% Joe Crone
%
% This template is based on:
% The University of Manchester, Presentation of Thesis Policy
% Research Office Graduate Education Team
% June 2017
% http://www.regulations.manchester.ac.uk/pgr-presentation-theses/
%
%%%%%%%%%%%%%%%%%%%%%%%%%%%%%%%%%%%%%%%%%%%%%%%%%%%%%%%%%%
\documentclass[../main.tex]{subfiles}
\begin{document}

% Title
%--------------------------------------------------------
\chapter{Introduction}
\label{Introduction} % to reference use \ref{ChapterTemplate}

Energy recovery Linacs (ERLs) are ideal drivers of inverse Compton scattering sources (ICS) due to the combination of linac quality beams and high repetition rate, allowing production of a tunable high-flux, narrowband scattered photon beam. The pioneering demonstration of multi-pass energy recovery in a superconducting RF (SRF) linac with FFAG return loop at the Cornell University Brookhaven National Laboratory Energy Recovery Linac Test Accelerator (CBETA) \cite{hoffstaetter2017cbeta} reveals a route to high energy electron beams for ERL driven ICS production of X-rays and $\gamma$-rays.

Due to a $E_{\gamma} \propto 4\gamma^{2}$ scattered photon energy $E_{\gamma}$ dependence, where $\gamma$ is the Lorentz factor, ICS is the prime candidate for production of high energy photons above photon energies available at conventional X-ray production facilities such as Free Electron Lasers (FEL)($E_{\gamma} <$~25~keV \cite{schneidmiller2011photon}) and the largest synchrotrons ($E_{\gamma} <$~500~keV) \textcolor{blue}{***Needs citation***}. Therefore, inverse Compton scattering sources are also the eminent method for high-flux production of $\gamma$-rays ($E_{\gamma} \sim$~1~MeV), which could support applications like nuclear resonance fluorescence (NRF) and nuclear photonics. ICS sources can be optimised to produce photons in smaller natural bandwidth than synchrotron radiation, alleviating the need for monochromators which inherently deplete the flux of the source. 

Experience gained this year from participating in CBETA commissioning and designing an X-ray ICS utilizing CBETA will ultimately motivate design choices and optimisations for an inverse Compton source operating on the posited Daresbury Industrial Accelerator for Nuclear Physics Applications (DIANA) ERL. The design values for the CBETA ICS are 4.64$\times 10^{8}$~ph/s in a 0.5\% bandwidth \textcolor{blue}{***Check flux figure hasn't changed***}up to a maximum photon energy of 401.4~keV. In comparison, the DIANA $\gamma$-ray source will be capable of producing $\sim 10^{11}$~ph/s in a 0.5\% bandwidth with scattered photon energies in the 1-20~MeV range, competitive with the flagship ELI-NP-GBS \cite{adriani2014technical} linac based inverse Compton scattering source. 

\section{Synchrotron Radiation Production}
\section{Bremsstrahlung Radiation Production}
\textcolor{blue}{**TALK ABOUT BREM**}
\label{sec:bremsstrahlung}

\section{Thesis Layout and Scope}

The thesis is concerned with the investigation of high-energy radiation (x-ray and $\gamma$-ray) production via the interaction of ultra-relativistic electron beams with laser pulses -- known as inverse Compton scattering -- where the electron bunch is generated using an energy recovery linac. Therefore, the thesis has three main foci: 
\begin{enumerate}
    \item{Possible designs of ICS sources as an application of ERLs, the efficacy of the ERL driven approach in comparison to other accelerator types and ERL driven ICS source advantages for operation of x-ray and $\gamma$-ray light sources.}
    \item{The optimum configuration of the electron beam for operation of ICS sources as high flux, narrow bandwidth light sources most favoured by experimentalists.}
    \item{Identification of areas where ICS sources are most applicable in comparison to other light source facility types and the experiments ICS sources enable.}
\end{enumerate}

Consequently, the thesis is structured as follows. Firstly,  Chapter~\ref{Energy_Recovery_Linac_Design} presents an overview of the beam dynamics considerations for design of an ERL based ICS source, then the theory relevant to characterisation and design of an inverse Compton scattering source is presented in Chapter~\ref{Photon_Production_by_Inverse_Compton_Scattering}. Developed methods for optimisation and characterisation of ICS sources are explained and demonstrated in Chapter~\ref{Optimisation_and_Characterisation_of_Inverse_Compton Scattering_Spectra}. Improvements in the characterisation of an ICS source are made via the derivation of an analytical calculation for the flux of an ICS source post-collimation and creation of a semi-analytical spectrum code named \textsc{ICARUS}. A series of optimisations of transverse electron bunch parameters for high flux, narrow bandwidth radiation are produced. The methods developed in Chapter~\ref{Optimisation_and_Characterisation_of_Inverse_Compton Scattering_Spectra} are applied to designs of ERL driven ICS sources in the rest of the thesis. In Chapter~\ref{CBETA_Inverse_Compton_Scattering_Source_Design} an x-ray ERL driven source design based upon CBETA -- the worlds first multi-turn SRF ERL -- is produced, characterised and compared with other competing x-ray sources. Potential applications of such a source are explained. Understanding of the application of ICS sources to x-ray production then led to design of $\gamma$-ray ICS source utilising the conceptual DIANA ERL in Chapter~\ref{DIANA_Inverse_Compton_Source_Design}. The DIANA $\gamma$-ray ICS source design is optimised, characterised and compared to Bremsstrahlung $\gamma$-ray production with several applications of such a source investigated. Finally, in Chapter~\ref{Conclusion} the main findings throughout the thesis are re-iterated and future work relevant to ERL driven ICS sources is discussed.       

\end{document}