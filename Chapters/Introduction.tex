%%%%%%%%%%%%%%%%%%%%%%%%%%%%%%%%%%%%%%%%%%%%%%%%%%%%%%%%%%
%
% Doctoral Thesis Template @ The University of Manchester
% LaTeX Chapter Template
% Version 1 (23/07/2020)
% Joe Crone
%
% This template is based on:
% The University of Manchester, Presentation of Thesis Policy
% Research Office Graduate Education Team
% June 2017
% http://www.regulations.manchester.ac.uk/pgr-presentation-theses/
%
%%%%%%%%%%%%%%%%%%%%%%%%%%%%%%%%%%%%%%%%%%%%%%%%%%%%%%%%%%
\documentclass[../main.tex]{subfiles}
\begin{document}

% Title
%--------------------------------------------------------
\chapter{Introduction}
\label{Introduction} % to reference use \ref{ChapterTemplate}

Energy recovery Linacs (ERLs) are ideal drivers of inverse Compton scattering sources (ICS) due to the combination of linac quality beams and high repetition rate, allowing production of a tunable high-flux, narrowband scattered photon beam from an inverse Compton scattering (ICS) source. Energy recovery linacs, specifically multi-turn ERLs can accelerate electron beams on the 10's~\si{\mega\electronvolt}-scale to the \si{\giga\electronvolt}-scale within a small accelerator footprint whilst maintaining small emittances on the order of 1~\si{\milli\meter}--\si{\milli\radian} and short bunches in the picosecond domain. Parameters such as these have been demonstrated by the first multi-turn superconducting ERL demonstration (at low electron beam current) with the 4-turn energy recovery commissioning run at the Cornell University Brookhaven National Laboratory Energy Recovery Linac Test Accelerator (CBETA) \cite{hoffstaetter2017cbeta,bartnik2020cbeta}. These are ideal parameters for accelerator production of radiation because a high electron beam energy is required to generate short wavelength radiation and a small physical size of the interacted electron beam is required to generate a small, dense and monochromatic radiation beam. Exploitation of ERLs for radiation production has been demonstrated most prominently by the high average power Jefferson Laboratory Free Electron Laser (FEL) \cite{neil2006jlab}, however both the ALICE \cite{priebe2010first} and cERL \cite{akagi2016narrow} single turn ERLs have previously demonstrated ERL based x-ray production from ICS albeit at low flux as a consequence of the low average electron beam current. 

Due to a $E_{\gamma} \propto 4\gamma^{2}$ scattered photon energy $E_{\gamma}$ dependence (see Section~\ref{sec:intro_ICS}), where $\gamma$ is the Lorentz factor, ICS is the prime candidate for production of high energy photons above photon energies available at conventional x-ray production facilities such as FELs ($E_{\gamma} <$~25~keV \cite{schneidmiller2011photon}) and the largest synchrotrons ($E_{\gamma} <$~500~keV \cite{spring8beamlines}). Therefore, inverse Compton scattering sources are the eminent method for high-flux production of $\gamma$-rays ($E_{\gamma}>$~1~MeV). Application of high flux, monochromatic $\gamma$-ray sources to nuclear physics experiments would support applications like nuclear resonance fluorescence (NRF) \cite{hayakawa2010nondestructive} and nuclear photonics \cite{budker2021expanding}. The efficacy of $\gamma$-ray production by inverse Compton scattering for experimentation in nuclear physics has been proven by the HI$\gamma$S ICS source \cite{weller2009research} however, improvements in bandwidth (energy spread or linewidth of the produced radiation) are required for high resolution experiments. Hence, the flagship ELI-NP-GBS project \cite{adriani2014technical,elinp2019vega,tanaka2020current} has been designed for ICS production of $\gamma$-rays up to energies of 19.5~\si{\mega\electronvolt} in a narrow 0.5\% FWHM bandwidth \cite{elinp2019vega}. Whilst other $\gamma$-ray sources exist, such as bremsstrahlung sources, the ICS process offers unparalleled monochromatic radiation production most favoured by experimentalists because ICS sources can be optimised to produce photons in a smaller natural bandwidth than synchrotron radiation via simple collimation, alleviating the need for monochromation which inherently depletes the number of generated photons.

The current status of energy recovery linac development and the most common accelerator driven radiation production methods are further explained in the remaining sections of the introduction to provide the motivation for the development of ERL driven ICS sources. The review of ERLs focuses on radiation production from ERLs, specifically the development of high energy and high average beam current ERLs which are key for short wavelength (beyond nanometer scale) photon production and production of the maximum number of photons. Accelerator driven radiation production methods such as synchrotron radiation facilities, free electron lasers and bremsstrahlung sources are reviewed because these are the production methods most relevant to production of large quantites of short wavelength photons. The scope and structure of the thesis are then explained in Section~\ref{sec:thesis_layout_scope}.    

\section{Energy Recovery Linacs}

The energy recovery linac is a re-circulated linac concept invented by M. Tigner in 1965 \cite{tigner1965possible} to improve the efficiency of particle colliders by re-circulating the particle beam post-interaction for recovery of its kinetic energy in the same RF cavities that provided the initial acceleration. The recovered energy can then be applied to accelerate a following electron bunch, without additional application of RF power, to improve the efficiency of particle accelerators. Another benefit is that the particle beam is decelerated during the energy recovery process therefore the dumped power and kinetic energy of the particle beam is reduced, avoiding neutron activation in the beam stop. A further, more detailed explanation of an energy recovery linac is presented in Chapter~\ref{Energy_Recovery_Linac_Design}.

The ERL concept was first demonstrated in the superconducting accelerator driven Free Electron Laser (SCA/FEL) \cite{smith1987development} -- the first superconducting RF (SRF) ERL. The SCA/FEL was capable of producing $E_{e} = 93$~\si{\mega\electronvolt} electron bunches with 5~\si{\milli\meter}--\si{\milli\radian} normalised emittance, a 5~\si{\pico\second} bunch duration and 150~\si{\micro\ampere} average beam current. The SCA/FEL was a pulsed accelerator, where a train of several electron bunches are produced before a longer time interval between production of several other bunches. As well as the first demonstration of an ERL, the SCA/FEL demonstrated the efficacy of an ERL as a driver of a light source by demonstrating a UV FEL operating at $\lambda = 0.2$~\si{\micro\meter}. Unlike Tigner's initial vision of ERL driven particle colliders \cite{tigner1965possible}, ERL particle beams have consistently been applied to drive radiation production.

ERLs have also been demonstrated using normal conducting RF (NCRF) cavities, such as the Chalk River Reflexotron ($E_{e} = 25$~\si{\mega\electronvolt}) \cite{schriber1977experimental} and the Los Alamos FEL ($E_{e} = 23.5$~\si{\mega\electronvolt})\cite{feldman1987energy}, however these are typically low electron energy machines ($E_{e} < 30$~\si{\mega\electronvolt}). Normal conducting ERLs are limited because NCRF accelerating cavities are susceptible to cavity losses and RF transport losses \cite{adolphsen2022european} in comparison to SRF cavities, therefore the increase in efficiency -- re-use of RF power -- from energy recovery is less beneficial because the RF power is dissipated. As more RF cavities are required for acceleration of particles to higher energies the power dissipation problems in NCRF is exacerbated by ERL based production of short wavelength radiation, which requires higher energy particle beams. For high average beam current (10's~\si{\milli\ampere}) and beyond moderate energies ($E_{e} > 100$~\si{\mega\electronvolt}) with efficient use of RF acceleration, the development of superconducting RF ERLs is required. Superconducting ERLs are consequently the chosen technology for radiation production because the quantity of photons generated is increased with increasing average beam current.  

All ERLs discussed so far were demonstrated using singular or trains of electron bunches, however by using continuous wave ERLs, where electron bunches are continually produced (see Section~\ref{sec:high_current_ERL}), the average beam current of the ERL can be improved. Continuous wave ERLs were first demonstrated \cite{neil2000sustained} by the IR FEL demo \cite{benson1999first,neil2000sustained} at Jefferson Laboratory (J-Lab), where a single turn ERL was used to drive an infrared (IR) FEL. The J-Lab IR FEL demo produced an average beam current of 4.8~\si{\milli\ampere} with a 48~\si{\mega\electronvolt} electron beam producing a peak beam current of 60~\si{\ampere} for production of $\lambda =$ 3--5.3~\si{\micro\meter} infrared radiation with an average power of 1.72~\si{\kilo\watt}. Subsequently, with experience operating the IR FEL demo, an upgraded J-Lab IR FEL was designed \cite{benson2002have} and built \cite{behre2004first} with an upgraded higher average beam current injector, improved beam dynamics design and improved FEL design. The J-Lab IR FEL upgrade produced a maximal average beam current of 9.1~\si{\milli\ampere} (the record for SRF ERLs) with a 160~\si{\mega\electronvolt} electron bunch -- a total beam power of 1.46~\si{\mega\watt} \cite{neil2006jlab}. The IR Upgrade ERL demonstrated nearly 1.5~\si{\mega\watt} of electron beam average power with only around 300~\si{\kilo\watt} of installed RF power \cite{adolphsen2022european}, which demonstrates the ability of the ERL concept to produce high power light sources whilst reducing the required RF power.  

Another ERL, the Continuous Electron Beam Accelerator Facility (CEBAF) \cite{bogacz2003cebaf,tennant2003beam}, has subsequently been demonstrated at J-Lab though CEBAF was originally designed as a re-circulated linac. CEBAF demonstrated single turn energy recovery in 2003 \cite{bogacz2003cebaf,tennant2003beam} with energy recovery of a 1.02~\si{\giga\electronvolt} electron beam, the highest energy electron beam energy recovered, with a normalised transverse emittance of $\epsilon_{nx} \left(\epsilon_{ny}\right) = 2.39 \left(2.06\right)$~\si{\milli\meter}--\si{\milli\radian} \cite{tennant2003beam}. ERLs within the \si{\giga\electronvolt}-scale have uses as $\gamma$-ray factories \cite{budker2021expanding} utilising ICS sources and for particle collider applications within high energy particle physics such as the proposed LHeC electron--hadron collider \cite{agostini2021large}. 

Since demonstration of the J-Lab FEL, the high average beam current frontier of 10's~\si{\milli\ampere} has been pursued by the compact ERL (cERL) at KEK \cite{akagi2016narrow}, where demonstration of a 10~\si{\milli\ampere} average beam current is planned with 100\% energy recovery efficiency \cite{adolphsen2022european}. Currently, an average beam current of 1~\si{\milli\ampere} has been achieved at cERL \cite{obina20191} with an energy recovery efficiency of $100\pm 0.5$\%. Light sources other than FELs have been applied to ERLs such as the 
ICS sources applied to the ALICE ERL at Daresbury Laboratory \cite{priebe2008inverse,priebe2010first}, the first SRF ERL in Europe primarily used as an ERL IR FEL \cite{thompson2014status}, and the ICS source applied to cERL \cite{akagi2016narrow}. The ICS sources take advantage of the high electron beam brightness delivered by an ERL much like FELs. However, ICS source demonstrations upon ERLs have been limited to low energies (ALICE $E_{e}=30$~\si{\mega\electronvolt}, cERL  $E_{e}=20$~\si{\mega\electronvolt}) and low current $I < 1$~\si{\milli\ampere}, so only x-ray photons at low fluxes can be produced. 

The Novosibirsk Recuperator, a NCRF ERL used to drive terrahertz and IR FELs, is the first demonstrated multi-turn ERL \cite{gavrilov2007status} in which the electron bunch is accelerated for multiple passes before deceleration for multiple passes. Multi-turn ERLs are advantageous because the electron beam can be repetitively accelerated, allowing for a larger maximum electron energy for the same RF acceleration section as a single turn ERL. For example, single turn NCRF ERLs such as the Reflexotron have achieved an electron beam energy of 25~\si{\mega\electronvolt} \cite{schriber1977experimental}, whereas the Novosibirsk recuperator accelerates electrons up to energies of 42~\si{\mega\electronvolt} \cite{shevchenko2020novosibirsk}  The highest average electron beam current in an ERL demonstrated at 30~\si{\milli\ampere} has also been achieved in the Novosibirsk FEL \cite{gavrilov2007status}. A recent upgrade of the RF gun has demonstrated up to 100~\si{\milli\ampere} average beam current \cite{matveev2020simulation}, which could provide an average beam current in an NCRF ERL an order of magnitude larger than the 9.1~\si{\milli\ampere} \cite{neil2006jlab} demonstrated for SRF ERLs.  

Multi-turn ERLs have been demonstrated with SRF accelerating structures firstly by the CBETA ERL \cite{bartnik2020cbeta} and then recently by S-DALINAC ERL \cite{adolphsen2022european}. Up to 81.8\% of the electron bunch energy was recovered in S-DALINAC during multi-turn commissioning -- the highest energy recovery efficiency demonstrated for a multi-turn SRF ERL. Both accelerators, CBETA \cite{gulliford2021measurement} and S-DALINAC
\cite{steinhorst2021rf} were first operated as single turn ERLs with 99.4\% and 90.1\% energy recovery efficiency respectively. However, currently only low average beam currents have been demonstrated by multi-turn ERLs, with the S-DALINAC multi-turn ERL demonstrating maximum 8~\si{\micro\ampere} average beam current. The single turn CBETA ERL demonstrated a nominal electron beam energy of 42~\si{\mega\electronvolt} \cite{gulliford2021measurement}, whereas with the 4-turn configuration of CBETA a maximum electron beam energy of 150~\si{\mega\electronvolt} was demonstrated \cite{bartnik2020cbeta}; a factor of 3.57 increase in electron beam energy within an identical footprint. Therefore, CBETA demonstrates the applicability of multi-turn ERLs to the generation of high energy electron beams within compact footprint accelerators. CBETA commissioning is explained in more detail in Section~\ref{sec:CBETA_commissioning}, where a design of an ICS source driven by CBETA is proposed.

A comprehensive plot of the electron beam energy and average beam current achieved by various ERL projects is shown in Fig.~\ref{fig:ERL_Landscape}, which is also indicative of the average electron beam power of ERLs. Currently a maximum electron beam power demonstrated in an ERL is 1.46~\si{\mega\electronvolt} using the J-Lab upgraded FEL \cite{neil2006jlab}, and the maximum electron beam energy demonstrated in an ERL is the $\sim1$~\si{\giga\electronvolt} energy recovery demonstration using CEBAF \cite{bogacz2003cebaf,tennant2003beam}. However, several projects are in development seeking to demonstrate ERLs with higher average current, higher electron bunch energies and multi-turn designs, such as PERLE \cite{angal2018perle}, bERLinPro \cite{kuske2012conceptual} and ER@CEBAF \cite{meot2016er}.

\begin{figure}[!h]
\centering
\includegraphics[width=0.9\textwidth]{Figures/CBETA_Multi-Pass_Commissioning/Tennant_ERL_Landscape.pdf}
\caption{The landscape of legacy, ongoing and proposed ERL projects plotted as a function of maximum electron beam energy and average beam current. Gridlines denote the average electron beam power. Reproduced from the European Strategy for Particle Physics
Accelerator R&D Roadmap \cite{adolphsen2022european}.}
\label{fig:ERL_Landscape}
\end{figure}

The bERLinPro ERL is a single turn ERL which aims to demonstrate high current operation of an ERL with a maximum average beam current of 100~\si{\milli\ampere} at 50~\si{\mega\electronvolt} electron beam energy \cite{kuske2012conceptual,neumann2018berlinpro}. A normalised transverse emittance of 1~\si{\milli\meter}--\si{\milli\radian} emittance with a 2~\si{\pico\second} bunch duration is projected, which would be the highest brightness electron beam produced in an ERL. However, commissioning of the bERLinPro ERL has been stalled by damage to the main linac cryomodule \cite{neumann2018berlinpro}. Many obstacles to high average current operation in ERLs exist, as explained in Section~\ref{sec:high_current_ERL}, such as coherent synchrotron radiation production, beam breakup instability and beam halo -- the latter two are an active area of study for bERLinPro \cite{neumann2012status,hwang2019first}. High current operation is necessary for light source operation of ERLs, such as ICS sources and FELs.  

The ER@CEBAF experiment aims to extend the operation of electron ERL to high electron energy operation with a maximum electron energy of 7.5~\si{\giga\electronvolt} \cite{bogacz2016er,meot2016er}. In addition, the ER@CEBAF ERL is a 5-turn design -- the most turns of any designed or operated ERL -- with the potential to demonstrate the many-turn route to \si{\giga\electronvolt}-scale electron energy ERLs. Attaining high electron energies within a compact footprint is the main advantage of a multi-turn ERL over single turn ERLs, therefore ER@CEBAF is a necessary demonstrator for future ERL based collider projects such as the LHeC electron--positron collider \cite{valloni2013strawman,bruning2019exploring,holzer2021accelerator}. Impact of the synchrotron radiation losses of a 7.5~\si{\giga\electronvolt} electron beam energy, as in the ER@CEBAF ERL design, upon the beam dynamics of an ERL would be challenging therefore the ER@CEBAF project could allow for study of synchrotron losses upon momentum acceptance in the re-circulating beam transport optics \cite{adolphsen2022european}.  

PERLE: powerful energy recovery linac for experiments is the highest average electron beam power multi-turn ERL project currently being constructed, with an electron beam energy of 500~\si{\mega\electronvolt} and average electron beam current of 20~\si{\mill\ampere} (an average electron beam power of 10~\si{\mega\watt}) \cite{angal2018perle,bogacz2021perle}. PERLE -- a three turn common transport ERL -- is designed to provide a moderate energy, moderate current demonstration toward the proposed LHeC ERL project \cite{valloni2013strawman,bruning2019exploring,holzer2021accelerator}. Important ERL topics studied at PERLE will include handling a high average beam current, CW operation, low electron beam energy spread and emittance at an interaction point (IP) \cite{adolphsen2022european} necessary toward future colliders. To achieve a high average electron beam current PERLE utilises a high bunch charge of 500~\si{\pico\coulomb} \cite{hounsell2021optimization}, a factor $\sim3$ higher than previously demonstrated in an SRF ERL at the upgraded J-Lab FEL \cite{neil2006jlab}. The PERLE ERL is designed with an integrated final focus system which, with 500~\si{\mega\electronvolt} electron bunches, makes PERLE an ideal driver of a ICS source \cite{adolphsen2022european}; using an IR laser (Nd:YAG $\lambda=1064~\si{\nano\meter}$) PERLE is capable of generating up to 4.45~\si{\mega\electronvolt} $\gamma$-rays.      

Multi-turn ERLs in particular are a good choice for design of a light source because, as previously demonstrated, multi-turn ERLs can deliver high average electron beam power (\si{\mega\watt}-scale) with small emittances ($\epsilon_{n} < 1$~\si{\milli\meter}) and short bunch lengths (\si{\pico\second}-scale) which constitute high brightness electron beams. In addition, multiple re-circulations allows for higher energy electron beams necessary for small wavelength photon generation to be produced in a compact footprint, without an expensive high power accelerating section. Hence, multi-turn ERLs are investigated as drivers of light sources within this thesis.    

\section{Synchrotron Radiation Production Methods}
\label{sec:synchrotron_radiation_intro}

Synchrotron radiation is produced when the trajectory of a relativistic particle beam is curved by an applied electromagnetic field, such as a dipole magnetic field, and emits radiation. The particle beam is subject to a Lorentz force due to the applied magnetic field causing an acceleration and subsequently the emission of radiation. Viewed in the reference frame of the particle beam, the emission pattern of this radiation is the familiar dipole radiation pattern -- a toroidal radiation pattern with maximum intensity perpendicular to the applied field (parallel to the velocity of the electron). However, if the particle beam has an ultra-relativistic velocity -- the Lorentz speed factor is near unity ($\beta\sim1$) -- then in the laboratory frame the toroid is strongly deformed because of the Doppler effect and is elongated into a cone with axis parallel to the velocity of the particle \cite{ternov1995synchrotron}. Emission intensity in both the relativistic and non-relativistic case is shown in Fig.~\ref{fig:synchrotron_radiation_diagram}. Synchrotron radiation within an accelerator context is described in further detail in Section~\ref{sec:synchrotron_facility_comparison}.

\begin{figure}[!h]
\centering
\includegraphics[width=0.8\textwidth]{Figures/Introduction/Synchrotron_Radiation_Diagram.pdf}
\caption{Diagram of the emitted synchrotron radiation fields produced by a charge moving at relativistic velocity $v$ ($\beta\sim 1$)subject to a constant dipole magnetic field causing the electrons to traverse a curved trajectory. Reproduced and modified from \cite{eberhardt2015synchrotron}. Left: Toroidal field observed in the electron reference frame. Right: Elongated cone field, with opening angle $\theta = 1/\gamma$, observed in the laboratory frame.}
\label{fig:synchrotron_radiation_diagram}
\end{figure}
% first demonstration + description of process
% First dedicated experiments (2nd Gen)

\subsection{Synchrotron Radiation Sources}

Synchrotron radiation produced from an accelerator was first observed experimentally in 1947 at the GE synchrotron, New York, USA \cite{elder1948radiation} and the first classical description of synchrotron radiation was published by Schwinger in 1949 \cite{schwinger1949classical}, after experimental verification using the GE synchrotron. Synchrotron radiation is readily produced by the bending forces in accelerator magnets with wavelength inversely proportional to the magnetic flux density of the applied magnetic field; since the strength of magnets can be adjusted a large range of photon wavelengths can be produced. As the utility of synchrotron radiation was understood, numerous experiments were conducted parasitically such as the x-ray spectroscopic measurements of Berylium K-edges with the Cornell 320~\si{\mega\electronvolt} synchrotron \cite{johnston1954absorption}. The first dedicated accelerator for production of synchrotron radiation, known as a 2nd generation light source, was Tantalus I \cite{rowe1973tantalus}, first operated in 1968. Similar facilities were constructed across Europe in the following 10--15 years, with increased electron energy for access to higher photon energies, such as the worlds first high energy Synchrotron Radiation Source (SRS) at Daresbury \cite{munro2019fifty,robinson1981experiments}. 

% why is it useful in comparsion to other sources
Synchrotron radiation is useful experimentally, particularly at x-ray wavelengths, because significantly higher x-ray fluxes (no. photons) are produced than in conventional Bremsstrahlung generation such as x-ray machines typically used in medicine. For example the SPring-8 synchrotron radiation source has a maximum average brilliance of $10^{20}$~ph/\si{\second} \si{\milli\meter}$^{2}$--\si{\milli\radian}$^{2}$ 0.1\% BW \cite{spring8beamlines} whereas $10^{10}$~ph/\si{\second} \si{\milli\meter}$^{2}$--\si{\milli\radian}$^{2}$ 0.1\% BW is possible from modern bremsstrahlung based x-ray tubes \cite{behling2018diagnostic}. Short duration radiation pulses are required for investigation of temporally varying phenomena, such as most functional processes in structural biology which occur on picosecond to nano-second scales \cite{burnett2020uk}. Synchrotron radiation can satisfy this demand as 10's~\si{\pico\second} x-ray pulses are readily available at synchrotron light sources, for example the SPring-8 synchrotron radiation source produces an x-ray pulse duration of 32~\si{\pico\second} at 14~\si{\kilo\electronvolt} \cite{tanaka2001field}. Comparably, a 2~\si{\milli\second} x-ray pulse duration is readily achieved x-ray tubes \cite{behling2018diagnostic}. Synchrotron radiation wavelengths may also be tuned via variation of the magnet strength and emission angles $\theta$ of synchrotron radiation are also small, with radiation produced in a cone of opening angle $1/\gamma$. For an 8~\si{\giga\electronvolt} electron beam the opening angle of the radiation is $\sim 64$~\si{\micro\radian}, producing a small radiation spot size downstream on-sample.        

% Monochromation + why we want brilliance not flux
High brilliance is desired by many synchrotron facility users, such as those conducting spectroscopy and crystallogrphy experiments, because these experiments monochromate the produced synchrotron radiation pulse. Monochromation involves using Bragg diffraction from perfect (defectless) crystals to select a narrow bandwidth (small energy spread in the produced radiation) of the synchrotron radiation pulse. The diffraction angle is wavelength (energy) dependent therefore a narrow bandwidth can be selected via optics post monochromation. Further details on monochromation of synchrotron radiation are explained by Caciuffo et al \cite{caciuffo1987monochromators}. A small radiation spot size and angular divergence upon the monochromator is required so the radiation pulse impinges on the monochromator with similar angle irrespective of position in the radiation pulse. Therefore, maximising the synchrotron radiation flux to the monochromator -- dependent on radiation pulse spot size and divergence -- is ideal for synchrotron radiation users, and brilliance is a measure of this. Brilliance of a radiation beam can only be improved via improvements in the electron beam -- the source of the radiation -- which is readily achieved by decreasing the emittance of the electron beam. Therefore, storage rings were designed to produce lower emittance electron beams by using optics solutions such as the double bend achromat design proposed by Chasman and Green \cite{chasman1975preliminary}.          

% 3rd Generation light sources
% undulators + wigglers 
% GeV scale km circumference machines
Synchrotron radiation facilities were further advanced by the emergence of lower emittance designs, higher electron beam energies and the advent of insertion devices such as wigglers and undulators -- with these advancements the synchrotron radiation sources are termed 3rd generation light sources. Undulators consist of a straight accelerator beamline populated by permanent magnets of varying polarity and magnetisation direction. In their simplest and most common form, the planar undulator, this consists of two perpendicular periodic series of alternating polarity vertical dipole magnets, as shown in Fig.~\ref{fig:planar_undulator}. Planar undulators cause the electron bunch to travel on a sinusoidal oscillatory trajectory with synchrotron radiation emitted in each bend. The radiation produced by each subsequent oscillatory period of the undulator overlaps spatially creating a high brilliance fundamental radiation peak and, due to the short bend, is monochromatic. Wiggler insertion devices operate via a similar principle, however these use higher field magnets with fewer periods, resulting in a high flux continuous spectrum of radiation which can achieve smaller wavelengths (higher energies). 

\begin{figure}[!h]
\centering
\includegraphics[width=\textwidth]{Figures/Introduction/Planar_Undulator.pdf}
\caption{Diagram of a planar undulator with magnet gap $g$ and undulator period $\lambda_{u}$ with alternating polarity magnets (red, yellow). The electrons to follow an oscillatory trajectory (blue) within the planar undulator field, which causes the emission of synchrotron radiation (red). Throughout the undulator the self-interaction of the electron beam with the synchrotron radiation produces a microbunched electron beam and coherent radiation. Reproduced from Kentenoglu and Yavas \cite{ketenoglu2010asynchronously}.}
\label{fig:planar_undulator}
\end{figure}

The first undulator, was demonstrated on a linac in 1953 at Stanford university \cite{motz1953experiments}, then subsequently demonstrated in straight sections of the LPI RAS synchrotron light source in 1977 \cite{bessonov2010light}.  However, as undulators require a large number of magnets, undulators could not be practically implemented until the design of permanent magnet undulators by Hallbach et al \cite{halbach1983permanent}. Undulators were then widely utilised in synchrotron sources. Similarly, wigglers were demonstrated in 1979 \cite{berndt1979initial} but became favoured devices for synchrotron production at the shortest wavelengths through use of high field superconducting magnets. For example, a 10~\si{\Tesla} wiggler was demonstrated with an 8~\si{\giga\electronvolt} electron beam at SPring-8, generating \si{\mega\electronvolt}-scale $\gamma$-rays \cite{soutome2003generation}.    

\subsection{Free Electron Lasers}

% Free electron lasers
An undulator alone produces incoherent radiation, where different electrons in the bunch radiate at varying times, consequently the radiation pulse duration from an undulator can be large and the power emitted is low. However, the power of undulator radiation can be increased via coherent enhancement, where constructive interference occurs from consecutively emitted coherent undulator radiation. Coherent radiation involves simultaneous production of radiation from a group of charged particles (typically electrons) and can be produced from an undulator when the longitudinal distribution of the electron beam is matched to the wavelength of the produced radiation (i.e. the electron bunch length and emitted wavelength are identical). Achieving this longitudinal distribution can be challenging, but the electron bunch can be microbunched where the accelerator electron bunch is subdivided into smaller units such that the emission wavelength and electron bunch length match. When coherent radiation production in an undulator is enabled this is termed lasing as the electron beam acts as a gain medium similar to that of a conventional laser \cite{brau1988free}, hence this radiation production method is known as a free electron laser. The power of a free electron laser consequently increases proportionally to $N^{2}$, where $N$ is the number of electrons traversing the undulator \cite{pellegrini2016physics}. Microbunching also enables the production of very short wavelength radiation to the femtosecod scale and below. Therefore, a free electron laser provides a monochromatic, high power temporally and spatially coherent radiation pulse to users.

The required microbunched electron beam longitudinal distribution for free electron lasing can be achieved in numerous ways, but is typically achieved by self amplification by spontaneous emission (SASE) or seeding approaches. In SASE operation, self-interaction between the electron bunch and the generated undulator radiation causes energy exchange between the radiation field and the electron bunch which, with many electron bunch--radiation filed interactions, drives microbunching of the electron bunch \cite{kondratenko1980generating,bonifacio1984collective}. The microbunched electron beam has the required longitudinal distribution and therefore coherent emission occurs. However, temporal coherence is compromised because the SASE process is started randomly from `shot noise' i.e the microbunching is generated by the incoherent undulator radiation, though this can be overcome via monochromating the radiation pulse before it is used to seed lasing in a process known as self seeding, such as the diamond monochromator used for self-seeding the LCLS x-ray FEL \cite{emma2010first
}. Self-interaction of the undulator radiation with the electron bunch can be achieved via use of a small undulator and a series of mirrors, known as an optical cavity \cite{petrillo2012photon}, to circulate the produced radiation for interaction or by use of a long undulator where radiation produced from the $n+1$\textit{th} interacted electron bunch interacts with the $n$\textit{th} electron bunch. A diagram of a long planar undulator used for SASE FEL generation is shown in Fig.~\ref{fig:planar_undulator}. % this is SASE - SASE with optical mirrors too (long undulator or optical mirrors) % mention starting from noise? - worse temporal coherence?

% Seeding - limited by available seeds i.e no x-ray seeding possible (need initial coherent source)
Self-interaction can be seeded, by using an initial, often low power, external source of coherent radiation such as a conventional laser which initially microbunches the electron beam \cite{allaria2012highly}. A coherent seed means the produced radiation is also coherent, and avoids the temporal coherence issues associated with SASE methods. Higher harmonics of the initial seeding laser may be used in order to generate higher wavelength radiation. However, seeded FELs are fundamentally limited by a lack of coherent sources beyond the \si{\nano\meter}-scale. The first free electron laser was demonstrated at Stanford University in by Deacon et al \cite{deacon1977first} in 1977, with infrared lasing (coherent photon production) at a wavelength of $\lambda = 3.41$~\si{\micro\meter} and 7~\si{\kilo\watt} peak power. The initial Stanford FEL experiment was an example of a seeded FEL. The lowest wavelength seeded FEL demonstrated to date is the FERMI@ELETTRA FEL \cite{allaria2012highly} with a minimum wavelength of $\sim10$~\si{\nano\meter}; seeded FELs are yet to probe the x-ray regime. The first x-ray FEL LCLS was first demonstrated in 2010 at SLAC, with a minimum wavelength of 1.2~\si{\angstrom}, a sub pico-second pulse duration (500--10~\si{\femto\second}), a 0.2--0.5\% FWHM bandwidth and up to 40~\si{\giga\watt} peak power \cite{emma2010first}.

\subsection{Comparison of Synchrotron Radiation Sources}

The peak brilliance -- a measure of the photons produced per pulse duration per unit area in phase space within a small 0.1\% energy spread (bandwidth) (see Chapter~\ref{Photon_Production_by_Inverse_Compton_Scattering}) -- of a series of world-leading light sources are shown in Fig.~\ref{fig:light_source_tuning_curves}. Brilliance is often termed brightness, both are identical quantities, however the convention of brightness for electrons and brilliance for photons is used in this thesis. 

\begin{figure}[!h]
\centering
\includegraphics[width=0.6\textwidth]{Figures/Introduction/Light_Source_Brilliance_Energy.pdf}
\caption{Peak brilliance--photon energy tuning curves of a collection of existing x-ray light source facilities, showing both free electron lasers and synchrotron light sources \cite{geloni2017physics}. Note that brilliance is termed brightness here, whilst the two are identical brightness is reserved for discussion of electrons in this thesis.}
\label{fig:light_source_tuning_curves}
\end{figure}

Ultimately, as demonstrated in Chapter~\ref{CBETA_Inverse_Compton_Scattering_Source_Design}, practical considerations such as size and magnetic field strength limit synchrotron radiation facilities and free electron lasers to x-ray production. For development of a $\gamma$-ray source, methods based on synchrotron or undulator radiation fail to produce the \si{\mega\electronvolt}-scale photons desired by nuclear physics experiments \cite{budker2021expanding}. However, synchrotron radiation facilities and FELs are the dominant radiation production methods in the x-ray regime and below, with unparalleled photon flux and brilliance.  

\section{Bremsstrahlung Radiation Production}
\label{sec:bremsstrahlung}
% Hywels lecture notes give a good brief overview of Brem (in the chapter comments directory) + have the review he gave me
% Duane - Hunt law
% Continuous spectrum - hard to monochromate - why no gamma monochromators?
% High Z good, why? High field for breaking
% bit of history? x-ray generation?

In the bremsstrahlung process, which translates to braking radiation, a charged particle traverses within the vicinity of an atomic nucleus and the strong electric field of the atomic nuclei acts on the charged particle causing an acceleration, similar to the transversely applied magnetic field in synchrotron radiation, and therefore the charged particle radiates. A more comprehensive description is presented in Section~\ref{sec:bremsstrahlung} and a full quantum electrodynamic description of bremsstrahlung has been derived by Bethe and Heitler \cite{bethe1934stopping}. A full review of the bremsstrahlung process is beyond the scope of this thesis, and the reader is directed to more comprehensive reviews such as that by Koch and Motz \cite{koch1959bremsstrahlung}. 

Typically, in a bremsstrahlung source a moderate energy electron beam (10-100's~\si{\mega\electronvolt}) is interacted with a dense target of high $Z$ material, with photons generated in the direction of the particle beam. The spectrum of radiation produced by a bremsstrahlung interaction is continuous and broadband, with the maximum photon frequency $\nu$ produced given by the Duane--Hunt law \cite{duane1915proceedings} $E_{k}=h\nu$ where $E_{k}$ is the kinetic energy of the charged particle and $h$ is Planck's constant. Therefore, with moderate energy \si{\mega\electronvolt}-scale particle beams $\gamma$-rays can be readily generated. Again, as in synchrotron radiation, when highly relativistic electron beams are used, the resultant photons are produced into an angular cone of opening angle $\theta\sim 1/\gamma$, with $\gamma$ the Lorentz factor \cite{chao2013handbook}, however there is no emission angle--photon energy correlation, therefore simple collimation is insufficient for energy selection. Monochromation of x-ray radiation is feasible, as explained in Section~\ref{sec:synchrotron_radiation_intro}, but efficient monochromation of $\gamma$-rays has not yet been demonstrated -- currently, 2~\si{\mega\electronvolt} $\gamma$-rays have been monochromated with an efficiency of 22\% \cite{jentschel2012gamma}. 

The power of the generated bremsstrahlung radiation increases with increasing atomic number $Z$ because the electric field strength increases as a function of $Z$. The density of the target also increases the power of the bremsstrahlung source due to the increased probability of interactions in a thick target. However, increasing the flux of a bremsstrahlung source can also be increased via maximising the electron beam current impinging upon the high-$Z$ target, through this causes heating of the target material and thus for high flux bremsstrahlung targets (converters) water cooling is required \cite{auslender2004bremsstrahlung}. High fluxes of both x-rays and $\gamma$-rays are available using a high-$Z$ water cooled target, for example the ARIEL project \cite{dilling2013ariel} where a 50~\si{\mega\electronvolt} electron linac and a tungsten target are used to generate up to $10^{14}$ $\gamma$-rays per second \cite{lebois2011simulations}. 

Bremsstrahlung radiation production is widely used for example in x-ray production for medical x-rays, from the first medical x-ray usage by Jones and Lodge in 1986 \cite{jones1896discovery} to computed tomography systems \cite{hounsfield1973computerized,cormack1963representation,cormack1964representation} and modern radiography, with higher flux sources on the order of $10^{10}$--$10^{12}$ ph/\si{\second} \cite{behling2018diagnostic}. Generation of $\gamma$-rays via bremsstrahlung radiation has also had a significant impact upon the study of nuclear physics. High fluxes of $\gamma$-rays available using bremsstrahlung have enabled experiments such as detection of clandestine nuclear material \cite{pruet2006detecting,jones2008bremsstrahlung} and photofission cross section determination nuclear structure experiments \cite{dickey1975u,naik2011mass} and photonuclear medical isotope production \cite{danon2008medical}.

\section{Inverse Compton Scattering Sources}
\label{sec:intro_ICS}
% electron photon collider
% laser undulator
% limitations of the compton cross section - flux
% pg 30 1st year report has a hand wavy double doppler classical explanation

An alternative radiation production mechanism to the more commonplace synchrotron sources and free electron laser facilities is to use the inverse Compton scattering process. The inverse Compton scattering process was first considered by Feenberg and Primakoff \cite{feenberg1948interaction} as a mechanism whereby cosmic rays are reduced in energy as they propagate throughout the universe. The cosmic rays -- relativistic charged particles -- interact with starlight to emit shorter wavelength radiation, with the concurrent reduction in energy of the relativistic charged particle. The inverse Compton process is opposite to Compton scattering \cite{compton1923quantum}, where a non-relativistic charged particle interacts with a photon, lengthening the incident photon wavelength and increasing the energy of the incident particle. An extended discussion of Compton and inverse Compton scattering is found in Section~\ref{sec:electron_photon_interactions}. 

Savedoff \cite{savedoff1959crab} and later Felten and Morrison \cite{felten1963recoil} suggested the use of inverse Compton scattering as a radiation production method, specifically for $\gamma$-rays, which are difficult to obtain due to their inherently short wavelengths (high energies). Though inverse Compton scattering can be conducted with any charged particle, electrons are favoured because their low mass allows for production of high energy radiation and consequently our discussion is limited to electrons. Radiation sources using the principle of inverse Compton scattering can be considered as electron--photon colliders or as laser undulator devices, as shown diagrammatically in Fig.~\ref{fig:ICS_laser_undulator_collision}. The electron--photon collider model involves a collision between a photon and relativistic electron where the electron recoils and the photon gains energy -- energy is transferred to the incident photon. In the laser undulator approach, the large sinusoidally varying electric fields of the incident photon pulse act upon the electron as in an undulator; the trajectory of the incident particle is oscillatory and synchrotron radiation is generated. The radiation produced from inverse Compton scattering is incoherent and coherent photon production has not been demonstrated using ICS though proposed schemes exist \cite{graves2012intense,graves2014compact,nanni2018nanomodulated}. Both models are equally valid as a consequence of wave--particle duality \cite{de1923waves}, and are used interchangeably to described particular phenomena.
\begin{figure}[!h]
\centering
\includegraphics[width=\textwidth]{Figures/Introduction/ICS_laser_undulator_collision.pdf}
\caption{Two equivalent wave and particle models of inverse Compton scattering. Left: Laser undulator model. The trajectory of an electron bunch (red) becomes oscillatory due to transverse acceleration by the electric field of a counter propagating incident laser pulse (blue). The transverse oscillation results in the emission of high energy photons (orange), as in an undulator. Right: Photon--Electron collider model. An incident electron bunch (red) collides with an incident laser pulse (blue) resulting in the backscattering of high energy photons (orange) causing the incident electron bunch to recoil and decelerate.}
\label{fig:ICS_laser_undulator_collision}
\end{figure}

Classically the inverse Compton scattering process can be considered as a double Doppler shift of the incident photon. Firstly, the incident photon must be Lorentz transformed into the frame of the electron bunch via a relativistic Doppler shift
\begin{equation}
f'=\gamma\left(1+\beta\cos\phi\right)f,
\label{eq:Doppler_shift}    
\end{equation}
where $f'$ is the frequency of the incident photon in the electron frame, $f$ is the frequency of the incident photon in the laboratory frame, $\phi$ is the crossing angle between the electron bunch and the laser pulse and $\beta = v/c$. If the electron bunch is ultra-relativistic ($\beta\rightarrow 1$) and the interaction between the electron bunch and laser pulse is head-on ($\phi$=0), the Doppler shift reduces to $f'\approx2\gamma f$. The Lorentz transformation must again be applied to transform the photon from the electron frame back to the laboratory frame via another Doppler shift yielding  
\begin{equation}
f'' \approx 2\gamma^{2}\left(1+\beta\cos\phi\right)f, 
\label{eq:2nd_Doppler_shift}    
\end{equation}
resulting in the frequency, and consequently energy relation
\begin{align}
f'' \approx 4\gamma^{2}f, \nonumber \\
E_{\gamma} \approx 4\gamma^{2}E_{L}
\end{align}
where $E_{\gamma} = hf''$ is the energy of the generated photon and $E_{L} = hf$ is the incident photon energy. Therefore, moderate energy particle beams can generate high energy photons from modest incident photon energies. A full quantum derivation of this relationship is shown within Chapter~\ref{Photon_Production_by_Inverse_Compton_Scattering}. 

The double Doppler shift ($E_{\gamma}\propto\gamma^{2}$) of the incident radiation allows for the generation of high energy radiation from a moderate electron beam energy. For example, 1~\si{\mega\electronvolt} $\gamma$-ray can be produced using inverse Compton scattering by a typical $\lambda = 1$~\si{\micro\meter} infrared laser and an electron beam with a kinetic energy of $\sim 230$~\si{\mega\electronvolt}. Comparatively, taking the parameters of the LCLS-II hard x-ray undulator
(undulator period $\lambda_{u} = 26$~\si{\milli\meter}, magnet flux density $B = 1.01$~\si{\tesla}) \cite{wallen2016status} with the same 230~\si{\mega\electronvolt} electron beam yields a photon energy of 4.81~\si{\electronvolt} from the fundamental harmonic -- orders of magnitude below what is available with a 'laser undulator'. Therefore, ICS is a premier method of generating sub-angstrom wavelength radiation. 

As a radiation production method, a major drawback of the inverse Compton scattering reaction is the low probability of the electron--photon collision reflected in the Klein--Nishina cross section \cite{klein1929streuung}, which typically has a value close to the Thomson cross section ($\sigma_{T} = 0.665$~\si{\barn}). Therefore, generation of large quantities of radiation is challenging. However, the ICS interaction is beneficial because, as demonstrated in Section~\ref{sec:electron_photon_interactions}, there is an energy--angle correspondence in the produced radiation which allows certain energy photons to be selected by a simple collimator. An energy--angle correspondence does not exist in synchrotron or bremsstrahlung radiation production, therefore this is a unique feature of inverse Compton scattering. Use of inverse Compton scattering as a radiation production method is further explored in Chapter~\ref{Photon_Production_by_Inverse_Compton_Scattering}. 

\section{Thesis Layout and Scope}
\label{sec:thesis_layout_scope}

The thesis is concerned with the investigation of high-energy radiation (x-ray and $\gamma$-ray) production via the interaction of ultra-relativistic electron beams with laser pulses -- known as inverse Compton scattering -- where the electron bunch is generated using an energy recovery linac. Therefore, the thesis has three main foci: 
\begin{enumerate}
    \item{Possible designs of ICS sources as an application of ERLs, the efficacy of the ERL driven approach in comparison to other accelerator types and ERL driven ICS source advantages for operation of x-ray and $\gamma$-ray light sources.}
    \item{The optimum configuration of the electron beam for operation of ICS sources as high flux, narrow bandwidth light sources most favoured by experimentalists.}
    \item{Identification of areas where ICS sources are most applicable in comparison to other light source facility types and the experiments ICS sources enable.}
\end{enumerate}

Consequently, the thesis is structured as follows. Firstly,  Chapter~\ref{Energy_Recovery_Linac_Design} presents an overview of the beam dynamics considerations for design of an ERL based ICS source, then the theory relevant to characterisation and design of an inverse Compton scattering source is presented in Chapter~\ref{Photon_Production_by_Inverse_Compton_Scattering}. Developed methods for optimisation and characterisation of ICS sources are explained and demonstrated in Chapter~\ref{Optimisation_and_Characterisation_of_Inverse_Compton Scattering_Spectra}. Improvements in the characterisation of an ICS source are made via the derivation of an analytical calculation for the flux of an ICS source post-collimation and creation of a semi-analytical spectrum code named \textsc{ICARUS}. A series of optimisations of transverse electron bunch parameters for high flux, narrow bandwidth radiation are produced. The methods developed in Chapter~\ref{Optimisation_and_Characterisation_of_Inverse_Compton Scattering_Spectra} are applied to designs of ERL driven ICS sources in the rest of the thesis. In Chapter~\ref{CBETA_Inverse_Compton_Scattering_Source_Design} an x-ray ERL driven source design based upon CBETA -- the worlds first multi-turn SRF ERL -- is produced, characterised and compared with other competing x-ray sources. Potential applications of such a source are explained. Understanding of the application of ICS sources to x-ray production then led to design of $\gamma$-ray ICS source utilising the conceptual DIANA ERL in Chapter~\ref{DIANA_Inverse_Compton_Source_Design}. The DIANA $\gamma$-ray ICS source design is optimised, characterised and compared to Bremsstrahlung $\gamma$-ray production with several applications of such a source investigated. Finally, in Chapter~\ref{Conclusion} the main findings throughout the thesis are re-iterated and future work relevant to ERL driven ICS sources is discussed.       

\end{document}