%%%%%%%%%%%%%%%%%%%%%%%%%%%%%%%%%%%%%%%%%%%%%%%%%%%%%%%%%%
%
% Thesis Template @ The University of Manchester
% LaTeX Template
% Version 1 (24/08/2018)
% Joe Crone
%
% This template is based on:
% The University of Manchester, Presentation of Thesis Policy
% Research Office Graduate Education Team
% June 2017
% http://www.regulations.manchester.ac.uk/pgr-presentation-theses/
%
%%%%%%%%%%%%%%%%%%%%%%%%%%%%%%%%%%%%%%%%%%%%%%%%%%%%%%%%%%



% Document Class
\documentclass[11pt,oneside]{thesisformat}

% Packages
\usepackage{subfiles}

% Thesis Information
\title{Thesis Template, The University of Manchester}
\def\thesistitle{Design and Optimisation of High-Energy Inverse Compton Scattering Sources Driven by Multi-Pass Energy Recovery Linacs}
\def\name{Joe Crone}
\def\faculty{Faculty of Science and Engineering}
\def\school{School of Natural Sciences}


%%%%%%%%%%%%%%%%%%%%%%%%%%%%%%%%%%%%%%%%%%%%%%%%%%%%%%%%%%

\begin{document}

% Title Page
%--------------------------------------------------------
\maketitle


% Contents Page (+ word count)
%--------------------------------------------------------
\tableofcontents
\vfill
\begin{center}
\quickwordcount{main}
\end{center}
\vfill

% List of Tables
%--------------------------------------------------------
\listoftables


% List of Figures
%--------------------------------------------------------
\listoffigures


% List of Abbreviations
%--------------------------------------------------------

% [optional] Lay Abstract
%--------------------------------------------------------
%\chapter*{Lay Abstract}
%\addcontentsline{toc}{chapter}{Lay Abstract}


% Declaration
%--------------------------------------------------------
\chapter*{Declaration}
\addcontentsline{toc}{chapter}{Declaration}

No portion of the work referred to in the thesis has been submitted in support of an application for another degree or qualification of this or any other university or other institute of learning.


% Copyright Statement
%--------------------------------------------------------
\chapter*{Copyright Statement}
\addcontentsline{toc}{chapter}{Copyright Statement}

\begin{enumerate}[(i)]
\item The author of this thesis (including any appendices and/or schedules to this thesis) owns certain copyright or related rights in it (the  “Copyright”) and s/he has given The University of Manchester certain  rights to use such Copyright, including for administrative purposes.
\item Copies of this thesis, either in full or in extracts and whether in hard or electronic copy, may be made only in accordance with the Copyright, Designs and Patents Act 1988 (as  amended) and regulations issued under it or, where appropriate, in accordance with licensing agreements which the University has from time to time. This page must form part of any such copies made.
\item The ownership of certain Copyright, patents, designs, trademarks and other intellectual property (the “Intellectual Property”) and any reproductions of copyright works in the thesis, for example graphs and tables (“Reproductions”), which may be described in this thesis, may not be owned by the author and may be owned by third parties. Such Intellectual Property and Reproductions cannot and must not be made available for use without the prior written permission of the owner(s) of the relevant Intellectual Property and/or Reproductions.
\item Further information on the conditions under which disclosure, publication and commercialisation of this thesis, the Copyright and any Intellectual Property  and/or Reproductions described in it may take place is available in the University IP Policy (see \url{http://documents.manchester.ac.uk/DocuInfo.aspx?DocID=24420}), in any relevant Thesis restriction declarations deposited in the University Library, The University Library’s regulations (see \url{http://www.library.manchester.ac.uk/about/regulations/}) and in The University’s policy on Presentation of Theses.
\end{enumerate}

% Abstract
%--------------------------------------------------------
\chapter*{Abstract}
\addcontentsline{toc}{chapter}{Abstract}

% TOO LONG!

% Little bit about ICS sources - why offer narrowband radiation, why offer high photon energy, why limited in flux, why use electrons? 

High quality x-ray sources are required for fundamental research in atomic physics and material science. Third generation synchrotron light sources fulfil this need in most aspects, producing a high flux with x-ray energies up to 100's~\si{\kilo\electronvolt} and a narrow bandwidth . However, the maximum photon energy produced by synchrotrons is limited by facility size, electron beam energy and magnet strength constraints. Hence, in this thesis an inverse Compton scattering (ICS) source has been designed for production of high energy x-rays ($E_{\gamma} \leq 402.5$~\si{\kilo\electronvolt}) from the CBETA multi-turn energy recovery linac (ERL), with high flux ($\mathcal{F} = 3.22\times 10^{10}$~ph/\si{\second}) and narrow bandwidth ($\Delta E_{\gamma}/E_{\gamma} = 0.5$\% \textit{rms}).   

Similarly, high quality $\gamma$-ray sources ($E_{\gamma} > 1$~\si{\mega\electronvolt}) are in demand for experimentation in nuclear photonics, photonuclear radioisotope production, nuclear forensics and proliferation. Whilst bremsstrahlung and radioisotope $\gamma$-ray sources could be used, they are not ideal as bremsstrahlung is inherently broadband and radioactive isotopes produce a low flux. Currently ICS sources, such as HI$\gamma$S, produce $\gamma$-rays up to 100~\si{\mega\electronvolt} with high photon fluxes ($\mathcal{F}=5\times 10^{8}$~ph/\si{\second}), however the bandwidth ($\Delta E_{\gamma}/E_{\gamma} = 2.5$\% FWHM) is too large for some experiments. Hence, the DIANA ERL driven ICS source is designed to provide narrowband ($\Delta E_{\gamma}/E_{\gamma} = 0.5$\% \textit{rms}) $\gamma$-ray production ($E_{\gamma} \leq$ 20.11) at higher flux ($\mathcal{F} = 6.08\times 10^{10}$).  

Various accelerators can provide electron beams to drive ICS sources, though large scattered photon fluxes require a high average electron beam current and small emittance, whilst narrow bandwidths require small emittance and small electron beam energy spreads. Therefore, this thesis develops optimisation methods for ICS production of narrow bandwidth photons at high flux. Currently, most ICS sources utilise storage rings, with high average beam current and moderate electron bunch energy spread, or linacs, with small emittance and energy spread. ERLs can provide electron beams with small emittance, energy spread and high average beam current simultaneously. Hence ERLs are ideal drivers of ICS sources.      

In this thesis two ICS sources are designed: the CBETA x-ray ICS source and the DIANA $\gamma$-ray ICS source. Methods for predicting the flux and the produced photon spectrum are developed and a series of optimisations toward maximal narrowband photon production are proposed. Applications of the produced narrow-band, high energy photons are then investigated for x-rays and $\gamma$-rays and photon production from ERL driven ICS sources is compared with other light sources.   

% [optional] Dedication / Acknowledgement
%--------------------------------------------------------
\chapter*{Acknowledgement}
\addcontentsline{toc}{chapter}{Acknowledgement}



% [optional] Preface / 'The Author' - for the external examiner
%--------------------------------------------------------
%\chapter*{Preface}
%\addcontentsline{toc}{chapter}{Preface}


% Introduction
%--------------------------------------------------------
\subfile{Chapters/Introduction}

% Part 1:
%--------------------------------------------------------
% Theory of Energy Recovery Linac Design, \subfile only likes single word .tex names
\subfile{Chapters/Energy_Recovery_Linac_Design}

% Part 2:
%--------------------------------------------------------
% Photon Production by Inverse Compton Scattering
\subfile{Chapters/Photon_Production_by_Inverse_Compton_Scattering}


% Part 3:
%--------------------------------------------------------
% Optimisation and Charaterisation of Inverse Compton Scattering Spectra
\subfile{Chapters/Optimisation_and_Characterisation_of_Inverse_Compton_Scattering_Spectra}


% Part 3:
%--------------------------------------------------------
% CBETA Multipass Commissioning
%\subfile{Chapters/CBETA_Multi-Pass_Commissioning}

% Part 4:
%--------------------------------------------------------
% CBETA ICS Design
\subfile{Chapters/CBETA_Inverse_Compton_Source_Design}

% Part 5:
%--------------------------------------------------------
% DIANA ICS Design
\subfile{Chapters/DIANA_Inverse_Compton_Source_Design}



% Conclusion
%--------------------------------------------------------
\subfile{Chapters/Conclusion}

% Bibliography
%--------------------------------------------------------
%\addcontentsline{toc}{chapter}{Bibliography}
\printbibliography
\bibliographystyle{unsrt}
\bibliography{sample}

% Appendix
%--------------------------------------------------------
\appendix

%%%%%%%%%%%%%%%%%%%%%%%%%%%%%%%%%%%%%%%%%%%%%%%%%%%%%%%%%%
%
% Doctoral Thesis Template @ The University of Manchester
% LaTeX Appendix Template
% Version 1 (24/08/2018)
% Kathryn Fowler
%
% This template is based on:
% The University of Manchester, Presentation of Thesis Policy
% Research Office Graduate Education Team
% June 2017
% http://www.regulations.manchester.ac.uk/pgr-presentation-theses/
%
%%%%%%%%%%%%%%%%%%%%%%%%%%%%%%%%%%%%%%%%%%%%%%%%%%%%%%%%%%


% Title
%--------------------------------------------------------
\chapter{Appendix Template}
\label{AppendixTemplate} % to reference use \ref{ChapterTemplate}


% Section 1
%--------------------------------------------------------
\section{Section 1}
\subsection{Subsection 1}


% Section 2
%--------------------------------------------------------
\section{Section 2}


% Section 3
%--------------------------------------------------------
\section{Section 3}





\end{document}
